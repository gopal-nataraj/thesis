% introduction

Magnetic resonance imaging (MRI)
is a non-invasive tool
that has earned widespread clinical adoption
due (among other factors) 
to its potential for excellent soft tissue contrast,
its avoidance of ionizing radiation,
and its flexibility to characterize many physical phenomena. 
Despite its numerous advantages,
MRI requires highly specialized hardware,
ongoing liquid-helium cooling
of its superconducting main magnet,
and comparably long scan times.
For these reasons,
MRI is (somewhat inherently) expensive relative
to other medical imaging modalities.
Accordingly,
one broad initiative 
recently advocated by the MR community 
is to increase the \emph{value}
of MRI examinations.

Two reasonable measures
of an acquisition's value
are its sensitivity
to a given disorder
and its specificity
in distinguishing it
from others.
The field of \emph{pathology}
seeks to ascribe physical processes
to disorders of interest
with high sensitivity and specificity.
The field of \emph{quantitative MRI} (QMRI)
seeks to build MRI biomarkers
that measurably describe such physical processes
and thereby provide indirect information
about the onset and progression
of underlying conditions.

QMRI poses several challenges
beyond those of commonplace anatomical MRI
and thus remains yet to be widely adopted clinically.
For example, latent parameter ``maps'' 
that describe relevant physical processes
are often related to the received MR signal
through complicated, 
highly nonlinear relationships. 
Furthermore,
practical MR pulse sequences
produce signals
that are usually functions
of not only desired
but also nuisance parameters.
Scan repetition is often necessary
for accurate estimation
of multiple desired and nuisance parameters,
which can increase scan times.
Mitigating these challenges 
(and likely others)
is essential
to furthering widespread clinical adoption
of QMRI techniques.

\section{Thesis Overview}
\label{s,intro,over}

In this thesis,
we seek to build a systematic framework
towards QMRI.
We borrow tools
from optimization, statistics, and machine learning
to construct time-efficient workflows
for quantifiably characterizing
physical processes of interest.
We apply this framework
to challenging QMRI problems
that are motivated
by pathological studies.
Our goal is to introduce tools 
that aid in identifying clinical tasks 
for which QMRI should
(or should not) be part
of a targeted, high-value MRI examination.

We consider two distinct subproblems in our framework.
Questions in \emph{acquisition design}
(Chapters~\ref{c,scn-dsgn},\ref{c,mwf})
ask how to assemble 
fast collections of scans
that yield data 
rich in information 
about physical processes of interest.
Questions in \emph{parameter estimation}
(Chapters~\ref{c,relax},\ref{c,krr})
ask how to quickly and reliably quantify parameters 
associated with these relevant physical processes.
The overall framework seeks to
first design fast and informative scans 
based on the application,
and to then accurately and precisely estimate 
application-specific parameters of interest.
 
\section{Thesis Organization}
\label{s,intro,org}

The main body of this thesis is organized as follows:
\begin{itemize}
\item 
	Chapter~\ref{c,bkgrd} reviews 
	relevant background MR material
	about DESS,SPGR,blahblah...
\item 
	Chapter~\ref{c,relax} discusses methods 
	for MRI parameter estimation 
	from likelihood models 
	and applies these methods 
	for model-based MR relaxometry, 
	(\ie, estimation of relaxation parameters $\To,\Tt$),
	of interest for many neurological applications.
	It derives some content 
	(especially regarding applications)
	from conference papers 
	\cite{nataraj:14:rje,nataraj:14:mbe}.
\item
	Chapter~\ref{c,scn-dsgn} introduces
	a minimax optimization approach
	to aid robust and application-specific 
	MR scan selection and optimization 
	for precise latent parameter estimation.
	It optimizes several practical acquisitions 
	and uses the likelihood-based estimation techniques 
	introduced in Chapter~\ref{c,relax}
	to assess the utility
	of scan optimization
	through simulations, 
	phantom studies, 
	and \invivo experiments.
	It derives content
	mainly from journal paper
	\cite{nataraj::oms}
	and conference paper
	\cite{nataraj:15:amm}.
\item 
	Chapter~\ref{c,krr} describes 
	MRI parameter estimation
	using kernel ridge regression.
	It derives content 
	from conference paper
	\cite{nataraj:17:dfm}.
\item
	Chapter~\ref{c,mwf} introduces a multi-compartmental model
	for relevant MR pulse sequences
	and proposes a new acquisition 
	useful for myelin water fraction estimation,
	of interest in white matter disorders.
	It applies kernel-based MR parameter estimation
	to estimate myelin water fraction,
	in simulations and \invivo experiments.
	It derives some content from conference paper
	\cite{nataraj:17:mwf}.
\item
	Chapter~\ref{c,ss-rf} presents
	some relatively immature ideas
	on steady-state radiofrequency (RF) pulse design
	as well as associated challenges.
	This work is presently unpublished
	and may offer avenues for further development.
\item 
	Chapter~\ref{c,future} summarizes several items 
	of possible future work
	(on both short- and long-term timescales) 
	and presents a timeline
	for completion of this thesis.
\end{itemize}

The appendices are organized as follows:
\begin{itemize}
\item
	Appendix~\ref{a,cc-multi} proposes an algorithm
	for combining multiple MRI datasets
	(as is necessary for many parameter estimation problems),
	when each dataset is acquired 
	using multiple receiver coils.
\item
	Appendix~\ref{a,dess-diff} presents an analysis
	of DESS in the presence of diffusion,
	shows that neglecting diffusive effects
	during $\Tt$ estimation from DESS
	can cause significant bias,
	and suggests acquisition modifications
	for mitigating this bias.
\end{itemize}

