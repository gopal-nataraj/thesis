% introduction

Magnetic resonance imaging (MRI)
is a non-invasive tool
that has earned widespread clinical adoption
due (among other factors) 
to its potential for excellent soft tissue contrast,
its avoidance of ionizing radiation,
and its flexibility to characterize many physical phenomena. 
Despite its numerous advantages,
MRI requires highly specialized hardware,
ongoing liquid-helium cooling
of its superconducting main magnet,
and comparably long scan times.
For these reasons,
MRI is (somewhat inherently) expensive relative
to other medical imaging modalities.
Accordingly,
one broad initiative 
recently advocated by the MR community 
is to increase the \emph{value}
of MRI examinations.

Two reasonable measures
of an acquisition's value
are its \emph{sensitivity}
to a given disorder's onset and progression
and its \emph{specificity}
in distinguishing it
from others.
The field of \emph{pathology}
seeks to ascribe physical processes
to disorders of interest
with high sensitivity and specificity.
The field of \emph{quantitative MRI} (QMRI)
seeks to build measurable MRI biomarkers
that describe such physical processes
and thereby provide indirect information
about underlying conditions.

QMRI poses several challenges
beyond those of MR image reconstruction,
and thus remains yet to be widely adopted clinically.
For example, latent parameter ``maps'' 
that describe relevant physical processes
are often related to the received MR signal
through complicated, 
highly nonlinear relationships. 
Furthermore,
practical MR pulse sequences
produce signals
that are usually functions
of not only desired
but also nuisance parameters.
Scan repetition is often necessary
for accurate estimation
of multiple desired and nuisance parameters,
which can increase scan times.
Mitigating these challenges 
(and likely others)
is essential
to furthering widespread clinical adoption
of QMRI techniques.

\section{Thesis Overview}
\label{s,intro,over}


