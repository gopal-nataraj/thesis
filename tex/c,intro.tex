% introduction

Magnetic resonance imaging (MRI)
is a non-invasive tool
that has earned widespread clinical adoption
due (among other factors) 
to its potential for excellent soft tissue contrast,
its absence of ionizing radiation,
and its flexibility to characterize 
a diversity of physical phenomena. 
Despite its numerous advantages,
MRI requires highly specialized hardware,
ongoing liquid-helium cooling
of its superconducting main magnet,
and comparably long scan times.
For these reasons,
MRI is expensive relative
to other medical imaging modalities.
Accordingly,
one broad initiative 
recently advocated by the MR community 
is to increase the \emph{value}
of MRI examinations.

Two reasonable measures
of an MRI acquisition's value
are its \emph{sensitivity}
to a given disorder
and its \emph{specificity}
in distinguishing it
from others.
The field of \emph{quantitative MRI} (QMRI)
seeks to use MRI data 
to build MR \emph{biomarkers},
or measurable tissue properties
that can increase the sensitivity and specificity of MRI
for specific disorders of interest.

QMRI has potential
to be more informative than conventional MRI.
Conventional MRI is \emph{qualitative}:
it produces images comprised of voxels
(\ie, three-dimensional pixels)
that are \emph{informative only relative to each other},
not individually.
Conventional MRI voxels are qualitative
because they directly localize the MR signal,
a typically complex function
of not only biomarkers
but also two types of confounds:
\emph{nuisance markers}
that characterize undesired signal sources
and/or MRI system imperfections;
and \emph{acquisition parameters}
that characterize the MRI system's tunable ``knobs''. 
QMRI seeks to remove confound influence
by instead imaging the biomarkers directly.
Each QMR image voxel
is thus a measurement 
of a given biomarker
at a specific location.
QMRI can therefore provide localized biomarker measurements
(\eg, myelin water content)
related to a specific physiological process
(\eg, demyelination)
that can, through longitudinal study,
be used to monitor the onset and progression of disease
(\eg, multiple sclerosis).

QMRI poses several challenges
beyond those of conventional MRI 
that currently limit its feasibility 
for routine clinical use.
For example,
accurate biomarker quantification 
traditionally requires multiple MR scans
and thus long scan times.
Furthermore,
it has previously been unclear
how to tune acquisition parameters 
of these multiple scans
to ensure that biomarkers can be quantified precisely.
Finally, MR biomarker quantification
is a challenging estimation problem
for which efficient algorithms
have previously been unavailable.
Addressing these challenges is essential
for widespread clinical adoption of QMRI.

\section{Thesis Overview}
\label{s,intro,over}

This thesis seeks
to address the above challenges
by building an automated workflow for QMRI.
We borrow tools
from optimization, statistics, and machine learning
to develop fast workflows
for quantifying biomarkers
that characterize specific physiological processes. 
We apply this framework
to challenging QMRI problems 
of clinical interest.
Our goal is to introduce fast, automated tools
that will increase the clinical value of QMRI.

Our solutions to two distinct subproblems in QMRI
constitute two stages of our proposed QMRI workflow.
Questions in \emph{acquisition design}
(Chapters~\ref{c,scn-dsgn}, \ref{c,mwf})
ask how to assemble 
fast collections of scans
that yield data 
rich in information 
about physical processes of interest.
Questions in \emph{parameter estimation}
(Chapters~\ref{c,relax}, \ref{c,krr})
ask how to quickly and reliably quantify biomarkers 
associated with these relevant physical processes.
The overall workflow seeks to
first design fast and informative scans 
based on the application,
and to then accurately and precisely estimate 
clinically relevant biomarkers.
 
\section{Thesis Proposal Organization}
\label{s,intro,org}

The main body of this thesis proposal is organized as follows:
\begin{itemize}
\item 
	Chapter~\ref{c,bkgrd} reviews 
	relevant background material
	on MRI and optimization.
\item 
	Chapter~\ref{c,relax} discusses methods 
	for MRI parameter estimation 
	from likelihood models 
	and applies these methods 
	to model-based MR relaxometry, 
	(\ie, estimation of relaxation parameters $\To,\Tt$),
	of interest for many neurological applications.
	It derives some content 
	(especially regarding applications)
	from conference papers 
	\cite{nataraj:14:rje,nataraj:14:mbe}.
\item
	Chapter~\ref{c,scn-dsgn} introduces
	a minimax optimization approach
	to aid robust and application-specific 
	MR scan selection and optimization 
	for precise latent parameter estimation.
	It optimizes several practical acquisitions 
	and uses the likelihood-based estimation techniques 
	introduced in Chapter~\ref{c,relax}
	to assess the utility
	of scan optimization
	through simulations, 
	phantom studies, 
	and \invivo experiments.
	It derives content
	mainly from journal paper
	\cite{nataraj::oms},
	which substantially extends conference paper
	\cite{nataraj:15:amm}.
\item 
	Chapter~\ref{c,krr} describes 
	scalable MRI parameter estimation
	using kernel ridge regression.
	It derives some content 
	from conference paper
	\cite{nataraj:17:dfm}.
\item
	Chapter~\ref{c,mwf} studies multi-compartmental models
	for relevant MR pulse sequences
	and proposes a new acquisition 
	useful for myelin water fraction estimation,
	of interest in white matter disorders.
	It applies kernel-based MR parameter estimation
	to estimate myelin water fraction,
	in simulations and preliminary \invivo experiments.
	It derives some content from conference paper
	\cite{nataraj:17:mwf}.
\begin{comment}
\item
	Chapter~\ref{c,ss-rf} presents
	some relatively immature ideas
	on steady-state radiofrequency (RF) pulse design
	as well as associated challenges.
	This work is presently unpublished
	and may offer avenues for further research.
\end{comment}
\item 
	Chapter~\ref{c,future} summarizes several items 
	of possible future work
	(on both short- and long-term timescales) 
	and presents a timeline
	for completion of this thesis.
\end{itemize}

In the thesis, the appendices will be organized as follows:
\begin{itemize}
\item
	Appendix~\ref{a,cc-multi} will propose an algorithm
	for combining multiple MRI datasets
	(as is necessary for many parameter estimation problems),
	when each dataset is acquired 
	using multiple receiver coils.
\item
	Appendix~\ref{a,dess-diff} will present an analysis
	of model mismatch due to the presence of diffusion,
	will show that neglecting diffusive effects
	during $\Tt$ estimation 
	can cause significant bias,
	and will suggest acquisition modifications
	for mitigating this bias.
\end{itemize}
