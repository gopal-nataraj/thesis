% introduction

Magnetic resonance imaging (MRI)
is a non-invasive tool
that has earned widespread clinical adoption
due (among other reasons) 
to its potential for excellent soft tissue contrast,
its absence of ionizing radiation,
and its flexibility to characterize 
a diversity of physical phenomena. 
Despite its numerous advantages,
MRI requires highly specialized hardware,
ongoing liquid-helium cooling
of its superconducting main magnet,
and long scan times.
For these reasons,
MRI is expensive relative
to other medical imaging modalities.
To better focus expenditures,
one broad initiative advocated by the MR community 
is to increase the \emph{value}
of MRI examinations.
One popular (and ambitious) measure of value
is an MRI acquisition's specificity
in distinguishing one disorder
from a collection of candidates.
The field of quantitative MRI (QMRI)
seeks to estimate MR \emph{biomarkers},
or measurable tissue properties
that may be indicative 
of specific disorders of interest.

QMRI has potential
to be more informative than conventional MRI.
Conventional MRI is qualitative:
it produces images comprised of voxels
(\ie, three-dimensional pixels)
that are informative only relative to each other,
not individually.
Conventional MRI voxels are qualitative
because they directly localize the MR signal,
a typically complex function
of not only biomarkers
but also two types of confounds:
\emph{nuisance markers}
that characterize undesired signal sources
and/or MRI system imperfections;
and \emph{acquisition parameters}
that characterize the MRI system's tunable ``knobs''. 
QMRI seeks to remove confound influence
by instead imaging the biomarkers directly.
Each QMR image voxel
is thus a measurement 
of a given biomarker
at a specific location.
QMRI can therefore provide localized biomarker measurements
(\eg, myelin water content)
related to a specific physiological process
(\eg, demyelination)
that can through longitudinal study
be used to monitor the onset and progression of disease
(\eg, multiple sclerosis).

QMRI poses several challenges
beyond those of conventional MRI 
that currently limit its feasibility 
for routine clinical use.
For example,
accurate biomarker quantification 
traditionally requires multiple MR scans
and thus long scan times.
Furthermore,
it has previously been unclear
how to tune acquisition parameters 
of these multiple scans
to ensure that biomarkers can be quantified precisely.
Finally, MR biomarker quantification
is a challenging estimation problem
for which efficient algorithms
have previously been unavailable.
Addressing these challenges is essential
for widespread clinical adoption of QMRI.

\section{Overview}
\label{s,intro,over}

This thesis seeks
to address the above challenges
by developing an automated workflow for QMRI.
We exploit tools
from optimization, statistics, and machine learning
to develop fast algorithms
for quantifying biomarkers
that characterize specific physiological processes. 
We apply this framework
to challenging QMRI problems 
of clinical interest.
Our goal is to introduce fast, automated tools
that will increase the clinical value of QMRI.

Our solutions to two distinct subproblems in QMRI
constitute two stages of our proposed QMRI workflow.
Questions in \emph{acquisition design}
(Chapters~\ref{c,scn-dsgn}, \ref{c,mwf})
ask how to assemble 
fast collections of scans
that yield data 
rich in information 
about physical processes of interest.
Questions in \emph{parameter estimation}
(Chapters~\ref{c,relax}, \ref{c,perk})
ask how to quickly and reliably quantify biomarkers 
associated with these relevant physical processes.
The overall workflow seeks to
first design fast and informative scans 
based on the application,
and to then accurately and precisely estimate 
clinically relevant biomarkers.
 
\section{Organization}
\label{s,intro,org}

The main body of this thesis is organized as follows.
\edit{%
	Within this main body,
	Chapters~\ref{c,scn-dsgn}-\ref{c,mwf}
	organize the key contributions to science
	of this thesis.
}%
\begin{itemize}
\setlength\itemsep{0.5em}
	\item{% 
		Chapter~\ref{c,bkgrd} reviews 
		relevant background material
		on MRI and optimization.
	}%
	\item{% 
		Chapter~\ref{c,relax} discusses methods 
		for QMRI parameter estimation 
		from likelihood models 
		and applies these methods 
		for model-based MR relaxometry,
		a simple and popular application.
		It partially derives content
		from conference papers 
		\cite{nataraj:14:rje,nataraj:14:mbe}.
	}%
	\item{%
		Chapter~\ref{c,scn-dsgn} introduces
		a minimax optimization problem
		to aid robust and application-specific 
		MR scan selection and optimization 
		for precise latent parameter estimation.
		It optimizes several practical acquisitions 
		and uses the likelihood-based estimation techniques 
		introduced in Chapter~\ref{c,relax}
		to assess the utility
		of scan optimization
		through simulations 
		as well as phantom and \invivo experiments.
		It mainly derives content
		from published journal paper
		\cite{nataraj:17:oms}
		that extends conference paper
		\cite{nataraj:15:amm}.
	}%
	\item{% 
		Chapter~\ref{c,perk} introduces
		a fast, general algorithm
		for dictionary-free QMRI parameter estimation
		via regression with kernels (PERK).
		It demonstrates orders-of-magnitude acceleration 
		over likelihood-based estimators
		through simulations
		as well as phantom and \invivo experiments.
		It also characterizes PERK performance
		through bias-covariance analysis 
		and several robustness studies.
		It mainly derives content 
		from accepted journal paper
		\cite{nataraj::dfm} 
		that extends two conference papers
		\cite{nataraj:17:dfm,nataraj:17:slw}.
	}%
	\item{%
		Chapter~\ref{c,mwf} introduces a new method 
		for imaging an MR biomarker of clinical interest. 
		It applies ideas 
		developed in earlier chapters
		to design a new fast QMRI workflow
		that may be specific to healthy myelin,
		whose degradation is associated
		with certain white matter disorders.
		It demonstrates this new method 
		of potentially myelin-specific imaging
		in simulations and \invivo experiments.
		It partially derives content 
		from in-preparation journal paper \cite{nataraj::fmw}
		that extends conference paper \cite{nataraj:17:mwf}.
	}%
	\item{% 
		Chapter~\ref{c,future} suggests 
		several future research directions.
	}%
\end{itemize}

\vspace{0.5em}
{\setlength{\parindent}{0ex}
The appendices contain unpublished, less mature work
and are organized as follows:}
\begin{itemize}
	\setlength\itemsep{0.5em}
	\item{%
		Appendix~\ref{a,cc-multi} proposes an algorithm
		for simultaneously coil-combining 
		a collection of MR coil image datasets
		without prior knowledge 
		of coil sensitivity maps.
		Several chapters in the main body 
		used this algorithm for coil data combination.
	}%
	\item{%
		Appendix~\ref{a,ss-rf} develops 
		from first principles a new model
		for the influence of RF pulses
		on the steady-state (SS) transverse magnetization
		and then proposes two algorithms
		for SS-informed RF pulse design.
	}%
%	\item{%
%		Appendix~\ref{a,dess-diff} presents an analysis
%		of model mismatch due to the presence of diffusion,
%		demonstrates that neglecting diffusive effects
%		during $\Tt$ estimation 
%		can cause significant bias,
%		and suggests acquisition modifications
%		for mitigating this bias.
%	}%
\end{itemize}
