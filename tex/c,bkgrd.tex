% background

This chapter focuses
only on background information 
pertinent to multiple subsequent chapters.
We present further topic-specific information 
at the beginnings 
of corresponding chapters.
Section~\ref{s,bkgrd,mri} places emphasis
on reviewing necessary MR fundamentals,
and Section~\ref{s,bkgrd,opt}
proceeds to a shorter discussion
regarding optimization
as it pertains to QMRI.

\section{Relevant MR Physics}
\label{s,bkgrd,mri}

This section begins
with the fundamental Bloch equations
and derives the signal models
associated with 
Spoiled Gradient-Recalled Echo (SPGR) 
and Dual-Echo Steady-State (DESS),
two MR pulse sequences
used extensively in this thesis.
Our coverage of MRI
is far from comprehensive, 
and omits fundamental but tangential topics
such as signal localization.
We refer the interested reader
to books such as 
\cite{macovski:83,haacke:99,nishimura:96:pom}.

\subsection{Bloch Equations}
\label{ss,bkgrd,mri,bloch}

The Bloch equations
\cite{bloch:1946:ni-paper}
describe the magnetization dynamics
of \emph{spins}, 
or (loosely) atomic nuclei with nonzero 
angular momentum
and thus nonzero magnetic moment,
\eg $^1$H.
If the dominant source
of magnetic flux 
arises 
(as is typical in MRI)
from a main magnetic field
that is oriented along the $z$-axis,
the equations read
\begin{align}
	\dela{t} \mxy\prt &= i\gam\paren{\mz\prt \bxy\prt - \mxy\prt \bz\prt} -
		\frac{\mxy\prt}{\Tt\pr} ;
		\label{eq:bkgrd,mxy} \\
	\dela{t} \mz\prt &= \gam\paren{\mx\prt \by\prt - \my\prt \bx\prt} - 
		\frac{\mz\prt - \mzero\pr}{\To\pr}.
		\label{eq:bkgrd,mz}
\end{align}
Here, 
$\mxy\prt := \mx\prt + i\my\prt \in \complex$
and
$\mz\prt \in \real$
are the transverse and longitudinal components 
of the magnetization vector
at position $\bmr \in \reals{3}$ and time $t\geq0$;
$\bxy\prt := \bx\prt + i\by\prt \in \complex$
and 
$\bz\prt \in \real$
are the transverse and longitudinal components 
(in an inertial reference frame)
of the applied magnetic field;
$\To\pr$ and $\Tt\pr$
are spin-lattice and spin-spin relaxation time constants;
$\mzero\pr$ 
is the equilibrium magnetization
and is proportional to the density
of spins per unit volume
as well as the main field strength;
$\gam$
is the gyromagnetic ratio;
and $i := \sqrt{-1}$.
As written,
equations \eqref{eq:bkgrd,mxy}-\eqref{eq:bkgrd,mz}
only model dominant temporal dynamics;
later chapters consider second-order effects
such as multiple magnetization compartments
(Chapter~\ref{c,mwf})
and diffusion
(Appendix~\ref{a,dess-diff}).

It is often convenient 
to study Bloch dynamics in a non-inertial reference frame
rotating clockwise about the $z$-axis 
at Larmor frequency 
$\omzero := \gam \Bzero$,
where 
$\Bzero \unit{k}$
is the (nearly uniform) main magnetic field.
In these coordinates,
the apparent transverse magnetic field 
$\bxyp\prt = \bxp\prt + i\byp\prt := \bxy\prt e^{i\omzero t}$
transforms only in phase,
but the apparent longitudinal magnetic field
$\bzp\prt := \bz\prt - \Bzero$
is greatly reduced in magnitude.
The magnetization components transform more simply as
$\mxyp\prt = \mxp\prt + i\myp\prt := \mxy\prt e^{i\omzero t}$
and 
$\mzp\prt := \mz\prt$.
Remarkably, inserting these coordinate transformations 
into \eqref{eq:bkgrd,mxy}-\eqref{eq:bkgrd,mz} 
does not change the form
of the dynamical equations:
\begin{align}
	\dela{t} \mxyp\prt &= i\gam\paren{\mzp\prt \bxyp\prt - \mxyp\prt \bzp\prt} -
		\frac{\mxyp\prt}{\Tt\pr} ;
		\label{eq:bkgrd,mxyp} \\
	\dela{t} \mzp\prt &= \gam\paren{\mxp\prt \byp\prt - \myp\prt \bxp\prt} - 
		\frac{\mzp\prt - \mzero\pr}{\To\pr}.
		\label{eq:bkgrd,mzp}
\end{align}
It thus suffices to consider 
how (small) perturbations $\bmbp\prt$
to main field $\Bzero \unit{k}$ 
influence rotating-frame magnetization $\bmmp\prt$
via Eq.~\eqref{eq:bkgrd,mxyp}-\eqref{eq:bkgrd,mzp}.
The inertial-frame magnetization $\bmm\prt$ 
is then easily constructed via
$\mxy\prt = \mxyp\prt e^{-i\omzero t}$ 
and $\mz\prt = \mzp\prt$.

It is challenging
to explicitly solve Eq.~\eqref{eq:bkgrd,mxyp}-\eqref{eq:bkgrd,mzp}
for arbitrary field perturbations $\bmb\prt$
and relaxation times $\To\pr$, $\Tt\pr$. 
Relevant special cases are discussed in the following.

\subsubsection{Excitation}
\label{sss,bkgrd,mri,bloch,ex}

blah

\subsubsection{Precession and Relaxation}
\label{sss,bkgrd,mri,bloch,pr}

\subsection{Steady-State Sequences}
\label{ss,bkgrd,mri,ss}

\subsubsection{Spoiled Gradient-Recalled Echo (SPGR) Sequence}
\label{sss,bkgrd,mri,ss,spgr}

\subsubsection{Dual-Echo Steady-State (DESS) Sequence}
\label{sss,bkgrd,mri,ss,dess}


\section{Relevant Optimization Tools}
\label{s,bkgrd,opt}

\subsection{(Coping with) Non-convex Optimization}
\label{ss,bkgrd,opt,ncvx}

\subsection{Partially Linear Models and the Variable Projection Method}
\label{ss,bkgrd,opt,vpm}