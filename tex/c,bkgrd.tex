% background

This chapter focuses
only on background information 
pertinent to multiple subsequent chapters.
We present further topic-specific information 
at the beginnings 
of corresponding chapters.
Section~\ref{s,bkgrd,mri} places emphasis
on reviewing necessary MR fundamentals,
and Section~\ref{s,bkgrd,opt}
proceeds to a shorter discussion
regarding optimization
as it pertains to QMRI.

\section{Relevant MR Physics}
\label{s,bkgrd,mri}

This section begins
with the fundamental Bloch equations
and derives the signal models
associated with 
two MR pulse sequences
used extensively in this thesis.
Our coverage of MRI
is far from comprehensive, 
and omits fundamental but tangential topics
such as signal localization.
We refer the interested reader
to books such as 
\cite{macovski:83,haacke:99,nishimura:96:pom}.

\subsection{Bloch Equations}
\label{ss,bkgrd,mri,bloch}

The Bloch equations
\cite{bloch:1946:ni-paper}
describe the magnetization dynamics
of \emph{spins}, 
or (loosely) atomic nuclei with nonzero 
angular momentum
and thus nonzero magnetic moment,
\eg $^1$H.
If the dominant source
of magnetic flux 
arises 
(as is typical in MRI)
from a main magnetic field
that is oriented along the $z$-axis,
the equations read
\begin{align}
	\dela{t} \mxy\prt &= i\gam\paren{\mz\prt \bxy\prt - \mxy\prt \bz\prt} -
		\frac{\mxy\prt}{\Tt\pr} ;
		\label{eq:bloch-mxy} \\
	\dela{t} \mz\prt &= \gam\paren{\mx\prt \by\prt - \my\prt \bx\prt} - 
		\frac{\mz\prt - \mzero\pr}{\To\pr}.
		\label{eq:bloch-mz}
\end{align}
Here, 
$\mxy\prt := \mx\prt + i\my\prt \in \complex$
and
$\mz\prt \in \real$
are the transverse and longitudinal components 
of the magnetization vector
at position $\bmr := [x, y, z]\tpose \in \reals{3}$ 
and time $t\geq0$;
$\bxy\prt := \bx\prt + i\by\prt \in \complex$
and 
$\bz\prt \in \real$
are the transverse and longitudinal components 
(in an inertial reference frame)
of the applied magnetic field;
$\To\pr$ and $\Tt\pr$
are spin-lattice and spin-spin relaxation time constants;
$\mzero\pr$ 
is the equilibrium magnetization
and is proportional to the density
of spins per unit volume
as well as the main field strength;
$\gam$
is the gyromagnetic ratio;
and $i := \sqrt{-1}$.
As written,
equations \eqref{eq:bloch-mxy}-\eqref{eq:bloch-mz}
only model dominant temporal dynamics;
later chapters consider second-order effects
such as multiple magnetization compartments
(Chapter~\ref{c,mwf})
and diffusion
(Appendix~\ref{a,dess-diff}).

It is often convenient 
to study Bloch dynamics in a non-inertial reference frame
rotating clockwise about the $z$-axis 
at Larmor frequency 
$\omzero := \gam \Bzero$,
where 
$\Bzero \unit{k}$
is the (nearly uniform) main magnetic field.
In these coordinates,
the apparent transverse magnetic field 
$\bxyp\prt = \bxp\prt + i\byp\prt := \bxy\prt e^{i\omzero t}$
transforms only in phase,
but the apparent longitudinal magnetic field
$\bzp\prt := \bz\prt - \Bzero$
is greatly reduced in magnitude.
The magnetization components transform more simply as
$\mxyp\prt = \mxp\prt + i\myp\prt := \mxy\prt e^{i\omzero t}$
and 
$\mzp\prt := \mz\prt$.
Remarkably, inserting these coordinate transformations 
into \eqref{eq:bloch-mxy}-\eqref{eq:bloch-mz} 
does not change the form
of the dynamical equations:
\begin{align}
	\dela{t} \mxyp\prt &= i\gam\paren{\mzp\prt \bxyp\prt - \mxyp\prt \bzp\prt} -
		\frac{\mxyp\prt}{\Tt\pr} ;
		\label{eq:bloch-mxyp} \\
	\dela{t} \mzp\prt &= \gam\paren{\mxp\prt \byp\prt - \myp\prt \bxp\prt} - 
		\frac{\mzp\prt - \mzero\pr}{\To\pr}.
		\label{eq:bloch-mzp}
\end{align}
It thus suffices to consider 
how perturbations $\bmbp\prt$
to main field $\Bzero \unit{k}$ 
influence rotating-frame magnetization $\bmmp\prt$
via Eqs.~\eqref{eq:bloch-mxyp}-\eqref{eq:bloch-mzp}.
The inertial-frame magnetization $\bmm\prt$ 
is then easily constructed via
$\mxy\prt = \mxyp\prt e^{-i\omzero t}$ 
and $\mz\prt = \mzp\prt$.

It is challenging
to explicitly solve Eqs.~\eqref{eq:bloch-mxyp}-\eqref{eq:bloch-mzp}
for arbitrary field perturbations $\bmbp\prt$. 
We discuss relevant special cases 
in the following.

\subsubsection{Non-Selective Excitation}
\label{sss,bkgrd,mri,bloch,ex}

Here,
we derive solutions 
to Eqs.~\eqref{eq:bloch-mxyp}-\eqref{eq:bloch-mzp}
in the case of short, spatially non-selective excitations.
We take the following common assumptions:
\begin{itemize}
	\item 
		We assume negligible spatial variation
		in the longitudinal magnetic field, 
		so $\bzp\prt \approx 0$. 
		This lack of spatial variation is reason
		for non-selective excitation.
	\item 
		We assume the transverse field
		separates in position and time;
		oscillates at the Larmor frequency
		(commonly in the radiofrequency (RF) range);
		and aligns at initial time $t \gets t_0$ 
		with the $x$-axis.
		Together,
		these assumptions restrict
		the so-called RF excitation to take form
		$\bxyp\prt \approx \stx\pr \bonexp\pt \unit{i} + 0\unit{j}$,
		where $\stx\pr \in \real$ is the RF transmit coil spatial variation
		and $\bonexp\pt \in \real$ is the RF excitation envelope.
	\item
		We assume that 
		the duration $\TP$
		of RF excitation
		(often $\TP\sim$1ms)
		is much shorter than relaxation time constants
		(typically $\To\sim$1000ms and $\Tt\sim$50ms
		in brain tissue)
		and thus neglect relaxation effects
		during excitation.
\end{itemize}
Under these assumptions, 
Eqs.~\eqref{eq:bloch-mxyp}-\eqref{eq:bloch-mzp}
reduce to the linear system
\begin{align}
	\dela{t}
	\begin{bmatrix}
		\mxp\prt \\
		\myp\prt \\
		\mzp\prt
	\end{bmatrix}
	=
	\begin{bmatrix}
		0 & 0 & 0 \\
		0 & 0 & \gam \stx\pr \bonexp\pt \\
		0 & -\gam \stx\pr \bonexp\pt & 0
	\end{bmatrix}
	\begin{bmatrix}
		\mxp\prt \\
		\myp\prt \\
		\mzp\prt
	\end{bmatrix}.
	\label{eq:bloch-rot}
\end{align}
Eq.~\eqref{eq:bloch-rot} admits the simple solution
(for $t\geq t_0$)
\begin{align}
	\begin{bmatrix}
		\mxp\prt \\
		\myp\prt \\
		\mzp\prt
	\end{bmatrix}
	= 
	\begin{bmatrix}
		1 & 0 & 0 \\
		0 & \cosa{\fliprt} & \sina{\fliprt} \\
		0 & -\sina{\fliprt} & \cosa{\fliprt} 
	\end{bmatrix}
	\begin{bmatrix}
		\mxp\prtz \\
		\myp\prtz \\
		\mzp\prtz
	\end{bmatrix},
	\label{eq:rot-sol}
\end{align}
where 
$\bmmp\prtz := \brac{\mxp\prtz,\myp\prtz,\mzp\prtz}\tpose$ 
is the initial magnetization and
\begin{align}
	\fliprt := \gam \stx\pr \int_{t_0}^t \bonexp\paren{\tau} \der{\tau} 
	\label{eq:flip-def}
\end{align}
is the nutation (or ``flip'') angle at time $t$.
Eq.~\eqref{eq:rot-sol} reveals 
that on-resonance RF excitation
causes the magnetization vector
to rotate clockwise
about an axis parallel to
the direction of excitation.
The nutation angle
accumulated over an RF pulse 
of duration $\TP$
is often decomposed as
$\flip\prtzt{t_0+\TP} =: \flipnom \stx\pr$,
where $\flipnom$
is a prescribed nominal flip angle.

For deriving signal models
in later sections,
it is convenient and intuitive
to define matrix operators
that summarize relevant dynamics.
Here, we rewrite Eq.~\eqref{eq:rot-sol} as 
\begin{align}
	\bmmp\prt = \bmRxp{\fliprt} \bmmp\prtz,
	\label{eq:mtx-ex}
\end{align}
where $\bmRxp{\fliprt}$ denotes a clockwise rotation
of angle $\fliprt$ 
about the $x'$-axis.

\subsubsection{Free Precession and Relaxation}
\label{sss,bkgrd,mri,bloch,pr}

Next,
we derive solutions
to the rotating-frame Bloch equations
when no RF excitation is present,
\ie $\bxyp\prt \approx 0$.
In this case,
Eqs.~\eqref{eq:bloch-mxyp}-\eqref{eq:bloch-mzp} decouple,
yielding separate dynamical equations
for the transverse and longitudinal magnetization components:
\begin{align}
	\dela{t} \mxyp\prt &= -i\gam \mxyp\prt \bzp\prt - \frac{\mxyp\prt}{\Tt\pr} ;
		\label{eq:bloch-free-mxyp} \\
	\dela{t} \mzp\prt &= -\frac{\mzp\prt - \mzero\pr}{\To\pr}.
		\label{eq:bloch-free-mzp}
\end{align}
Eqs.~\eqref{eq:bloch-free-mxyp}-\eqref{eq:bloch-free-mzp}
admit simple solutions
with no further assumptions: 
\begin{align}
	\mxyp\prt &= \mxyp\prtz e^{-(t-t_0)/\Tt\pr} e^{-i \phprt}; 
		\label{eq:mxy-fp} \\
	\mzp\prt &= \mzp\prtz e^{-(t-t_0)/\To\pr} + \mzero\pr \paren{1-e^{-(t-t_0)/\To\pr}},
		\label{eq:mz-fp}
\end{align}
where $\mxyp\prtz$ and $\mzp\prtz$ 
are the initial magnetization components and
\begin{align}
	\phprt := \gam \int_{t_0}^t \bzp\paren{\bmr,\tau} \der{\tau}
	\label{eq:ph-def}
\end{align}
denotes the phase accumulation
due to main field inhomogeneity
(often called off-resonance effects).
Eq.~\eqref{eq:mxy-fp} reveals that without RF excitations,
the transverse magnetization $\mxyp\prt$
relaxes to zero exponentially fast 
with time constant $\Tt\pr$,
while accruing phase due to off-resonance effects.
Eq.~\eqref{eq:mz-fp} similarly reveals
that without RF excitations,
longitudinal magnetization $\mzp\prt$
recovers to $\mzero\pr$ exponentially fast 
with time constant $\To\pr$.

As in Section~\ref{sss,bkgrd,mri,bloch,pr},
we rewrite Eqs.~\eqref{eq:mxy-fp}-\eqref{eq:mz-fp}
for $t\geq t_0$
using matrix operators:
\begin{align}
	\bmmp\prt = \bmRzp{\phprt} \bmE\prtzt{t} \bmmp\prtz 
		+ \bmmzero\prtzt{t} 
		\label{eq:mtx-pr}
\end{align}
where
$\bmmzero\prtzt{t} := \mzero\pr \paren{1-e^{-(t-t_0)/\To\pr}} \unit{k}$; 
\begin{align}
	\bmRzp{\phprt} :=
	\begin{bmatrix}
		\cosa{\phprt} & \sina{\phprt} & 0 \\
		-\sina{\phprt} & \cosa{\phprt} & 0 \\
		0 & 0 & 1
	\end{bmatrix}
	\label{eq:op-rotz}
\end{align}
denotes a clockwise rotation of angle $\phprt$ about the $z'$-axis; 
and
\begin{align}
	\bmE{\prtzt{t}} := 
	\begin{bmatrix}
		e^{-(t-t_0)/\Tt\pr} & 0 & 0 \\
		0 & e^{-(t-t_0)/\Tt\pr} & 0 \\
		0 & 0 & e^{-(t-t_0)/\To\pr} 
	\end{bmatrix}
	\label{eq:op-relax}
\end{align}
is an exponential relaxation operator.
Section~\ref{ss,bkgrd,mri,ss}
(and later chapters) 
use matrix dynamical representations
\eqref{eq:mtx-ex} and \eqref{eq:mtx-pr}
to succinctly describe pulse sequence
signal models.

\subsection{Steady-State Sequences}
\label{ss,bkgrd,mri,ss}

MRI experiments
typically involve 
repeated cycles of (pulsed) RF excitation;
signal localization (not discussed here);
and transverse $\Tt$ relaxation and free precession, 
alongside (relatively slow) longitudinal $\To$ recovery.
We can build models 
of the received MR signal
by considering the magnetization dynamics
induced by specific pulse sequences.

Classical pulse sequences
use relatively long cycle repetition times $\TR$
to ensure near-complete $\To$ recovery
of the magnetization vector
back to equilibrium state $\mzero\pr \unit{k}$
prior to the start of each RF cycle.
For such long-$\TR$ sequences,
it suffices 
to approximate the magnetization
as fully recovered 
(\ie, $\bmmp\paren{\bmr,t_0 + \rep\TR} \approx \mzero\pr \unit{k}, 
\forall \rep\in\set{0,1,2,\dots}$)
just prior to each RF excitation.
This approximation
yields a sequence of initial conditions
and allows computation of the magnetization
at corresponding times of data acquisition
via direct application of Bloch dynamics
\eqref{eq:mtx-ex} and \eqref{eq:mtx-pr}.
Resulting signal models
are typically simple expressions
of relaxation parameters $\To\pr$ and $\Tt\pr$;
however, model accuracy often depends strongly
on the long-$\TR$ assumption,
which requires long acquisitions.

Steady-state (SS) sequences
\cite{hinshaw:76:ifb}
utilize short $\TR$,
and can thus achieve much faster scan times.
Due to short repetition times,
SS sequences achieve only partial $\To$ recovery
in between RF excitations;
thus, their magnetization responses
do not obey the simple classical initial conditions
(for the second RF cycle onwards).
Although their transient magnetization dynamics
can be complicated,
SS sequences produce
(under certain assumptions \cite{scheffler:99:apd})
long-time magnetization responses
that eventually
\footnote{The progression to steady state takes 
	on the order of 
	$5\Tt/\TR$ RF cycles \cite{scheffler:99:apd},
	typically a small but not insignificant period 
	during which data acquisition is often foregone. 
	This transition can
	(in some cases) be accelerated 
	by prepending SS sequences
	with tailored ``magnetization-catalyzing'' modules
	\cite{hargreaves:01:car}.
}
achieve a
steady-state condition:
\begin{align}
	\lim_{t_0 \to \infty} \bmmp\paren{\bmr,t_0+\rep\TR} = \bmmp\paren{\bmr,t_0},
	\label{eq:ss-cond}
\end{align}
where repetition count 
$\rep \in \set{1,2,\dots}$
for fixed RF excitations
and off-resonance induced phase increments
(as is assumed in the following).
Subsections~\ref{sss,bkgrd,mri,ss,spgr}
and \ref{sss,bkgrd,mri,ss,dess}
use SS condition \eqref{eq:ss-cond}
and Bloch equation matrix operators
introduced in
\eqref{eq:mtx-ex} and \eqref{eq:mtx-pr}
to derive long-time signal models
for Spoiled Gradient-Recalled Echo (SPGR)
and Dual-Echo Steady-State (DESS),
two SS pulse sequences
useful for quantitative MRI.

\subsubsection{Spoiled Gradient-Recalled Echo (SPGR) Sequence}
\label{sss,bkgrd,mri,ss,spgr}

SPGR \cite{zur:91:sot}
is a fast pulse sequence
that repeats cycles of 
fixed RF excitation
(such that $\bonexp\paren{t+\rep\TR} = \bonexp\pt,
\forall t\in\brac{t_0,t_0+\TP}, r\in\set{1,2,\dots}$);
data acquisition;
relaxation and recovery;
and residual transverse magnetization ``spoiling''
(discussed later).
Here we develop
a simple and popular steady-state SPGR signal model.

Let $\bmmp\prtz$ denote the magnetization
at an initial time $t_0$ 
selected well into the steady-state
and just prior to excitation.
The SPGR sequence first applies
an RF excitation, 
which rotates the initial magnetization
as per \eqref{eq:mtx-ex}:
\begin{align}
	\bmmp\paren{\bmr,t_0+\TP} = \bmRxp{\flip\prtzt{t_0+\TP}} \bmmp\prtz.
	\label{eq:spgr-ex}
\end{align}
The excited magnetization 
then precesses and relaxes
as per \eqref{eq:mtx-pr}
until data acquisition,
defined to occur at
``echo time'' $\TE \in \brac{\TP,\TR}$
after the (midpoint of) RF excitation:
\begin{align}
	\bmmp\paren{\bmr,t_0+\frac{\TP}{2}+\TE} &=
	\bmRzp{\php\paren{\bmr,\frac{\TP}{2}+\TE;\TP}} 
	\bmE\paren{\bmr,\frac{\TP}{2}+\TE;\TP} \bmmp\paren{\bmr,t_0+\TP} \nonumber \\
	&+ \bmmzero\paren{\bmr,\frac{\TP}{2}+\TE;\TP}.
	\label{eq:spgr-daq}
\end{align}
Following signal reception,
the remaining transverse magnetization 
is spoiled 
\footnote{Transverse signal spoiling 
is often (nearly) achieved in practice
using strong induced field inhomogeneities 
(which cause rapid transverse signal dephasing)
in tandem with RF excitations
that additionally impart nonlinear
(often quadratically increasing)
transverse magnetization phase
\cite{zur:91:sot}.
Though the nonlinear RF phase
used in so-called ``RF-spoiling'' 
prevents any one spin
from reaching a true steady-state,
the signal integrated
over a typically-sized voxel
achieves SS-like behavior
\cite{denolin:05:nii}.
}
while the longitudinal component
is unaffected. 
We model an ideal spoiling operation as
\begin{align}
	\bmS \bmmp\paren{\bmr,\frac{\TP}{2}+\TE}, \where \qquad
	\bmS := 
	\begin{bmatrix}
		0 & 0 & 0 \\
		0 & 0 & 0 \\
		0 & 0 & 1
	\end{bmatrix}.
	\label{eq:spgr-spoil}
\end{align}
After spoiling, 
the longitudinal magnetization 
(partially) recovers
until $t \gets t_0+\TR$:
\begin{align}
	\bmmp\paren{\bmr,t_0+\TR} &= 
	\bmRzp{\php\paren{\bmr,\TR;\frac{\TP}{2}+\TE}} 
	\bmE\paren{\bmr,\TR;\frac{\TP}{2}+\TE} \bmS 
	\bmmp\paren{\bmr,t_0+\frac{\TP}{2}+\TE} \nonumber \\
	&+ \bmmzero\paren{\bmr,\TR;\frac{\TP}{2}+\TE}.
	\label{eq:spgr-pr}
\end{align}
In steady-state, 
one cycle of excitation, acquisition, spoiling, and recovery 
returns the magnetization back to its initial state.
We enforce this through the steady-state condition
\begin{align}
	\bmmp\paren{\bmr,t_0+\TP} = \bmRxp{\flip\prtzt{t_0+\TP}} \bmmp\paren{\bmr,t_0+\TR}
	\label{eq:spgr-ss}
\end{align}
which yields an algebraic
system of equations.
When it exists,
the solution is
\begin{align}
	\bmmp\paren{\bmr,t_0+\TP} = 
	\frac{1}{1-e^{-\paren{\TR-\TP}/\To\pr}\cosa{\alpha\pr}}
	\begin{bmatrix}
		0 \\
		\mzero\pr \sina{\flip\pr} \paren{1-e^{-\paren{\TR-\TP}/\To\pr}}\\
		\mzero\pr \cosa{\flip\pr} \paren{1-e^{-\paren{\TR-\TP}/\To\pr}}
	\end{bmatrix},
	\label{eq:spgr-bmmp-t0}
\end{align}
where $\flip\pr := \flip\prtzt{t_0+\TP}$ 
is a slight abuse of notation.
Remarkably, 
the SPGR steady-state magnetization
immediately after excitation
is approximately independent
of both off-resonance effects
and $\Tt\pr$.
Researchers more often cite the expression
\begin{align}
	\mxyp\paren{\bmr,t_0+\TP}
	&= \mxp\paren{\bmr,t_0+\TP} + i\myp\paren{\bmr,t_0+\TP} \nonumber \\
	&= \frac{i\mzero\pr \sina{\flip\pr} \paren{1-e^{-\TR/\To\pr}}}
	{1-e^{-\TR/\To\pr}\cosa{\alpha\pr}}
	\label{eq:spgr-mxyp-t0}
\end{align}
for the complex transverse magnetization
as it modifies \eqref{eq:spgr-bmmp-t0}
to include a simple first-order correction
for unaccounted $\To$ recovery during the RF pulse.
Substituting \eqref{eq:spgr-mxyp-t0} 
into \eqref{eq:spgr-daq} 
yields an expression 
for the transverse magnetization
at the echo time:
\begin{align}
	\mxyp\paren{\bmr,t_0+\frac{\TP}{2}+\TE} 
	&=
	\mxyp\paren{\bmr,t_0+\TP} 
	e^{-(\TE-\TP/2)/\Tt\pr} 
	e^{-i\php\paren{\bmr,t_0+\frac{\TP}{2}+\TE;t_0+\TP}} \nonumber \\
	&\approx
	\mxyp\paren{\bmr,t_0+\TP} 
	e^{-\TE/\Tt\pr} 
	e^{-i\php\paren{\bmr,t_0+\frac{\TP}{2}+\TE;t_0+\frac{\TP}{2}}},
	\label{eq:spgr-mxyp-te}
\end{align}
where the approximation
again keeps in line 
with literature expressions.

The received signal
is approximately proportional 
to the integrated transverse magnetization
over a volume $\setV$.
To derive expressions,
we take a few more usual assumptions:
\begin{itemize}
	\item We assume that
	the signal is localized
	to a scale over which
	there is off-resonance phase variation,
	but minimal variation
	of $\mzero\pr$, $\To\pr$, $\Tt\pr$, and $\flip\pr$.
	This assumption is reasonable
	\footnote{Model mismatch due
		to within-voxel spatial variation 
		of relaxation parameters
		can be significant,
		especially for large voxels.
		Chapter~\ref{c,mwf} studies 
		so-called partial volume effects
		and uses them for QMRI.
	} 
	when describing the signal 
	arising from a typical voxel.
	\item We assume that
	(free-precession) off-resonance phase 
	grows linearly with time,
	\ie $\php\paren{\bmr,t_0+\frac{\TP}{2}+\TE;t_0+\frac{\TP}{2}} \approx \omp\pr \TE$.
	We further assume
	that off-resonance frequency $\omp\pr$
	is distributed over the localized voxel
	as $\dist{\omp} \gets \operatorname{Cauchy}\paren{\ompmed,\Rtp}$,
	where $\ompmed\pr$ is the median off-resonance frequency
	and $\Rtp\pr$ is the broadening bandwidth.
\end{itemize}
With these additional assumptions,
the received steady-state SPGR (noiseless) signal model
for a typically sized voxel
centered at position $\bmr$ is (to within constants):
\begin{align}
	\spgr\paren{\bmr,t_0+\frac{\TP}{2}+\TE} 
	&\propto \int_{\setV\pr} \mxyp\paren{\bmr,t_0+\frac{\TP}{2}+\TE} \dercubed{\bmr} 
		\label{eq:spgr-int} \\
	&\approx \mxy\paren{\bmr,t_0+\TP} e^{-\TE/\Tt\pr} \int_\real e^{-i\omp \TE} 
		\dist{\omp}\paren{\omp} \der{\omp} \nonumber \\
	&= \mxy\paren{\bmr,t_0+\TP} e^{-\TE/\Tt\pr} e^{-\Rtp\pr\TE - i\ompmed\pr\TE} \nonumber \\
	&= \frac{i\mzero\pr \sina{\flip\pr} \paren{1-e^{-\TR/\To\pr}}}
	{1-e^{-\TR/\To\pr}\cosa{\alpha\pr}} e^{-\TE/\Tts\pr} e^{-i\ompmed\pr \TE},
		\label{eq:spgr-model}
\end{align}
where $\Tts\pr := \inv{\frac{1}{\Tt} + \Rtp}$
is a modified spin-spin relaxation time
that accounts for additional transverse magnetization decay
due to off-resonance effects.

\subsubsection{Dual-Echo Steady-State (DESS) Sequence}
\label{sss,bkgrd,mri,ss,dess}

DESS \cite{redpath:88:fan,bruder:88:ans}
is a fast pulse sequence 
that interlaces fixed, constant-phase RF excitations
with fixed dephasing ``gradients''
(\ie, induced main field inhomogeneities 
that vary nearly linearly with space)
to produce two distinct signals
per RF excitation.
Here we develop simple 
steady-state DESS signal models.

As in Subsection~\ref{sss,bkgrd,mri,ss,spgr},
let $\bmmp\prtz$ denote the magnetization
at an initial time $t_0$ 
selected well into the steady-state
and just prior to excitation.
The DESS sequence first applies
a fixed RF rotation
$\flip\pr := \flip\paren{\bmr,t_0+\rep\TR+\TP;t_0+\rep\TR},
\forall r \in \set{0,1,2,\dots}$:
\begin{align}
	\bmmp\paren{\bmr,t_0+\TP} = \bmRxp{\flip{\pr}} \bmmp\prtz.
	\label{eq:dess-ex}
\end{align}
The excited transverse magnetization
contributes to a first acquired signal;
dephases (but does not spoil completely) 
due to gradient dephasing
\footnote{It is worth distinguishing 
gradient dephasing
(commonly 
but somewhat misleadingly 
referred to as gradient spoiling)
from RF spoiling.
Gradient dephasing
(used in DESS)
primarily affects magnetization phase
and is modeled simply as precession.
RF spoiling
(used in SPGR)
combines gradient dephasing 
with nonlinear RF phase cycling
and suppresses magnetization magnitude 
in steady-state.
}
and contributes again to a second 
(smaller, but nonzero) acquired signal.
Since (with proper selection) 
dephasing gradients mainly contribute 
to off-resonance phase accrual,
the net effect
after data acquisition
and gradient spoiling
is well described 
simply by precession and relaxation:
\begin{align}
	\bmmp\paren{\bmr,t_0+\TR} &= 
	\bmRzp{\php\paren{\bmr}} \bmE\paren{\bmr,\TR;\TP} \bmmp\paren{\bmr,t_0+\TP} +
	\bmmzero\paren{\bmr,\TR;\TP},
	\label{eq:dess-pr}
\end{align}
where the abbreviation
$\php\pr := \php\paren{\bmr,t_0+\paren{\rep+1}\TR;t_0+\rep\TR+\TP},
\forall \rep \in \set{0,1,2,\dots}$
implies fixed phase accrual 
(due to gradient dephasing, 
field inhomogeneity, 
and other unaccounted effects)
over each repetition cycle.

In steady-state,
one cycle of excitation,
first acquisition, 
gradient spoiling,
second acquisition,
and (partial) recovery
returns the magnetization
back to its initial state.
We enforce this
through the steady-state condition
\begin{align}
	\bmmp\prtz = \bmmp\paren{\bmr,t_0+\TR}
	\label{eq:dess-ss}
\end{align}	
which yields an algebraic system of equations.
When it exists, 
the solution gives 
the steady-state magnetization
just prior to RF excitation:
\begin{align}
	\bmmp\prtz =
	\begin{bmatrix}
		\EtRP \sin{\flip\pr} \sin{\php\pr} \\
		-\EtRP \sin{\flip\pr} \paren{\EtRP-\cos{\php\pr}} \\
		1-\EtRP\cos{\php\pr} + \EtRP \cos{\flip\pr} \paren{\EtRP-\cos{\php\pr}}
	\end{bmatrix}
	q\paren{\bmr,\TFP},
	\label{eq:dess-bmmp-t0}
\end{align}
where 
$\TFP := \TR-\TP$ 
is the free precession interval;
$\Eo\paren{\bmr,t} := e^{-t/\To\pr}$ and 
$\Et\paren{\bmr,t} := e^{-t/\Tt\pr}$
are relaxation operators;
and 
$q\paren{\bmr,t} := $
$$
\frac{\mzero\pr \paren{1-\Eo\prt}
}{\paren{1-\Eo\prt\cos\flip\pr} \paren{1-\Et\prt\cos\php\pr} -
\Et\prt \paren{\Eo\prt - \cos\flip\pr} \paren{\Et\prt - \cos\php\pr}}.
$$
Substituting \eqref{eq:dess-bmmp-t0} 
into \eqref{eq:dess-ex}
produces a similar expression 
for the steady-state magnetization
immediately following RF excitation:
\begin{align}
	\bmmp\paren{\bmr,t_0+\TP} =
	\begin{bmatrix}
		\EtRP \sin{\flip\pr} \sin{\php\pr} \\
		\sin{\flip\pr} \paren{1 - \EtRP \cos{\php\pr}} \\
		\cos{\flip\pr} \paren{1 - \EtRP \cos{\php\pr}} + \EtRP \paren{\EtRP - \cos{\php\pr}}
	\end{bmatrix}
	q\paren{\bmr,\TFP}.
	\label{eq:dess-bmmp-tp}
\end{align}
The transverse magnetizations
before and after RF excitation are then
\begin{align}
	\mxyp\prtz &= 
		-i\sin{\flip\pr} \EtR \paren{\EtR-e^{-i\php\pr}} q\paren{\bmr,\TR}
		\label{eq:dess-mxyp-t0}; \\
	\mxyp\paren{\bmr,t_0+\TP} &=
		+i\sin{\flip\pr} \paren{1-\EtR e^{i\php\pr}} q\paren{\bmr,\TR},
		\label{eq:dess-mxyp-tp}
\end{align}
where \eqref{eq:dess-mxyp-t0}-\eqref{eq:dess-mxyp-tp}
include simple first-order corrections
for yet-unaccounted relaxation and recovery
during excitation.
Frequently, 
the DESS signals are acquired 
at symmetric echo times $\TE$
before and after the center of each RF pulse.
Substituting \eqref{eq:dess-mxyp-tp} 
into \eqref{eq:bloch-free-mxyp}
gives the magnetization
at the data acquisition time
after RF excitation:
\begin{align}
	\mxyp\paren{\bmr,t_0+\frac{\TP}{2}+\TE} 
		&= \mxyp\paren{\bmr,t_0+\TP} e^{-\paren{\TE-\TP/2}/\Tt\pr} 
			e^{-i\php\paren{\bmr,t_0+\frac{\TP}{2}+\TE;t_0+\TP}} \nonumber \\
		&\approx \mxyp\paren{\bmr,t_0+\TP} e^{-\TE/\Tt\pr} 
			e^{-i\php\paren{\bmr,t_0+\frac{\TP}{2}+\TE;t_0+\frac{\TP}{2}}}
			\label{eq:dess-mxyp-te1-ph} \\
		&\approx \mxyp\paren{\bmr,t_0+\TP} e^{-\TE/\Tt\pr} e^{-i\omp\pr\TE},	
			\label{eq:dess-mxyp-te1}
\end{align}
where in \eqref{eq:dess-mxyp-te1-ph}
we again approximately correct
for relaxation during excitation
and in \eqref{eq:dess-mxyp-te1}
we assume linear off-resonance phase accrual 
during free precession.
To compute the magnetization
at the acquisition time 
before excitation,
we consider the free precession and relaxation
that occurs between 
\footnote{Observe that we do not attempt
to express the magnetization 
prior to (the next) RF excitation
by simply operating on the magnetization after (the current) RF excitation
with further precession and relaxation.
The reason is due to the intermediate dephasing gradient,
which causes phase accrual
in excess of off-resonance effects
and thus forbids an approximation akin to \eqref{eq:dess-mxyp-te1}.
}
signal reception and excitation:
\begin{align}
	\mxyp\paren{\bmr,t_0} =
		\mxyp\paren{\bmr,t_0-\paren{\TE-\frac{\TP}{2}}} e^{-\paren{\TE-\TP/2}/\Tt\pr} e^{-i\php\paren{\bmr,t_0;t_0-\paren{\TE-\frac{\TP}{2}}}}.
		\label{eq:dess-mxyp-pr-te2}
\end{align}
Rearranging \eqref{eq:dess-mxyp-pr-te2} 
and applying approximations
similar to those of 
\eqref{eq:dess-mxyp-te1-ph}-\eqref{eq:dess-mxyp-te1},
\begin{align}
	\mxyp\paren{\bmr,t_0+\frac{\TP}{2}-\TE} 
		&= \mxyp\paren{\bmr,t_0} e^{+\paren{\TE-\TP/2}/\Tt\pr} 
			e^{+i\php\paren{\bmr,t_0;t_0-\paren{\TE-\frac{\TP}{2}}}} \nonumber \\
		&\approx \mxyp\paren{\bmr,t_0} e^{+\TE/\Tt\pr} 
			e^{+i\php\paren{\bmr,t_0+\frac{\TP}{2};t_0+\frac{\TP}{2}-\TE}}
			\label{eq:dess-mxyp-te2-ph} \\
		&\approx \mxyp\paren{\bmr,t_0} e^{+\TE/\Tt\pr} e^{+i\omp\pr\TE}.	
			\label{eq:dess-mxyp-te2}
\end{align}

The received signal is approximately proportional
to the integrated transverse magnetization 
over a volume $\setV$.
To derive expressions,
we retake assumptions used
in Subsection~\ref{sss,bkgrd,mri,ss,spgr}
and append an additional assumption
on the full-repetition phase accrual $\php\pr$:
\begin{itemize}
	\item We assume that
		the signal is localized
		to a scale over which
		there is off-resonance phase variation,
		but minimal variation
		of $\mzero\pr$, $\To\pr$, $\Tt\pr$, and $\flip\pr$.
		This assumption is reasonable
		\footnote{Model mismatch due
			to within-voxel spatial variation 
			of relaxation parameters
			can be significant,
			especially for large voxels.
			Chapter~\ref{c,mwf} studies 
			so-called partial volume effects
			and uses them for QMRI.
		} 
		when describing the signal 
		arising from a typical voxel.
	\item We assume that
		free precession off resonance frequency $\omp\pr$
		is distributed over the localized voxel
		as $\dist{\omp} \gets \operatorname{Cauchy}\paren{\ompmed,\Rtp}$,
		where $\ompmed\pr$ is the median off-resonance frequency
		and $\Rtp\pr$ is the broadening bandwidth.
	\item We assume that 
		the dephasing gradient imparts 
		an integral number $\cyc$ of across-voxel phase cycles
		\footnote{In theory,
			it suffices to design dephasing gradients
			to impart as few as one complete cycle
			of net phase variation across a voxel.
			In practice,
			field inhomogeneities will induce
			spurious through-voxel field gradients 
			that modify the effective dephasing gradient moment
			and thereby create partial phase cycles
			that distort the nominally uniform phase distribution.
			To reduce model mismatch 
			due to such ``partial spoiling'' effects,
			dephasing gradients are usually designed
			to nominally impart multiple complete cycles
			of across-voxel phase variation.
			However, 
			larger dephasing gradients
			cause greater DESS model mismatch
			due to diffusive signal loss.
			Appendix~\ref{a,dess-diff} 
			studies diffusion in DESS
			and discusses regimes of dephasing gradient moments
			which balance partial-spoiling versus diffusive
			sources of model mismatch.
		}
		such that full-repetition phase accrual $\php\pr$ 
		is distributed essentially uniformly
		as $\dist{\php} \gets \operatorname{uniform}\paren{0,2\pi\cyc},
		\cyc\in\set{1,2,3,\dots}$.  
\end{itemize}
With these assumptions, 
the received steady-state DESS (noiseless) signal models
for a typically sized voxel centered at position $\bmr$ are
(to within constants):
\begin{align}
	\dess\paren{\bmr,t_0+\frac{\TP}{2}+\TE} 
		&\propto \int_{\setV\paren{\bmr}} 
			\mxyp\paren{\bmr,t_0+\frac{\TP}{2}+\TE} \dercubed{\bmr}
			\label{eq:dess-def-int} \\
		&\approx \int_\real \int_\real \mxyp\paren{\bmr,t_0+\frac{\TP}{2}+\TE} 
			\dist{\php}\paren{\php} \dist{\omp}\paren{\omp} \der{\php} \der{\omp}
			\nonumber \\
		&\approx e^{-\TE/\Tt\pr}
			\int_\real \mxyp\paren{\bmr,t_0+\TP} \dist{\php}\paren{\php} \der{\php} 
			\int_\real e^{-i\omp\TE} \dist{\omp}\paren{\omp} \der{\omp}
			\nonumber \\
		&= +i\mzero\pr \Et\paren{\bmr,\TE} e^{-\paren{\Rtp\pr-i\ompmed\pr}\TE}
			\tan\frac{\alpha\pr}{2} 
			\brac{1 - \frac{\eta\paren{\bmr,\TR}}{\xi\paren{\bmr,\TR}}};
			\label{eq:dess-def-model} \\
	\dess\paren{\bmr,t_0+\frac{\TP}{2}-\TE}
		&\propto \int_{\setV\paren{\bmr}}
			\mxyp\paren{\bmr,t_0+\frac{\TP}{2}-\TE} \dercubed{\bmr}
			\label{eq:dess-ref-int} \\
		&\approx \int_\real \int_\real \mxyp\paren{\bmr,t_0+\frac{\TP}{2}-\TE} 
			\dist{\php}\paren{\php} \dist{\omp}\paren{\omp} \der{\php} \der{\omp}
			\nonumber \\
		&\approx e^{+\TE/\Tt\pr}
			\int_\real \mxyp\paren{\bmr,t_0} \dist{\php}\paren{\php} \der{\php} 
			\int_\real e^{+i\omp\TE} \dist{\omp}\paren{\omp} \der{\omp} 
			\nonumber \\
		&= -i\mzero\pr \Et^{-1}\paren{\bmr,\TE} e^{-\paren{\Rtp\pr+i\ompmed\pr}\TE}
			\tan\frac{\alpha\pr}{2}
			\brac{1 - \eta\paren{\bmr,\TR}},
			\label{eq:dess-ref-model}
\end{align}	
where \eqref{eq:dess-def-model} and \eqref{eq:dess-ref-model}
introduce intermediate variables
\begin{align}
	\eta\prt &:=
		\sqrt{\frac{1-\Et^2\prt}{1-\Et^2\prt/\xi^2\prt}};
		\nonumber \\
	\xi\prt &:=
		\frac{1-\Eo\prt \cos{\flip\pr}}{\Eo\prt - \cos{\flip\pr}}.
		\nonumber
\end{align}
In steady-state, 
the DESS signal is typically greatest 
immediately following excitation 
and defocuses with rate $\frac{1}{\Tt}+\Rtp$
until what we hereafter denote
the \emph{defocusing} echo time.
After a low-signal period between RF pulses,
the DESS signal then refocuses
with rate $\frac{1}{\Tt}-\Rtp$
from what we hereafter denote
the \emph{refocusing} echo time
until just prior the next excitation.
Fortuitously,
the defocusing \eqref{eq:dess-def-model}
and refocusing \eqref{eq:dess-ref-model}
DESS signal models
have significantly different dependence 
on relaxation parameters (especially $\Tt$)
and thus together are quite useful
for relaxation parameter estimation.

\section{Optimization in QMRI}
\label{s,bkgrd,opt}

This section overviews 
how optimization methods are leveraged 
in a substantial portion of this thesis 
to solve practical QMRI problems.
For such problems,
the central idea is to construct
a suitable scalar cost function $\cost$
of some design variables $\bmx$,
whose output $\costa{\bmx} \in \real$ 
is designed to provide a measure
of the undesirability of $\bmx$.
We then employ 
tailored optimization algorithms
to find an $\bmx$
that minimizes $\cost$
over a set $\setX$,
written as
\begin{align}
	\bmx^* \in \set{\argmin{\bmx\in\setX} \costa{\bmx}}.
	\label{eq:opt-global}
\end{align}
In either optimization-based 
parameter estimation (Chapter~\ref{c,relax})
or acquisition design (Chapter~\ref{c,scn-dsgn}),
we have reason to design $\cost$
to depend on corresponding design variables $\bmx$ 
through MR signal models.
Because these models are often 
(strongly) nonlinear functions
of design variables,
corresponding cost functions
are usually non-convex in $\bmx$
(though the search space $\setX$ 
is almost always assumed convex
in this thesis).
Thus,
most QMRI problems
in the form of \eqref{eq:opt-global}
are non-convex optimization problems.

In general, 
solving \eqref{eq:opt-global}
is more challenging when 
$\Psi$ is non-convex in $\bmx$
than otherwise,
due in part to the possible presence
of local extrema and/or saddle points.
In the following, 
we discuss two strategies 
used in this thesis 
to cope with non-convex optimization.
Subsection~\ref{ss,bkgrd,opt,loc}
relaxes \eqref{eq:opt-global}
to instead seek a local minimizer
via iterative methods.
Subsection~\ref{ss,bkgrd,opt,vpm}
restricts attention 
to signal models 
that are linear in a portion of $\bmx$
and discusses a specific problem
for which \eqref{eq:opt-global} simplifies 
for such partially linear structures.

\subsection{Iterative Local Optimization with Constraints}
\label{ss,bkgrd,opt,loc}

This subsection overviews
a method for finding a local minimizer $\est{\bmx}$
of possibly non-convex cost function $\cost$
over convex constraint set $\setX$.
Such $\est{\bmx} \in \setX$ must satisfy
for some $\delta>0$
\begin{align}
	\costa{\est{\bmx}} \leq \costa{\bmx} \qquad
		\forall \bmx \in \setX : \norm{\est{\bmx}-\bmx}_2 < \delta.
		\label{eq:opt-local}
\end{align}	
Observe that
a global optimizer $\bmx^*$ 
satisfies \eqref{eq:opt-local}
for arbitrarily large $\delta$;
thus, any global minimizer 
is a local minimizer
(but the converse is not necessarily true
unless $\cost$ is convex).

As even locally optimal minimizers
are often challenging to compute analytically,
many algorithms construct $\est{\bmx}$ 
by iteratively updating
an initial guess $\bmxi{0}$
until some convergence criterion is satisfied.
For a differentiable cost
and convex constraints, 
the gradient projection method
\cite{rosen:60:tgp}
is one such iterative algorithm
and repeats the following simple update:
\begin{align}
	\bmxi{i} \gets \proja{\setX}{\bmxi{i-1}-\bmPi \grada{\bmx}\costa{\bmxi{i-1}}},
	\label{eq:gpm}
\end{align}
where $\proj{\setX}$ denotes 
projection onto $\setX$
and $\bmPi$ is a diagonal preconditioning matrix
that permits elements of $\bmx$
to take scale-informed step sizes
along the negative gradient direction.

If $\cost$ is convex and sufficiently smooth,
iterates produced via \eqref{eq:gpm} 
converge to a limit point \cite{byrne:04:aut}
that is a constrained global minimum
(for appropriately selected $\bmPi$).
If instead $\cost$ is non-convex 
(but $\setX$ is still convex),
statements regarding convergence
\footnote{For example, 
it suffices to assume
that $\bmxi{0}$ lies
in the \emph{attraction basin} $\setB{\bmxt}$
of a given unconstrained local minimum $\bmxt$, 
where attraction basin 
is defined here as the largest convex set
containing $\bmxt$ 
over which $\cost$ is convex.
If $\setB{\bmxt} \cap \setX$ is nonempty 
and step sizes within $\bmPi$ are small enough 
to contain iterates
within $\setB{\bmxt}$,
then iterates converge
to the limit point $\proja{\setX}{\bmxt}$.
}
to a particular constrained local minimizer
require additional (strong) assumptions
regarding initialization
and in general 
are still much weaker 
than in the convex case.

Since non-convex cost functions
can have many local extrema
(whose associated costs can vary dramatically),
the utility of a locally optimal solution
depends strongly on initialization quality.
Accordingly,
this thesis uses iterative local optimization
for non-convex QMRI problems
where a reasonable initialization is available
and global optimization (to within quantization error) 
via exhaustive grid search
is intractable.

\subsection{Partially Linear Models and the Variable Projection Method}
\label{ss,bkgrd,opt,vpm}

(Constrained, weighted) nonlinear least-squares
is a specific non-convex optimization problem 
that is useful for many parameter estimation problems:
\begin{align}
	\bmx^* \in \set{\argmin{\bmx\in\setX} \norm{\bmy - \bmfa{\bmx}}^2_{\bmW}},
	\label{eq:nonlin-ls}
\end{align}
where $\bmf : \setX \mapsto \complexes{D}$ is 
a nonlinear forward model
that (barring noise) 
relates parameters 
$\bmx \in \setX \subseteq \complexes{L}$ 
to data $\bmy \in \complexes{D}$;
weighted 2-norm
$\norm{\bmiota}_\bmW := \sqrt{\bmiota\ctpose\bmW\bmiota}$
for a symmetric, positive-semidefinite weighting matrix
$\bmW \in \reals{D\times D}$ 
and arbitrary vector $\bmiota \in \complexes{D}$;
and $\paren{\cdot}\ctpose$ denotes conjugate transpose.
The variable projection method 
\cite{golub:03:snl}
reduces the complexity of \eqref{eq:nonlin-ls}
when the forward model takes
the partially linear structure
$\bmfa{\bmx} \equiv \bmAa{\bmxn}\bmxl$
and the feasible set takes 
the partially unconstrained form 
$\setX \equiv \complexes{\Ll} \times \setXn$,
where $\bmxl \in \complexes{\Ll}$; $\bmxn \in \setXn$;
and $\bmA : \setXn \mapsto \complexes{D\times\Ll}$ 
is a matrix function.
These restrictions on \eqref{eq:nonlin-ls} 
define a so-called separable least-squares problem:
\begin{align}
	\paren{\bmxl^*,\bmxn^*} \in 
		\set{\argmin{\substack{\bmxl \in \complexes{\Ll} \\ \bmxn \in \setXn}} 
		\norm{\bmy - \bmAa{\bmxn}\bmxl}^2_{\bmW}}.
	\label{eq:sep-ls}
\end{align}
The variable projection method simplifies \eqref{eq:sep-ls}
by exploiting the partially linear structure of $\bmf$ 
to explicitly express the optimal $\bmxl^*$ as a function 
of any fixed $\bmxn \in \setXn$:
\begin{align}
	\bmxl^*\paren{\bmxn} 
		&= \argmin{\bmxl \in \complexes{\Ll}} 
		\norm{\bmy - \bmAa{\bmxn}\bmxl}^2_{\bmW} 
		\nonumber \\
		&= \pinv{\bmW^{1/2}\bmAa{\bmxn}} \bmW^{1/2} \bmy
		\label{eq:sep-ls-lin} \\
		&= \inv{\bmA\ctpose\paren{\bmxn} \bmW \bmAa{\bmxn}} 
		\bmA\ctpose\paren{\bmxn} \bmW \bmy,
		\label{eq:sep-ls-fullrnk}
\end{align}
where $\pinv{\cdot}$ denotes pseudoinverse;
$\bmW^{1/2}$ denotes principle (matrix) square root;
and \eqref{eq:sep-ls-fullrnk} holds
if the matrix inversion within exists.
Substituting \eqref{eq:sep-ls-fullrnk}
into \eqref{eq:sep-ls} 
yields a new non-convex optimization problem
that contains $\Ll$ fewer unknowns than before:
\begin{align}
	\bmxn^* &\in \set{\argmin{\bmxn\in\setXn} 
	\norm{\bmy - \bmAa{\bmxn}\inv{\bmA\ctpose\paren{\bmxn} \bmW \bmAa{\bmxn}}
		\bmA\ctpose\paren{\bmxn} \bmW \bmy}^2_{\bmW^{1/2}}} 
		\nonumber \\
	&\equiv \set{\argmax{\bmxn\in\setXn}
		\bmy\ctpose \bmW \bmAa{\bmxn}
		\inv{\bmA\ctpose\paren{\bmxn} \bmW \bmAa{\bmxn}}
		\bmA\ctpose\paren{\bmxn} \bmW \bmy}, 
		\label{eq:sep-ls-nonlin}
\end{align}
where the equivalence leading to \eqref{eq:sep-ls-nonlin}
omits terms independent of $\bmxn$.

In low-dimensional
QMRI applications 
(\eg, those discussed in Chapter~\ref{c,relax}), 
reduced problem \eqref{eq:sep-ls-nonlin}
may be tractable via exhaustive grid search,
in which case a global optimum 
$\paren{\bmxl^*\paren{\bmxn^*},\bmxn^*}$
is achievable to within quantization error.
However, larger estimation problems 
involving more nonlinear parameters
might still be tractable 
only via iterative optimization
(see Subsection~\ref{ss,bkgrd,opt,loc})
towards a local solution 
$\paren{\est{\bmx}_\mathrm{L}\paren{\est{\bmx}_\mathrm{N}},\est{\bmx}_\mathrm{N}}$.
For such higher-dimensional applications,
Chapters~\ref{c,krr}-\ref{c,mwf} introduce novel methods 
that tackle problems similar to \eqref{eq:nonlin-ls},
while circumventing initialization-dependent local optimization.
