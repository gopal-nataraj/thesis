% background

This chapter focuses
only on background information 
pertinent to multiple subsequent chapters.
We present further topic-specific information 
at the beginnings 
of corresponding chapters.
Section~\ref{s,bkgrd,mri} places emphasis
on reviewing necessary MR fundamentals,
and Section~\ref{s,bkgrd,opt}
proceeds to a shorter discussion
regarding optimization
as it pertains to QMRI.

\section{Relevant MR Physics}
\label{s,bkgrd,mri}

This section begins
with the fundamental Bloch equations
and derives the signal models
associated with 
two MR pulse sequences
used extensively in this thesis.
Our coverage of MRI
is far from comprehensive, 
and omits fundamental but tangential topics
such as signal localization.
We refer the interested reader
to books such as 
\cite{macovski:83,haacke:99,nishimura:96:pom}.

\subsection{Bloch Equations}
\label{ss,bkgrd,mri,bloch}

The Bloch equations
\cite{bloch:1946:ni-paper}
describe the magnetization dynamics
of \emph{spins}, 
or (loosely) atomic nuclei with nonzero 
angular momentum
and thus nonzero magnetic moment,
\eg $^1$H.
If the dominant source
of magnetic flux 
arises 
(as is typical in MRI)
from a main magnetic field
that is oriented along the $z$-axis,
the equations read
\begin{align}
	\dela{t} \mxy\prt &= i\gam\paren{\mz\prt \bxy\prt - \mxy\prt \bz\prt} -
		\frac{\mxy\prt}{\Tt\pr} ;
		\label{eq:bloch-mxy} \\
	\dela{t} \mz\prt &= \gam\paren{\mx\prt \by\prt - \my\prt \bx\prt} - 
		\frac{\mz\prt - \mzero\pr}{\To\pr}.
		\label{eq:bloch-mz}
\end{align}
Here, 
$\mxy\prt := \mx\prt + i\my\prt \in \complex$
and
$\mz\prt \in \real$
are the transverse and longitudinal components 
of the magnetization vector
at position $\bmr \in \reals{3}$ and time $t\geq0$;
$\bxy\prt := \bx\prt + i\by\prt \in \complex$
and 
$\bz\prt \in \real$
are the transverse and longitudinal components 
(in an inertial reference frame)
of the applied magnetic field;
$\To\pr$ and $\Tt\pr$
are spin-lattice and spin-spin relaxation time constants;
$\mzero\pr$ 
is the equilibrium magnetization
and is proportional to the density
of spins per unit volume
as well as the main field strength;
$\gam$
is the gyromagnetic ratio;
and $i := \sqrt{-1}$.
As written,
equations \eqref{eq:bloch-mxy}-\eqref{eq:bloch-mz}
only model dominant temporal dynamics;
later chapters consider second-order effects
such as multiple magnetization compartments
(Chapter~\ref{c,mwf})
and diffusion
(Appendix~\ref{a,dess-diff}).

It is often convenient 
to study Bloch dynamics in a non-inertial reference frame
rotating clockwise about the $z$-axis 
at Larmor frequency 
$\omzero := \gam \Bzero$,
where 
$\Bzero \unit{k}$
is the (nearly uniform) main magnetic field.
In these coordinates,
the apparent transverse magnetic field 
$\bxyp\prt = \bxp\prt + i\byp\prt := \bxy\prt e^{i\omzero t}$
transforms only in phase,
but the apparent longitudinal magnetic field
$\bzp\prt := \bz\prt - \Bzero$
is greatly reduced in magnitude.
The magnetization components transform more simply as
$\mxyp\prt = \mxp\prt + i\myp\prt := \mxy\prt e^{i\omzero t}$
and 
$\mzp\prt := \mz\prt$.
Remarkably, inserting these coordinate transformations 
into \eqref{eq:bloch-mxy}-\eqref{eq:bloch-mz} 
does not change the form
of the dynamical equations:
\begin{align}
	\dela{t} \mxyp\prt &= i\gam\paren{\mzp\prt \bxyp\prt - \mxyp\prt \bzp\prt} -
		\frac{\mxyp\prt}{\Tt\pr} ;
		\label{eq:bloch-mxyp} \\
	\dela{t} \mzp\prt &= \gam\paren{\mxp\prt \byp\prt - \myp\prt \bxp\prt} - 
		\frac{\mzp\prt - \mzero\pr}{\To\pr}.
		\label{eq:bloch-mzp}
\end{align}
It thus suffices to consider 
how perturbations $\bmbp\prt$
to main field $\Bzero \unit{k}$ 
influence rotating-frame magnetization $\bmmp\prt$
via Eqs.~\eqref{eq:bloch-mxyp}-\eqref{eq:bloch-mzp}.
The inertial-frame magnetization $\bmm\prt$ 
is then easily constructed via
$\mxy\prt = \mxyp\prt e^{-i\omzero t}$ 
and $\mz\prt = \mzp\prt$.

It is challenging
to explicitly solve Eqs.~\eqref{eq:bloch-mxyp}-\eqref{eq:bloch-mzp}
for arbitrary field perturbations $\bmbp\prt$. 
We discuss relevant special cases 
in the following.

\subsubsection{Non-Selective Excitation}
\label{sss,bkgrd,mri,bloch,ex}

Here,
we derive solutions 
to Eqs.~\eqref{eq:bloch-mxyp}-\eqref{eq:bloch-mzp}
in the case of short, spatially non-selective excitations.
We take the following common assumptions:
\begin{itemize}
	\item 
		We assume negligible spatial variation
		in the longitudinal magnetic field, 
		so $\bzp\prt \approx 0$. 
		This lack of spatial variation is reason
		for non-selective excitation.
	\item
		We assume that 
		the duration $\TP$
		of RF excitation
		(often $\TP\sim$1ms)
		is much shorter than relaxation time constants
		(typically $\To\sim$1000ms and $\Tt\sim$50ms
		in brain tissue)
		and thus neglect relaxation effects
		during excitation.
	\item 
		We assume the transverse field
		separates in position and time;
		oscillates at the Larmor frequency
		(commonly in the radiofrequency (RF) range);
		and aligns at initial time $t \gets t_0$ 
		with the $x$-axis.
		Together,
		these assumptions restrict
		the so-called RF excitation to take form
		$\bxyp\prt \approx \stx\pr \bonexp\pt \unit{i} + 0\unit{j}$,
		where $\stx\pr \in \real$ is the RF transmit coil spatial variation
		and $\bonexp\pt \in \real$ is the RF excitation envelope.
\end{itemize}
Under these assumptions, 
Eqs.~\eqref{eq:bloch-mxyp}-\eqref{eq:bloch-mzp}
reduce to the linear system
\begin{align}
	\dela{t}
	\begin{bmatrix}
		\mxp\prt \\
		\myp\prt \\
		\mzp\prt
	\end{bmatrix}
	=
	\begin{bmatrix}
		0 & 0 & 0 \\
		0 & 0 & \gam \stx\pr \bonexp\pt \\
		0 & -\gam \stx\pr \bonexp\pt & 0
	\end{bmatrix}
	\begin{bmatrix}
		\mxp\prt \\
		\myp\prt \\
		\mzp\prt
	\end{bmatrix}.
	\label{eq:bloch-rot}
\end{align}
Eq.~\eqref{eq:bloch-rot} admits the simple solution
(for $t\geq t_0$)
\begin{align}
	\begin{bmatrix}
		\mxp\prt \\
		\myp\prt \\
		\mzp\prt
	\end{bmatrix}
	= 
	\begin{bmatrix}
		1 & 0 & 0 \\
		0 & \cosa{\fliprt} & \sina{\fliprt} \\
		0 & -\sina{\fliprt} & \cosa{\fliprt} 
	\end{bmatrix}
	\begin{bmatrix}
		\mxp\prtz \\
		\myp\prtz \\
		\mzp\prtz
	\end{bmatrix},
	\label{eq:rot-sol}
\end{align}
where 
$\bmmp\prtz := \brac{\mxp\prtz,\myp\prtz,\mzp\prtz}\tpose$ 
is the initial magnetization and
\begin{align}
	\fliprt := \gam \stx\pr \int_{t_0}^t \bonexp\paren{\tau} \der{\tau} 
	\label{eq:flip-def}
\end{align}
is the nutation (or ``flip'') angle at time $t$.
Eq.~\eqref{eq:rot-sol} reveals 
that on-resonance RF excitation
causes the magnetization vector
to rotate clockwise
about an axis parallel to
the direction of excitation.
The nutation angle
accumulated over an RF pulse 
of duration $\TP$
is often decomposed as
$\flip\prtzt{t_0+\TP} =: \flipnom \stx\pr$,
where $\flipnom$
is a prescribed nominal flip angle.

For deriving signal models
in later sections,
it is convenient and intuitive
to define matrix operators
that summarize relevant dynamics.
Here, we rewrite Eq.~\eqref{eq:rot-sol} as 
\begin{align}
	\bmmp\prt = \bmRxp{\fliprt} \bmmp\prtz,
	\label{eq:mtx-ex}
\end{align}
where $\bmRxp{\fliprt}$ denotes a clockwise rotation
of angle $\fliprt$ 
about the $x'$-axis.

\subsubsection{Free Precession and Relaxation}
\label{sss,bkgrd,mri,bloch,pr}

Next,
we derive solutions
to the rotating-frame Bloch equations
when no RF excitation is present,
\ie $\bxyp\prt \approx 0$.
In this case,
Eqs.~\eqref{eq:bloch-mxyp}-\eqref{eq:bloch-mzp} decouple,
yielding separate dynamical equations
for the transverse and longitudinal magnetization components:
\begin{align}
	\dela{t} \mxyp\prt &= -i\gam \mxyp\prt \bzp\prt - \frac{\mxyp\prt}{\Tt\pr} ;
		\label{eq:bloch-free-mxyp} \\
	\dela{t} \mzp\prt &= -\frac{\mzp\prt - \mzero\pr}{\To\pr}.
		\label{eq:bloch-free-mzp}
\end{align}
Eqs.~\eqref{eq:bloch-free-mxyp}-\eqref{eq:bloch-free-mzp}
admit simple solutions
with no further assumptions: 
\begin{align}
	\mxyp\prt &= \mxyp\prtz e^{-(t-t_0)/\Tt\pr} e^{-i \phprt}; 
		\label{eq:mxy-fp} \\
	\mzp\prt &= \mzp\prtz e^{-(t-t_0)/\To\pr} + \mzero\pr \paren{1-e^{-(t-t_0)/\To\pr}},
		\label{eq:mz-fp}
\end{align}
where $\mxyp\prtz$ and $\mzp\prtz$ 
are the initial magnetization components and
\begin{align}
	\phprt := \gam \int_{t_0}^t \bzp\paren{\bmr,\tau} \der{\tau}
	\label{eq:ph-def}
\end{align}
denotes the phase accumulation
due to main field inhomogeneity
(often called off-resonance effects).
Eq.~\eqref{eq:mxy-fp} reveals that without RF excitations,
the transverse magnetization $\mxyp\prt$
relaxes to zero exponentially fast 
with time constant $\Tt\pr$,
while accruing phase due to off-resonance effects.
Eq.~\eqref{eq:mz-fp} similarly reveals
that without RF excitations,
longitudinal magnetization $\mzp\prt$
recovers to $\mzero\pr$ exponentially fast 
with time constant $\To\pr$.

As in Section~\ref{sss,bkgrd,mri,bloch,pr},
we rewrite Eqs.~\eqref{eq:mxy-fp}-\eqref{eq:mz-fp}
for $t\geq t_0$
using matrix operators:
\begin{align}
	\bmmp\prt = \bmRzp{\phprt} \bmE\prtzt{t} \bmmp\prtz 
		+ \bmmzero\prtzt{t} 
		\label{eq:mtx-pr}
\end{align}
where
$\bmmzero\prtzt{t} := \mzero\pr \paren{1-e^{-(t-t_0)/\To\pr}} \unit{k}$; 
\begin{align}
	\bmRzp{\phprt} :=
	\begin{bmatrix}
		\cosa{\phprt} & \sina{\phprt} & 0 \\
		-\sina{\phprt} & \cosa{\phprt} & 0 \\
		0 & 0 & 1
	\end{bmatrix}
	\label{eq:op-rotz}
\end{align}
denotes a clockwise rotation of angle $\phprt$ about the $z'$-axis; 
and
\begin{align}
	\bmE{\prtzt{t}} := 
	\begin{bmatrix}
		e^{-(t-t_0)/\Tt\pr} & 0 & 0 \\
		0 & e^{-(t-t_0)/\Tt\pr} & 0 \\
		0 & 0 & e^{-(t-t_0)/\To\pr} 
	\end{bmatrix}
	\label{eq:op-relax}
\end{align}
is an exponential relaxation operator.
Section~\ref{ss,bkgrd,mri,ss}
(and later chapters) 
use matrix dynamical representations
\eqref{eq:mtx-ex} and \eqref{eq:mtx-pr}
to succinctly describe pulse sequence
signal models.

\subsection{Steady-State Sequences}
\label{ss,bkgrd,mri,ss}

MRI experiments
typically involve 
repeated cycles of (pulsed) RF excitation;
signal localization (not discussed here);
and transverse $\Tt$ relaxation and free precession, 
alongside (relatively slow) longitudinal $\To$ recovery.
We can build models 
of the received MR signal
by considering the magnetization dynamics
induced by specific pulse sequences.

Classical pulse sequences
use relatively long cycle repetition times $\TR$
to ensure near-complete $\To$ recovery
of the magnetization vector
back to equilibrium state $\mzero\pr \unit{k}$
prior to the start of each RF cycle.
For such long-$\TR$ sequences,
it suffices 
to approximate the magnetization
as fully recovered 
(\ie, $\bmmp\paren{\bmr,t_0 + n\TR} \approx \mzero\pr \unit{k}, 
\forall n\in\set{0,1,2,\dots}$)
just prior to each RF excitation.
This approximation
yields a sequence of initial conditions
and allows computation of the magnetization
at corresponding times of data acquisition
via direct application of Bloch dynamics
\eqref{eq:mtx-ex} and \eqref{eq:mtx-pr}.
Resulting signal models
are typically simple expressions
of relaxation parameters $\To\pr$ and $\Tt\pr$;
however, model accuracy often depends strongly
on the long-$\TR$ assumption,
which requires long acquisitions.

Steady-state (SS) sequences
\cite{hinshaw:76:ifb}
utilize short $\TR$,
and can thus achieve much faster scan times.
Due to short repetition times,
SS sequences achieve only partial $\To$ recovery
in between RF excitations;
thus, their magnetization responses
do not obey the simple classical initial conditions
(for the second RF cycle onwards).
Although their transient magnetization dynamics
can be complicated,
SS sequences produce
(under certain assumptions \cite{scheffler:99:apd})
long-time magnetization responses
that eventually
\footnote{The progression to steady state takes 
	on the order of 
	$5\Tt/\TR$ RF cycles \cite{scheffler:99:apd},
	typically a small but not insignificant period 
	during which data acquisition is often foregone. 
	This transition can
	(in some cases) be accelerated 
	by prepending SS sequences
	with tailored ``magnetization-catalyzing'' modules
	\cite{hargreaves:01:car}.
}
achieve a
steady-state condition:
\begin{align}
	\lim_{t_0 \to \infty} \bmmp\paren{\bmr,t_0+n\TR} = \bmmp\paren{\bmr,t_0},
	\label{eq:ss-cond}
\end{align}
where repetition count 
$n \in \set{0,1,2,\dots}$
for fixed RF excitations
and off-resonance induced phase increments
(as is assumed in the following).
Subsections~\ref{sss,bkgrd,mri,ss,spgr}
and \ref{sss,bkgrd,mri,ss,dess}
use SS condition \eqref{eq:ss-cond}
and Bloch equation matrix operators
introduced in
\eqref{eq:mtx-ex} and \eqref{eq:mtx-pr}
to derive long-time signal models
for Spoiled Gradient-Recalled Echo (SPGR)
and Dual-Echo Steady-State (DESS),
two SS pulse sequences
useful for quantitative MRI.

\subsubsection{Spoiled Gradient-Recalled Echo (SPGR) Sequence}
\label{sss,bkgrd,mri,ss,spgr}



\subsubsection{Dual-Echo Steady-State (DESS) Sequence}
\label{sss,bkgrd,mri,ss,dess}


\section{Relevant Optimization Tools}
\label{s,bkgrd,opt}

\subsection{(Coping with) Non-convex Optimization}
\label{ss,bkgrd,opt,ncvx}

\subsection{Partially Linear Models and the Variable Projection Method}
\label{ss,bkgrd,opt,vpm}