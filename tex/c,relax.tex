% relaxometry

\section{Introduction}
\label{s,relax,intro}

This chapter introduces methods
for QMRI parameter estimation
from statistical models
and applies these methods
to simple problems 
in $\To$ and $\Tt$ relaxometry,
which are of interest
for monitoring the progression
of various disorders \cite{cheng:12:pma}.
Section~\ref{s,relax,meth}
describes a general QMRI signal model,
introduces the notion of a QMRI scan profile,
formulates several likelihood-based estimators
using this model,
and discuss practical implementation issues.
Section~\ref{s,relax,exp}
demonstrates the utility
of likelihood-based parameter estimation
over conventional methods
through simulation, phantom, and \invivo experiments.
Section~\ref{s,relax,summ}
provides brief concluding remarks
and suggests future directions.

\section{Likelihood-Based Estimation in QMRI}
\label{s,relax,meth}
% signal model and construction of scan profile

\subsection{The Signal Model of a QMRI Scan Profile }
\label{ss,relax,meth,sig}

After image reconstruction,
many MRI pulse sequences
useful for parameter estimation
produce at each voxel 
centered at position $\bmr$
a set of noisy voxel values 
$\set{y_1\pr,\dots,y_D\pr}$, 
each of which can be described
with the following general model:
\begin{align}
	y_d\pr = s_d\paren{\bmx\pr; \bmnu\pr} + \epsilon_d\pr,
	\label{eq:relax,mod-scalar}
\end{align}
where 
$d \in \set{1,\dots,D}$. 
Here,
$\bmx\pr \in \complexes{L}$ 
collects $L$ \emph{latent} object parameters at $\bmr$;
$\bmnu\pr \in \complexes{K}$ 
collects $K$ \emph{known} object parameters at $\bmr$;
$s_d : \complexes{L} \times \complexes{K} \mapsto \complex$
is a (pulse-sequence dependent) function
that models the noiseless signal 
obtained from the $d$th dataset;
and $\epsilon_d \sim \cgauss{0}{\sigma_d^2}$ 
is assumed for simplicity 
\footnote{Though the noise distribution 
of $\bmk$-space raw data 
is usually well-modeled
as complex white Gaussian, 
the noise distribution 
of the $d$th reconstructed image $y_d$ depends both
on the acquisition and reconstruction.
If single receive channel $\bmk$-space data
is fully-sampled
on a Cartesian grid,
each dataset $y_d$ is recoverable
via separate Fourier transform,
and is thus complex Gaussian
and independent across datasets.
However if $\bmk$-space data
is multi-channel, undersampled, and/or Cartesian,
it may be preferable
that $y_d$ be estimated by more sophisticated techniques,
\eg \cite{fessler:03:nff, muckley:15:fpm}.
In such cases,
reconstructed image noise is unlikely
to be Gaussian-distributed.
}
to be (circularly-symmetric) complex Gaussian noise
\cite{macovski:96:nim, lei:07:som}
with zero mean and variance $\sigma_d^2$.
(Concrete examples will follow shortly.)

For accurate, well-conditioned QMRI parameter estimation,
it is typically necessary 
to acquire a collection of datasets,
which we refer to hereafter as a \emph{scan profile}.
A scan profile consists 
of $D$ datasets
from up to $D$ pulse sequences
(some sequences yield more than one dataset, \eg DESS).
Let 
$\bmy\pr := \brac{y_1\pr, \dots, y_D\pr}\tpose \in \complexes{D}$
collect voxel values centered at $\bmr$
from a given scan profile.
Then the vector signal model
\begin{align}
	\bmy\pr = \bms\paren{\bmx\pr; \bmnu\pr} + \bmeps\pr
	\label{eq:relax,scalar}
\end{align}
helps define the noiseless signal
$\bms := \brac{s_1, \dots, s_D}\tpose
: \complexes{L} \times \complexes{K} \mapsto \complexes{D}$
associated with that scan profile.
Here, noise
$\bmeps\pr := \brac{\epsilon_1\pr, \dots, \epsilon_D\pr}\tpose \in \complexes{D}$
typically has diagonal covariance structure
$\bmSig := \diag{\brac{\sigma_1,\dots,\sigma_D}\tpose}$
due to independence across datasets,
where $\diag{\cdot}$ assigns its argument 
to the diagonal entries 
of an otherwise zero matrix.

The following subsections
describe two concrete scan profiles
whose signals can be modeled 
via \eqref{eq:relax,scalar} 
and that we study through experiments
later in this chapter.

\subsubsection{Example: $\To$ estimation from SPGR scans}
\label{eq:relax,meth,sig,t1}

\subsubsection{Example: $\Tt$ estimation from DESS scans}
\label{eq:relax,meth,sig,t2}

\subsection{Latent Object Parameter Estimation}
\label{ss,relax,meth,est}
% from likelihood function to...
% ml cost
% rls cost 
% regularizers one might use

\subsection{Practical Considerations}
\label{ss,relax,meth,pract}
% preconditioner design

\section{Experimentation}
\label{s,relax,exp}
% experiments and results
% t1 estimation from spgr
% t2 estimation from dess

\section{Summary}
\label{s,relax,summ}
