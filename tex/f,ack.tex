% acknowledgments 
\setlength{\parindent}{0ex}%
\edit{%
As a skeptical but nevertheless curious Hindu,
I cannot help but believe
that I must have lived a pious prior life
to have landed Jeffrey Fessler and Jon-Fredrik Nielsen
as research co-advisors 
in this lifetime.
Jeff saw research potential in me
before I had even seriously considered graduate school:
he took time to speak to me
and encouraged me to apply
when I visited the University of Michigan (UM)
back in 2010 as an inexperienced sophomore.
Jeff has since then helped me develop
my intuition and yet also my rigor;
my persistence and yet also my open-mindedness;
my confidence and yet also my humility.
I hope to one day lead a team 
with even half as much wisdom and grace
as does Jeff.
Asking Jon to serve as my co-advisor in 2014
(also per Jeff's suggestion)
was perhaps the best decision
I made during graduate school.
Since our earliest meetings,
Jon has effortlessly balanced playing the roles
of educator, collaborator, 
and more recently even tennis partner.
Most practical things 
that I know about MRI today 
are due to Jon.
Together,
Jeff and Jon have provided
a carefully-designed combination
of autonomy and direction
that has been key 
to my development 
as an independent researcher.
}

\setlength{\parindent}{4ex}%
\edit{%
I have benefitted 
from quite fruitful interactions
with the rest of my thesis committee.
Douglas Noll and I have shared many conversations
to try and understand 
which of the many MR non-idealities
dominated algorithmic performance loss in practice
and 
to brainstorm how to compensate for these losses.
Clayton Scott was a key contributor
to the development of PERK (Ch.~\ref{c,perk}).
Scott Swanson and I have shared several conversations
about our new method for myelin water imaging (Ch.~\ref{c,mwf})
and his commentary will be critical 
to its clinical translation.

I have also enjoyed collaborating 
with very capable junior students.
Undergraduate student Mingjie Gao 
helped adapt PERK to MR fingerprinting data
\cite{nataraj:17:slw}
and ran myelin water imaging scan optimizations 
(used in Ch.~\ref{c,mwf}).
I recently learned
that he will proceed with graduate study
working with Jeff
and wish him good luck.
First-year graduate student Steven Whittaker
has begun to work
on exploiting off-resonance for myelin water imaging
(discussed in Ch.~\ref{c,future}).
He plans to lead our group's 
myelin water imaging research thrust
in the coming years.

I am grateful to UM
for the resources it provided
and for the relationships it enabled.
UM funded a large portion
of this early-stage and exploratory research
through grants and fellowships.
I met many wonderful people 
through the ``lab of Jeff'',
the functional MRI lab,
the graduate student council, 
teaching,
tutoring,
coursework,
and probably other venues too.
You have all made 
what is ultimately a solitary journey
much more fun and fulfilling -- thank you.

I next thank my roommates,
Adam Mendrela and John ``Trey'' Ruppe.
All three of us moved here together
from Cornell back in 2012.
None of us wore much facial hair back then
(Trey still does not).
I suppose Trey won the race to a doctorate;
he now works in the greater Boston area.
Adam has lived with me 
in the same apartment
for the entire duration of our PhDs;
I wish him the best of luck
as he wraps things up in the coming months.
These guys have helped me
work through every single one 
of the highs and lows
of graduate school.

I next thank my fianc\'{e}e, Manisha Rai.
Simply put,
she is the light of my life.
Manisha,
I cannot wait to get married
and to start the next chapter 
of our lives together.
I hope I can be even half as supportive
during your residency training
as you have been these past three years.

Finally,
I thank my family.
I doubt many people can say
that three generations (9 members!) of family
attended their PhD defense.
I am humbled by your unconditional love and support.
This thesis is at least as much yours as it is mine.
}%
