% myelin water fraction estimation from steady state sequences

%%%%%%%%%%%%%%%%%%%%%%%%%%%%%%%%%%%%%%%%%%%%%%%%%%%
\section{Introduction}
\label{s,mwf,intro}
%%%%%%%%%%%%%%%%%%%%%%%%%%%%%%%%%%%%%%%%%%%%%%%%%%%

Myelin is a lipid-rich material
that forms an insulating sheath
encasing neuronal axons 
predominantly in white matter (WM) regions
of the human brain
\cite{morell:84}.
Demyelination (\ie, myelin loss) is central
to the development 
of several neurodegenerative disorders
such as multiple sclerosis (MS)
\cite{goldenberg:12:msr}. 
Non-invasive myelin quantification in WM
is thus desirable 
for monitoring the onset and progression
of neurodegenerative disease.

% MR relaxation rates (esp t2) depend on water environment
% local water environment changes on sub-millimeter scale
% MR sensitive only on millimeter scale
% need to assign a macroscopic t2 to each pool or compartment
% thus multiple water compartments contribute to each voxel at typical resolutions
MR relaxation time constants
(especially spin-spin time constant $\Tt$)
depend on the macromolecular environment
surrounding excited water molecules.
In nervous tissue,
these environments vary spatially
on scales much smaller 
than the millimeter-scale resolutions
used in typical MR imaging experiments.
Thus, 
there is significant variation
of relaxation times 
within a typical imaging voxel 
containing nervous tissue.

% natural to try to separate t2 components in mri
% several groups measured first in vitro short t2 pool in white matter
% assigned to water in myelin (trapped in bilayers?)
% later measured in vivo, who termed mwf (mackay)
% mwf found to correlate well in vivo with histology (laule)
% mwf thus noninvasive MR-based biomarker for myelin content
Many researchers have attempted 
to characterize tissue microstructure
by estimating the distribution
of MR relaxation time constants
and associating certain ranges of time constants
with particular ``compartments'' or ``pools'' of water molecules
that exist in similar macromolecular environments.
Several \invitro NMR studies
of nervous animal tissue
prescribed a fast-relaxing water compartment
with $\Tt\sim$10-40ms 
initially to general protein 
and phospholipid structures \cite{vasilescu:78:wci}
and later more specifically
to water trapped between
the phospholipid bilayers 
of myelin
\cite{menon:91:aoc, stewart:93:ssr}.
Shortly thereafter,
the first MR images 
of so-called \emph{myelin water fraction} (MWF),
defined as the proportion of MR signal 
arising from the fast-relaxing water compartment
relative to total MR signal,
were demonstrated \emph{in vivo}
in the human brain \cite{mackay:94:ivv}.
More recently,
MWF has been shown 
to correlate well 
with histological measurements
of myelin content 
in animal models
of nerve injury \cite{gareau:00:mta}
and demyelination \cite{webb:03:imt}.
In humans, 
MWF has also been measured 
to be markedly lower
in ``normally appearing'' WM 
of MS patients versus controls \cite{laule:04:wca},
and to correlate strongly 
with post-mortem histological measurements
of myelin content
in MS patients \cite{laule:06:mwi}.
Thus,
there is reasonably strong evidence
that MWF
(as originally measured in \cite{mackay:94:ivv})
is a specific and non-invasive biomarker
for myelin content in WM.

% all aforementioned studies used MESE 
% MESE typically uses long tr to ensure full recovery between excitations
% thus MESE acquisitions take long time
% mcdespot based on ss techniques (specifically spgr and bssfp) 
% shown to disagree significantly with ss 
% maybe due to insufficient precision
All of the aforementioned studies
estimate MWF images
from a multi-echo spin echo (MESE) MRI pulse sequence
\cite{carr:54:eod}
with long repetition time $\TR\geq2$s
to ensure sufficient recovery
of the longitudinal magnetization 
in nervous tissue.
Whole-brain MWF imaging 
using such long-$\TR$ MESE acquisitions
at a typical imaging resolution
would require hours of scan time
and thus may not be clinically feasible.
As a more practical alternative,
scan profiles consisting of short-$\TR$ steady-state (SS) sequences
were proposed 
for whole-brain MWF imaging in about 30m scan time
\cite{deoni:08:gmt}.
Despite recent further refinements
\cite{deoni:11:com, deoni:13:oct},
MWF images from SS pulse sequences 
have thus far been shown
to be incomparable with MWF images
from MESE pulse sequences
\cite{zhang:15:com},
likely due in part
to insufficient unbiased parameter estimation precision
\cite{lankford:13:oti}.

This chapter introduces
a rapid SS MRI scan profile
for precise MWF imaging.
We apply QMRI acquisition design
(developed in Chapter~\ref{c,scn-dsgn})
to optimize the flip angles and repetition times
of combinations 
of spoil gradient-recalled echo (SPGR)
and dual-echo steady-state (DESS) sequences
for precise MWF estimation.
We rapidly estimate MWF and 
other nuisance latent object parameters
using QMRI parameter estimation
via kernel ridge regression (KRR)
(developed in Chapter~\ref{c,krr}).
We obtain proof-of-concept MWF maps \invivo
that are comparable
to results reported
in MESE literature.

The remainder of this chapter
is organized as follows.
Section~\ref{s,mwf,model} reviews and develops
simple two-compartment signal models
for SPGR and DESS pulse sequences, 
respectively.
Section~\ref{s,mwf,acq} designs 
a new SPGR/DESS scan profile 
for precisely estimating MWF parameter images
from SPGR/DESS image data.
Section~\ref{s,mwf,exp} applies
the proposed acquisition \invivo
and qualitatively compares resultant MWF images
to state-of-the-art MESE results.
Section~\ref{s,mwf,summ} summarizes contributions thus far
and discusses future work. 

%%%%%%%%%%%%%%%%%%%%%%%%%%%%%%%%%%%%%%%%%%%%%%%%%%%
\section{Multi-Compartmental Models for SS Sequences}
\label{s,mwf,model}
%%%%%%%%%%%%%%%%%%%%%%%%%%%%%%%%%%%%%%%%%%%%%%%%%%%

%%%%%%%%%%%%%%%%%%%%%%%%%%%%%%%%%%%%%%%%%%%%%%%%%%%
\subsection{A Two-Compartment DESS Model}
\label{ss,mwf,model,dess}

% dess-2comp notes

%%%%%%%%%%%%%%%%%%%%%%%%%%%%%%%%%%%%%%%%%%%%%%%%%%%
\subsection{A Two-Compartment SPGR Model}
\label{ss,mwf,model,spgr}

% state results from deoni

%%%%%%%%%%%%%%%%%%%%%%%%%%%%%%%%%%%%%%%%%%%%%%%%%%%
\section{SPGR/DESS Acquisition Design for MWF Imaging}
\label{s,mwf,acq}
%%%%%%%%%%%%%%%%%%%%%%%%%%%%%%%%%%%%%%%%%%%%%%%%%%%

% ismrm conf paper

%%%%%%%%%%%%%%%%%%%%%%%%%%%%%%%%%%%%%%%%%%%%%%%%%%%
\section{Experimentation}
\label{s,mwf,exp}
%%%%%%%%%%%%%%%%%%%%%%%%%%%%%%%%%%%%%%%%%%%%%%%%%%%

% maybe repeat some details from ch 4
% krr details
% preliminary image

%%%%%%%%%%%%%%%%%%%%%%%%%%%%%%%%%%%%%%%%%%%%%%%%%%%
\section{Summary and Future Work}
\label{s,mwf,summ}
%%%%%%%%%%%%%%%%%%%%%%%%%%%%%%%%%%%%%%%%%%%%%%%%%%%

\todo{discuss negative $\ff$ range}
