% myelin water fraction estimation from steady state sequences

%%%%%%%%%%%%%%%%%%%%%%%%%%%%%%%%%%%%%%%%%%%%%%%%%%%
\section{Introduction}
\label{s,mwf,intro}
%%%%%%%%%%%%%%%%%%%%%%%%%%%%%%%%%%%%%%%%%%%%%%%%%%%

Myelin is a lipid-rich material
that forms an insulating sheath
encasing neuronal axons 
predominantly in white matter (WM) regions
of the human brain
\cite{morell:84}.
Demyelination (\ie, myelin loss) is central
to the development 
of several neurodegenerative disorders
such as multiple sclerosis (MS)
\cite{goldenberg:12:msr}. 
Non-invasive myelin quantification in WM
is thus desirable 
for monitoring the onset and progression
of neurodegenerative disease.

% MR relaxation rates (esp t2) depend on water environment
% local water environment changes on sub-millimeter scale
% MR sensitive only on millimeter scale
% need to assign a macroscopic t2 to each pool or compartment
% thus multiple water compartments contribute to each voxel at typical resolutions
MR relaxation time constants
(especially spin-spin time constant $\Tt$)
depend on the macromolecular environment
surrounding excited water molecules.
In nervous tissue,
these environments vary spatially
on scales much smaller 
than the millimeter-scale resolutions
used in typical MR imaging experiments.
Thus, 
there is significant variation
of relaxation times 
within a typical imaging voxel 
containing nervous tissue.

% natural to try to separate t2 components in mri
% several groups measured first in vitro short t2 pool in white matter
% assigned to water in myelin (trapped in bilayers?)
% later measured in vivo, who termed mwf (mackay)
% mwf found to correlate well in vivo with histology (laule)
% mwf thus noninvasive MR-based biomarker for myelin content
Many researchers have attempted 
to characterize tissue microstructure
by estimating the distribution
of MR relaxation time constants
and associating certain ranges of time constants
with particular ``compartments'' or ``pools'' of water molecules
that exist in similar macromolecular environments.
Several \invitro NMR studies
of nervous animal tissue
prescribed a fast-relaxing water compartment
with $\Tt\sim$10-40ms 
initially to general protein 
and phospholipid structures \cite{vasilescu:78:wci}
and later more specifically
to water trapped between
the phospholipid bilayers 
of myelin
\cite{menon:91:aoc, stewart:93:ssr}.
Shortly thereafter,
the first MR images 
of so-called \emph{myelin water fraction} (MWF),
defined as the proportion of MR signal 
arising from the fast-relaxing water compartment
relative to total MR signal,
were demonstrated \emph{in vivo}
in the human brain \cite{mackay:94:ivv}.
More recently,
MWF has been shown 
to correlate well 
with histological measurements
of myelin content 
in animal models
of nerve injury \cite{gareau:00:mta}
and demyelination \cite{webb:03:imt}.
In humans, 
MWF has also been measured 
to be markedly lower
in ``normally appearing'' WM 
of MS patients versus controls \cite{laule:04:wca},
and to correlate strongly 
with post-mortem histological measurements
of myelin content
in MS patients \cite{laule:06:mwi}.
Thus,
there is reasonably strong evidence
that MWF
(as originally measured in \cite{mackay:94:ivv})
is a specific and non-invasive biomarker
for myelin content in WM.

% all aforementioned studies used MESE 
% MESE typically uses long tr to ensure full recovery between excitations
% thus MESE acquisitions take long time
% mcdespot based on ss techniques (specifically spgr and bssfp) 
% shown to disagree significantly with ss 
% maybe due to insufficient precision
All of the aforementioned studies
estimate MWF images
from a multi-echo spin echo (MESE) MRI pulse sequence
\cite{carr:54:eod}
with long repetition time $\TR\geq2$s
to ensure sufficient recovery
of the longitudinal magnetization 
in nervous tissue.
Whole-brain MWF imaging 
using such long-$\TR$ MESE acquisitions
at a typical imaging resolution
would require hours of scan time
and thus may not be clinically feasible.
As a more practical alternative,
scan profiles consisting of short-$\TR$ steady-state (SS) sequences
were proposed 
for whole-brain MWF imaging in about 30m scan time
\cite{deoni:08:gmt}.
Despite recent further refinements
\cite{deoni:11:com, deoni:13:oct},
MWF images from SS pulse sequences 
have thus far been shown
to be incomparable with MWF images
from MESE pulse sequences
\cite{zhang:15:com},
likely due in part
to insufficient unbiased parameter estimation precision
\cite{lankford:13:oti}.

This chapter introduces
a rapid SS MRI scan profile
for precise MWF imaging.
We apply QMRI acquisition design
(developed in Chapter~\ref{c,scn-dsgn})
to optimize the flip angles and repetition times
of combinations 
of spoiled gradient-recalled echo (SPGR)
\cite{zur:91:sot}
and dual-echo steady-state (DESS) 
\cite{redpath:88:fan, bruder:88:ans} sequences
for precise MWF estimation.
We rapidly estimate MWF and 
other nuisance latent object parameters
using QMRI parameter estimation
via kernel ridge regression (KRR)
(developed in Chapter~\ref{c,krr}).
We obtain proof-of-concept MWF maps \invivo
that are comparable
to results reported
in MESE literature.

The remainder of this chapter
is organized as follows.
Section~\ref{s,mwf,model} reviews and develops
simple two-compartment signal models
for SPGR and DESS pulse sequences, 
respectively.
Section~\ref{s,mwf,acq} designs 
a new SPGR/DESS scan profile 
for precisely estimating MWF parameter images
from SPGR/DESS image data.
Section~\ref{s,mwf,exp} applies
the proposed acquisition \invivo
and qualitatively compares resultant MWF images
to state-of-the-art MESE results.
Section~\ref{s,mwf,summ} summarizes contributions thus far
and discusses future work. 

%%%%%%%%%%%%%%%%%%%%%%%%%%%%%%%%%%%%%%%%%%%%%%%%%%%
\section{Multi-Compartmental Models for SS Sequences}
\label{s,mwf,model}
%%%%%%%%%%%%%%%%%%%%%%%%%%%%%%%%%%%%%%%%%%%%%%%%%%%

This section develops multi-compartmental signal models
for the SPGR and DESS pulse sequences.
Subsection~\ref{ss,mwf,model,spgr} reviews and extends
a concise Bloch-matrix derivation \cite{spencer:00:mos}
of an SPGR signal model 
that accounts for exchange 
between multiple compartments.
Subsection~\ref{ss,mwf,model,dess} applies 
the Bloch-matrix representation
to derive analogous 
(but previously unpublished)
multi-compartmental DESS signal models. 
Though in the derivations below
we focus for simplicity
on only two exchanging compartments,
the Bloch-matrix formulation allows
for straightforward generalization
to three or more interacting compartments.

%%%%%%%%%%%%%%%%%%%%%%%%%%%%%%%%%%%%%%%%%%%%%%%%%%%
\subsection{A Two-Compartment SPGR Signal Model}
\label{ss,mwf,model,spgr}

The McConnell equations \cite{mcconnell:58:rrb}
extend the Bloch equations \cite{bloch:1946:ni-paper}
to account for first-order reversible chemical exchange
between two or more intra-voxel compartments. 
Here,
we specifically consider the interaction
of a fast-relaxing water compartment
(characterized by comparatively short spin-lattice $\tf{1}$ 
and spin-spin $\tf{2}$ relaxation times) 
with a slow-relaxing water compartment
(characterized by longer relaxation times $\ts{1},\ts{2}$).
In primed coordinates rotating clockwise 
about the longitudinal $z$-axis
at the Larmor frequency,
the dynamics 
of corresponding fast-relaxing and slow-relaxing
compartmental magnetization vectors
$\bmmpf := \brac{\mxpf, \mypf, \mzpf}\tpose$
and
$\bmmps := \brac{\mxps, \myps, \mzps}\tpose$
are coupled via first-order exchange rates 
$\rfs$ (from fast to slow compartment) 
and $\rsf$ (vice-versa).
In the absence of RF excitation,
these magnetization dynamics
decouple in transverse and longitudinal components;
the two-compartment transverse equations 
extend \eqref{eq:bloch-free-mxyp} to read
\begin{align}
	\dela{t} \mxypf\prt &= 
		-i\gam \mxypf\prt \ompf\pr - \frac{\mxypf\prt}{\tf{2}\pr} 
		- \rfs\pr \mxypf\prt + \rsf\pr \mxyps\prt;
		\label{eq:mwf,mxy-f} \\
	\dela{t} \mxyps\prt &= 
		-i\gam \mxyps\prt \omps\pr - \frac{\mxyps\prt}{\ts{2}\pr} 
		- \rsf\pr \mxyps\prt + \rfs\pr \mxypf\prt,
		\label{eq:mwf,mxy-s}
\end{align}
where 
$\mxypf\prt := \mxpf\prt + i\mypf\prt$ 
and $\mxyps\prt := \mxps\prt + i\myps\prt$
are complex compartmental representations
of the transverse magnetization
at position $\bmr$ 
and time $t$; 
$\ompf\pr \in \real$ and $\omps\pr \in \real$ 
allow for compartment-specific 
but time-invariant off-resonance effects;
and $\gam \in \real$ is the gyromagnetic ratio.
Without excitation,
analogous two-compartment longitudinal equations 
extend \eqref{eq:bloch-free-mzp} to read
\begin{align}
	\dela{t} \mzpf\prt &=
		-\frac{\mzpf\prt - \ff\pr\mzero\pr}{\tf{1}\pr} 
		- \rfs\pr \mzpf\prt + \rsf\pr \mzps\prt;
		\label{eq:mwf,mz-f} \\
	\dela{t} \mzps\prt &= 
		-\frac{\mzps\prt - \fs\pr\mzero\pr}{\ts{1}\pr}
		- \rsf\pr \mzps\prt + \rsf\pr \mzpf\prt,
		\label{eq:mwf,mz-s}
\end{align}
where $\ff\pr \in \brac{0,1}$ and $\fs\pr \in \brac{0,1}$ 
denote fast- and slow-relaxing
compartmental fractions
and $\mzero\pr$ denotes the total 
\footnote{The numerous parameters introduced here
	are often interdependent. 
	In regions with only two water compartments present,
	$\ff\pr + \fs\pr = 1$.
	In regions where only two water pools exhibit exchange 
	and that exchange is in chemical equilibrium,
	$\ff\pr \rfs\pr = \fs\pr \rsf\pr$.
	\label{foot:constraint}
}
equilibrium magnetization.
 
Equations~\eqref{eq:mwf,mxy-f}-\eqref{eq:mwf,mz-s}
comprise a first-order affine dynamical system
and can be equivalently and concisely written
in matrix form as
\begin{align}
	\dela{t} \bmmp\prt = \bmA\pr \bmmp\prt + \bmc\pr,
	\label{eq:mwf,mdot}
\end{align}	
where $\bmmp\prt := \vect{\brac{\bmmpf\prt, \bmmps\prt}\tpose} \in \reals{6}$
collects the compartmental magnetization vectors
and $\vect{\cdot}$ denotes vectorization;
system matrix $\bmA\pr \in \reals{6 \times 6}$ 
admits a block diagonal form
$\bmA\pr \equiv \brac{\bmAxy\pr, \zeros{4 \times 2}; \zeros{2 \times 4}, \bmAz\pr}$,
with $\bmAxy\pr :=$
\begin{align}
	\begin{bmatrix}
		-\frac{1}{\tf{2}\pr}-\rfs\pr & \rsf\pr & \ompf\pr & 0 \\
		\rfs\pr & -\frac{1}{\ts{2}\pr}-\rsf\pr & 0 & \omps\pr \\
		-\ompf\pr & 0 & -\frac{1}{\tf{2}\pr}-\rfs\pr & \rsf\pr \\
		0 & -\omps\pr & \rfs\pr & -\frac{1}{\ts{2}\pr}-\rsf\pr 
	\end{bmatrix}
	\label{eq:mwf,Axy}
\end{align}
collecting transverse dynamics and
\begin{align}
	\bmAz\pr :=
	\begin{bmatrix}
		-\frac{1}{\tf{1}\pr}-\rfs\pr & \rsf\pr \\
		\rfs\pr & -\frac{1}{\ts{1}\pr}-\rsf\pr
		\label{eq:mwf,Az}
	\end{bmatrix}
\end{align}
collecting longitudinal dynamics; and
$\bmc\pr := \brac{0, 0, 0, 0, 
\frac{\ff\pr \mzero\pr}{\tf{1}}, 
\frac{\fs\pr \mzero\pr}{\ts{1}}}\tpose$.
If $\bmA\pr$ is invertible,
a matrix exponential solution 
to \eqref{eq:mwf,mdot}
exists and reads
\begin{align}
	\bmmp\prt = e^{t\bmA\pr} \bmmp\prtz +
		\paren{e^{t\bmA\pr} - \eye{6}} \inv{\bmA\pr} \bmc\pr
		\qquad \forall t \geq t_0,
		\label{eq:mwf,epr}
\end{align}
where $\bmmp\prtz$ is the magnetization
at an initial time $t_0$
and $\eye{6} \in \reals{6 \times 6}$ 
is an identity matrix.
In chemical equilibrium
(described in Footnote~\ref{foot:constraint}),
a direct calculation involving 
compartmental equilibrium magnetization
$\bmmzero\pr := \brac{0,0,0,0,\ff\pr\mzero\pr,\fs\pr\mzero\pr}\tpose$
reveals that $\bmA\pr \bmmzero\pr = -\bmc\pr$,
in which case \eqref{eq:mwf,epr} may be simplified as
\begin{align}
	\bmmp\prt = e^{t\bmA\pr} \bmmp\prtz +
		\paren{\eye{6} - e^{t\bmA\pr}} \bmmzero\pr
		\qquad \forall t \geq t_0,
		\label{eq:mwf,epr-eq}
\end{align}
a form
that bears much resemblance  
to the matrix operator solution
\eqref{eq:mtx-pr}
to the single-compartment Bloch equations
in the presence
of only free precession and relaxation.

To develop a two-compartment SPGR signal model,
we modify the derivation 
of the single-compartment SPGR model
(presented in Subsection~\ref{sss,bkgrd,mri,ss,spgr})
to account for inter-compartmental exchange
(via \eqref{eq:mwf,epr-eq})
during intervals of free precession and relaxation.
As before,
let $\bmmp\prtz$ denote the magnetization
at an initial time $t_0$ selected 
well into the steady-state
and just prior to RF excitation.
The SPGR sequence first applies
an RF excitation
with pulse duration $\TP$.
If we neglect relaxation,
off-resonance,
and exchange during excitation
(which is reasonable 
for sufficiently short $\TP$) 
and further assume
that each compartment 
experiences this excitation
with equal transmit field sensitivity,
we may model RF excitation as a rotation
by a compartment-wise constant 
(but spatially varying)
nutation angle $\flip\prtzt{t_0+\TP}$
(defined in \eqref{eq:flip-def}),
abbreviated $\flip\pr$ hereafter.
For clockwise rotations
about the $x'$-axis,
such near-instantaneous rotation
may be represented as
\begin{align}
	\bmmp\paren{\bmr,t_0+\TP} = 
		\bmRxpsix{\flip\pr} \bmmp\prtz,
		\label{eq:mwf,spgr-ex}
\end{align}
where $\bmRxpsix{\flip\pr} := \bmRxp{\flip\pr} \otimes \eye{2} 
\in \reals{6 \times 6}$
is a compartment-wise rotation operation;
$\bmRxp{\flip\pr}$ is defined in \eqref{eq:rot-sol};
and $\otimes$ denotes the Kronecker product.

The compartments exchange
while their magnetization vectors precess and relax
as per \eqref{eq:mwf,epr}
until data acquisition
at echo time $\TE \in \brac{\frac{\TP}{2},\TR}$
after the midpoint of RF excitation:
\begin{align}
	\bmmp\paren{\bmr,t_0+\frac{\TP}{2}+\TE} =
		e^{\paren{\TE-\frac{\TP}{2}}\bmA\pr} \bmmp\paren{\bmr,t_0+\TP} 
		+ \paren{\eye{6} - e^{\paren{\TE-\frac{\TP}{2}}\bmA\pr}} \bmmzero\pr.
	\label{eq:mwf,spgr-daq}
\end{align}
Following signal reception,
the remaining transverse magnetization
(in both compartments) 
is spoiled 
while longitudinal magnetization components
in each compartment
remain unaffected.
We model ideal spoiling 
in both compartments as 
\begin{align}
	\bmSsix \bmmp\paren{\bmr,t_0+\frac{\TP}{2}+\TE},
	\label{eq:mwf,spgr-spoil}
\end{align}
where $\bmSsix := \bmS \otimes \eye{2} \in \reals{6 \times 6}$
is a compartment-wise spoiling operator
and $\bmS$ is defined in \eqref{eq:spgr-spoil}.
After spoiling,
compartments continue to exchange
while their longitudinal magnetization components
partially recover
until $t \gets t_0 + \TR$,
completing one repetition cycle:
\begin{align}
	\bmmp\paren{\bmr,t_0+\TR} &=
		e^{\paren{\TR-\paren{\frac{\TP}{2}+\TE}}\bmA\pr} \bmSsix 
		\bmmp\paren{\bmr,t_0+\frac{\TP}{2}+\TE} 
		\nonumber \\
		&+ \paren{\eye{6} - e^{\paren{\TR-\paren{\frac{\TP}{2}+\TE}}\bmA\pr}} \bmmzero\pr.
	\label{eq:mwf,spgr-epr}
\end{align}
In steady-state, one cycle of excitation, acquisition, spoiling, and recovery 
returns the magnetization back to its initial state.
We enforce this through the steady-state condition
\begin{align}
	\bmmp\paren{\bmr,t_0+\TP} = 
		\bmRxpsix{\flip\pr} \bmmp\paren{\bmr,t_0+\TR},
		\label{eq:mwf,spgr-ss}
\end{align}
which yields an algebraic system
of equations.
When it exists,
the solution is
\begin{align}
	\bmmp\paren{\bmr,t_0+\TP} &=
		\inv{\eye{6}-\bmRxpsix{\flip\pr} \bmSsix e^{\paren{\TR-\TP}\bmA\pr}} \bmRxpsix{\flip\pr}
		\label{eq:mwf,spgr-m-t0-alg} \\
	&\times \paren{\eye{6}-\bmSsix e^{\paren{\TR-\TP}\bmA\pr} -
		e^{\paren{\TR-\paren{\frac{\TP}{2}+\TE}}\bmA\pr}\paren{\eye{6}-\bmSsix}}
		\bmmzero\pr
		\nonumber \\
	&= \inv{\eye{6}-\bmRxpsix{\flip\pr} \bmSsix e^{\paren{\TR-\TP}\bmA\pr}} \bmRxpsix{\flip\pr}
		\label{eq:mwf,spgr-m-t0} \\
	&\times \paren{\eye{6}-\bmSsix e^{\paren{\TR-\TP}\bmA\pr}} \bmmzero\pr,
		\nonumber 
\end{align}
where \eqref{eq:mwf,spgr-m-t0-alg} is due
to straightforward matrix operations
and \eqref{eq:mwf,spgr-m-t0} makes use 
of the ideal spoiling property
$\paren{\eye{6}-\bmSsix}\bmmzero\pr = \zeros{6}\xspace \forall\bmr$.
Substituting \eqref{eq:mwf,spgr-m-t0}
into \eqref{eq:mwf,spgr-daq}
yields an expression
for the compartmental magnetization 
at the echo time.

The matrix exponentials 
in \eqref{eq:mwf,spgr-m-t0} 
are cumbersome to expand in general
but fortunately always arise
prepended by ideal spoiling operator $\bmSsix$.
The combined form simplifies as
\begin{align}
	\bmSsix e^{\paren{\TR-\TP}\bmA\pr} \equiv
		\brac{\zeros{4 \times 4}, \zeros{4 \times 2}; 
		\zeros{2 \times 4}, e^{\paren{\TR-\TP} \bmAz\pr}}
	\label{eq:mwf,spgr-spoil-simp}
\end{align}
and admits 
an explicit representation 
for \eqref{eq:mwf,spgr-m-t0}
that can be found 
via standard solvers. 
Note that
if RF pulses are assumed instantaneous
(\ie, $\TP \gets 0$)
the longitudinal components
of \eqref{eq:mwf,spgr-m-t0}
are equivalent to \cite[Eq.~34]{spencer:00:mos}.
Another instructive (and much simpler) special case
neglects exchange
(\ie, $\rfs \gets 0$ and $\rsf \gets 0$),
in which case \eqref{eq:mwf,spgr-m-t0} reduces to
\begin{align}
	\bmmp\paren{\bmr,t_0+\TP} =
	\begin{bmatrix}
		0 \\
		0 \\
		\frac{\ff\pr \sina{\flip\pr}\paren{1-e^{-\paren{\TR-\TP}/\tf{1}\pr}}}
			{1-e^{-\paren{\TR-\TP}/\tf{1}\pr}\cosa{\flip\pr}} \\
		\frac{\fs\pr \sina{\flip\pr}\paren{1-e^{-\paren{\TR-\TP}/\ts{1}\pr}}}
			{1-e^{-\paren{\TR-\TP}/\ts{1}\pr}\cosa{\flip\pr}} \\
		\frac{\ff\pr \cosa{\flip\pr}\paren{1-e^{-\paren{\TR-\TP}/\tf{1}\pr}}}
			{1-e^{-\paren{\TR-\TP}/\tf{1}\pr}\cosa{\flip\pr}} \\
		\frac{\fs\pr \cosa{\flip\pr}\paren{1-e^{-\paren{\TR-\TP}/\ts{1}\pr}}}
			{1-e^{-\paren{\TR-\TP}/\ts{1}\pr}\cosa{\flip\pr}}
	\end{bmatrix}
	\mzero\pr,
	\label{eq:mwf-spgr-m-t0-exchg0}
\end{align}
a two (non-exchanging) compartment extension
to the SPGR steady-state solution \eqref{eq:spgr-bmmp-t0}.

The received signal
is approximately proportional
to the integrated transverse magnetization
arising from both compartments.
To derive expressions,
we take further assumptions,
the first two of which are two-compartment extensions 
of assumptions used 
in \S\ref{sss,bkgrd,mri,ss,spgr}: 
\begin{enumerate}
	\item
		We assume
		that the signal is localized
		to a scale over which 
		there is within-voxel variation
		of each compartment's off-resonance frequency,
		but minimal intra-voxel variation
		of other space-varying parameters
		$\mzero, \ff, \fs, \tf{1}, \ts{1}, \tf{2}, \ts{2}, \rfs, \rsf$.
		This assumption effectively fixes all compartmental properties 
		other than $\ompf$ and $\omps$
		over the volume $\setV$ of a sufficiently small voxel.
		\label{item:spgr,int}
		
	\item 
		We assume 
		that the fast-relaxing
		and slow-relaxing compartments' off-resonance frequencies	
		are independently distributed 
		within a localized voxel 
		with marginal distributions
		$\dist{\ompf} := \operatorname{Cauchy}\paren{\ompfmed,\Rtpf}$
		and $\dist{\omps} := \operatorname{Cauchy}\paren{\ompsmed,\Rtps}$,
		where $\ompfmed,\ompsmed$ are median off-resonance frequencies
		and $\Rtpf,\Rtps$ are broadening bandwidths.
		\label{item:spgr,freq}
		
	\item 
		We assume 
		for short echo time $\TE$
		that negligible exchange occurs
		between excitation and signal reception.
		This assumption facilitates expansion 
		of the matrix exponentials
		in \eqref{eq:mwf,spgr-daq}
		and separates broadening integrals 
		by compartment.
		\label{item:spgr,exchg0}
\end{enumerate}
With these assumptions,
the noiseless two-compartment steady-state SPGR signal model
for a voxel
centered at position $\bmr$ 
and with volume defined by $\setV\pr$ 
is (to within constants):
\begin{alignat}{2}
	\spgr\paren{\bmr,t_0+\frac{\TP}{2}+\TE} 
		&\propto&
			&\int_{\setV\pr} \brac{1, 1, i, i, 0, 0} \bmmp\paren{\bmr,t_0+\frac{\TP}{2}+\TE} \dercubed{\bmr} 
			\label{eq:mwf,spgr,int} \\
		&\approx& 
			&\paren{\int_{\real^2} \brac{1, 1, i, i, 0, 0}
				e^{\paren{\TE-\frac{\TP}{2}}\bmA\pr} 
				\dista{\ompf} \dista{\omps} \der{\ompf} \der{\omps}} 
				\bmmp\paren{\bmr,t_0+\TP} 
				\label{eq:mwf,spgr,int-freq} \\
		&=&
			&\begin{bmatrix}
				e^{-\paren{1/\tf{2}\pr+\Rtpf\pr+i\ompfmed\pr}\paren{\TE-\frac{\TP}{2}}} \\
				e^{-\paren{1/\ts{2}\pr+\Rtps\pr+i\ompsmed\pr}\paren{\TE-\frac{\TP}{2}}} \\
				ie^{-\paren{1/\tf{2}\pr+\Rtpf\pr+i\ompfmed\pr}\paren{\TE-\frac{\TP}{2}}} \\
				ie^{-\paren{1/\ts{2}\pr+\Rtps\pr+i\ompsmed\pr}\paren{\TE-\frac{\TP}{2}}} \\
				0 \\
				0
			\end{bmatrix}
			\tpose \bmmp\paren{\bmr,t_0+\TP}
			\label{eq:mwf,spgr,int-exchg0} \\
		&=\,& 
			&\mxypf\paren{\bmr,t_0+\TP} 
				e^{-\paren{1/\tf{2}\pr+\Rtpf\pr+i\ompfmed\pr}\paren{\TE-\frac{\TP}{2}}} +
				\nonumber \\
		&& 
			&\mxyps\paren{\bmr,t_0+\TP} 
				e^{-\paren{1/\ts{2}\pr+\Rtps\pr+i\ompsmed\pr}\paren{\TE-\frac{\TP}{2}}}.
			\label{eq:mwf,spgr,mod-exchg0}
\end{alignat}
Observe that \eqref{eq:mwf,spgr,int-freq}
uses Assumption~\ref{item:spgr,int}
to leave $\bmmp\paren{\bmr,t_0+\TP}$
outside of broadening integrals
defined by Assumption~\ref{item:spgr,freq}
and 
\eqref{eq:mwf,spgr,int-exchg0}
uses Assumption~\ref{item:spgr,exchg0}
to expand the matrix exponential
and evaluate broadening integrals component-wise.
Eq.~\eqref{eq:mwf,spgr,mod-exchg0} naturally extends
the single-compartment SPGR model \eqref{eq:spgr-model} 
and shows clearly
that compartmental signals simply add
if exchange between excitation and signal reception is neglected. 

% here very simple model
% but already shows nontrivial phase dependence
% on ompmed for compartments in general 
Though the manipulations leading
to \eqref{eq:mwf,spgr,mod-exchg0} require strong assumptions,
they serve to minimally demonstrate 
the nontrivial two-compartment SPGR signal dependence
on off-resonance distributions.
Unlike in the single-compartment case,
the signal decay and dephasing terms
due to off-resonance effects
$e^{-\paren{\Rtpf+i\ompfmed}\TE}$ and $e^{-\paren{\Rtps+i\ompsmed}\TE}$
may \emph{not} be considered
to simply modify spin density $\mzero$
(in fact,
to first order
they modify compartmental fractions $\ff$ and $\fs$
of usual interest).
Since off-resonance distributions
do often differ significantly in cerebral tissue
\cite{miller:10:aot-I, miller:10:aot-II}, 
accurate \invivo parameter estimation
from this model 
requires either joint estimation
of compartmental distributions
or acquisition with echo times much shorter
than compartmental broadening timescale differences. 

%%%%%%%%%%%%%%%%%%%%%%%%%%%%%%%%%%%%%%%%%%%%%%%%%%%
\subsection{A Two-Compartment DESS Signal Model}
\label{ss,mwf,model,dess}

To develop two-compartment DESS signal models,
we first adapt the McConnell equation solutions
of Subsection~\ref{ss,mwf,model,spgr}
to describe compartmental magnetization evolution
in the presence of time-dependent field inhomogeneities.
We then modify corresponding derivations
of single-compartment DESS models
(presented in Subsection~\ref{sss,bkgrd,mri,ss,dess}),
to account for inter-compartmental exchange 
during intervals of free precession and relaxation.

As discussed
in Subsection~\ref{sss,bkgrd,mri,ss,dess},
the DESS pulse sequence interlaces fixed RF excitations
with fixed dephasing gradients
to produce two distinct signals
per excitation.
Because these dephasing gradients
create time-dependence
in compartmental off-resonance effects,
DESS signal dynamics
cannot in general be described
by matrix exponential solution \eqref{eq:mwf,epr-eq}.
Instead, 
the dynamics (without excitation) arise as a solution to
\begin{align}
	\dela{t} \bmmp\prt = \bmA\prt \bmmp\prt + \bmc\pr,
	\label{eq:mwf,mdot,t-dep}
\end{align}
where
$\bmA\prt \equiv \brac{\bmAxy\prt, \zeros{4 \times 2};
\zeros{2 \times 4}, \bmAz\pr}$
contains now time-dependent compartmental off-resonance effects
$\ompf\prt, \omps\prt$
that appear only in submatrix
$\bmAxy\prt :=$
\begin{align}
	\begin{bmatrix}
		-\frac{1}{\tf{2}\pr}-\rfs\pr & \rsf\pr & \ompf\prt & 0 \\
		\rfs\pr & -\frac{1}{\ts{2}\pr}-\rsf\pr & 0 & \omps\prt \\
		-\ompf\prt & 0 & -\frac{1}{\tf{2}\pr}-\rfs\pr & \rsf\pr \\
		0 & -\omps\prt & \rfs\pr & -\frac{1}{\ts{2}\pr}-\rsf\pr
	\end{bmatrix}.
	\label{eq:mwf,Axy-t-dep}
\end{align}
If $\bmA\prt$ is invertible,
an exponential series solution
to \eqref{eq:mwf,mdot,t-dep} exists
and could be expressed exactly
using the Magnus expansion \cite{magnus:54:ote}. 
Here, 
we utilize a first-order Magnus expansion 
that in chemical equilibrium yields
a simple approximate 
\footnote{
Approximation \eqref{eq:mwf,epr-eq,t-dep} is exact 
if commutation relation
$$\bmA\paren{\bmr,t_1}\bmA\paren{\bmr,t_2} = \bmA\paren{\bmr,t_2} \bmA\paren{\bmr,t_1}$$
holds for all $t_1,t_2 \geq t_0$,
which in turn holds pointwise if and only if
\begin{align}
	\rfs\pr \paren{\ompf\paren{\bmr,t_1} - \omps\paren{\bmr,t_1} -
		\paren{\ompf\paren{\bmr,t_2} - \omps\paren{\bmr,t_2}}} &= 0;
		\label{eq:mwf,epr,appx-conf-rfs} \\
	\rsf\pr \paren{\ompf\paren{\bmr,t_1} - \omps\paren{\bmr,t_1} -
		\paren{\ompf\paren{\bmr,t_2} - \omps\paren{\bmr,t_2}}} &= 0.
		\label{eq:mwf,epr,appx-conf-rsf}
\end{align}
Conditions~\eqref{eq:mwf,epr,appx-conf-rfs}-\eqref{eq:mwf,epr,appx-conf-rsf} hold exactly
in the special cases 
of equal field inhomogeneity across compartments
(\ie, $\ompf = \omps$);
no exchange 
(\ie, $\rfs \gets 0$ and $\rsf \gets 0$); 
or time-independent off-resonance 
(\ie, $\ompf\prt \equiv \ompf\pr$ and $\omps\prt \equiv \omps\pr\, \forall \bmr,t$).
}
solution for all $t \geq t_0$:
\begin{align}
	\bmmp\prt \approx
		e^{\int_{t_0}^t \bmA\paren{\bmr,t'} \der{t'}} \bmmp\paren{\bmr,t_0} +
		\paren{e^{\int_{t_0}^t \bmA\paren{\bmr,t'} \der{t'}} - \eye{6}} \bmmzero\pr.
	\label{eq:mwf,epr-eq,t-dep}
\end{align}
Higher-order expansions
would better capture compartmental magnetization interactions
due to off-resonance effects
and are not considered hereafter 
for simplicity.

We next use \eqref{eq:mwf,epr-eq,t-dep}
to proceed in developing a two-compartment DESS signal model.
As before,
let $\bmmp\prtz$ denote the two-compartment magnetization 
at an initial time $t_0$
selected well into the steady-state
and just prior to RF excitation.
The DESS sequence first applies a fixed RF excitation,
which we again assume
is of sufficiently short duration $\TP$ 
as to permit neglect 
of within-pulse relaxation, off-resonance, and exchange effects.
Further assuming
that each compartment experiences excitation
with equal transmit field sensitivity,
we model RF excitation as a simple rotation operation
by angle $\flip\pr$:
\begin{align}
	\bmmp\paren{\bmr,t_0+\TP} = 
		\bmRxpsix{\flip\pr} \bmmp\prtz.
		\label{eq:mwf,dess-ex}
\end{align}
The transverse components
of $\bmmp\paren{\bmr,t_0+\TP}$
contribute to a first acquired signal;
dephase (but do not spoil completely)
due to gradient dephasing,
and contribute again 
to a second (smaller, but nonzero) acquired signal.
We assume that the dephasing gradient 
is of sufficiently small gradient area
so as to contribute
mainly to compartmental off-resonance phase accrual 
and
negligibly to self-diffusion 
within each compartment
or inter-diffusion between compartments.
Then, compartmental magnetization evolution
during data acquisition 
and gradient dephasing
and until repetition time $\TR$
is reasonably described 
by \eqref{eq:mwf,epr-eq,t-dep}:
\begin{align}
	\bmmp\paren{\bmr,t_0+\TR} \approx
		e^{\int_{t_0+\TP}^{\TR} \bmA\paren{\bmr,t'} \der{t'}} \bmmp\paren{\bmr,t_0+\TP} +
		\paren{e^{\int_{t_0+\TP}^{\TR} \bmA\paren{\bmr,t'} \der{t'}} - \eye{6}} \bmmzero\pr.
	\label{eq:mwf,dess-epr}
\end{align}
Conveniently,
approximation \eqref{eq:mwf,dess-epr}
depends on compartmental off-resonance effects
entirely through compartmental phase functions
$\phipf\pr := \int_{t_0+\TP}^{\TR} \ompf\paren{\bmr,t'} \der t'$
and
$\phips\pr := \int_{t_0+\TP}^{\TR} \omps\paren{\bmr,t'} \der t'$,
which will later aid across-voxel integration. 

In steady state,
one cycle of excitation,
first acquisition,
gradient spoiling,
second acquisition,
and partial recovery
returns the compartmental magnetization
back to its initial state.
We enforce this through 
the usual steady-state condition
\begin{align}
	\bmmp\prtz = \bmmp\paren{\bmr,t_0+\TR}
	\label{eq:mwf,dess-ss}
\end{align}
which yields an algebraic system of equations.
The solution
(if it exists)
gives the steady-state compartmental magnetization
just prior to RF excitation:
\begin{align}
	\bmmp\prtz = 
		\inv{\eye{6} - e^{\int_{t_0+\TP}^{\TR} \bmA\paren{\bmr,t'} \der{t'}} \bmRxpsix{\flip\pr}}
		\paren{e^{\int_{t_0+\TP}^{\TR} \bmA\paren{\bmr,t'} \der{t'}} - \eye{6}} \bmmzero\pr.
	\label{eq:mwf,dess-m-t0}
\end{align}
Substituting \eqref{eq:mwf,dess-m-t0} into \eqref{eq:mwf,dess-ex}
would yield a similar expression
for the steady-state compartmental magnetization
immediately following RF excitation.
Equivalently,
one may substitute \eqref{eq:mwf,dess-ss} into \eqref{eq:mwf,dess-ex}
before solving for $\bmmp\paren{\bmr,t_0+\TP}$ directly,
which gives
\begin{align}
	\bmmp\paren{\bmr,t_0+\TP} &=
		\inv{\eye{6} - \bmRxpsix{\flip\pr} e^{\int_{t_0+\TP}^{\TR} \bmA\paren{\bmr,t'} \der{t'}}}
		\nonumber \\
	&\times \bmRxpsix{\flip\pr} 
		\paren{e^{\int_{t_0+\TP}^{\TR} \bmA\paren{\bmr,t'} \der{t'}} - \eye{6}} \bmmzero\pr.
		\label{eq:mwf,dess-m-tp}
\end{align}
Unlike the SPGR two-compartment magnetization 
\eqref{eq:mwf,spgr-m-t0},
analogous DESS expressions 
\eqref{eq:mwf,dess-m-t0}-\eqref{eq:mwf,dess-m-tp} depend strongly
on transverse submatrix $\int_{t_0}^{\TP} \bmAxy\paren{\bmr,t'} \der{t'}$.
Consequently,
it remains challenging (even with computer solvers)
to expand corresponding matrix exponentials
and thereby find explicit representations 
of \eqref{eq:mwf,dess-m-t0}-\eqref{eq:mwf,dess-m-tp}
in general.

Frequently,
the DESS signals are acquired
at symmetric echo times $\TE$ before and after
the center of each RF pulse.
Since no gradient dephasing occurs
during intervals
between refocusing and defocusing echoes
that contain excitations,
we can reasonably assume
linear off-resonance phase accrual 
over these intervals
and therefore use \eqref{eq:mwf,epr-eq} 
(instead of \eqref{eq:mwf,epr-eq,t-dep})
to evolve the magnetization accordingly.
Substituting \eqref{eq:mwf,dess-m-tp} 
into \eqref{eq:mwf,epr-eq}
gives the magnetization
at the data acquisition time
after RF excitation:
\begin{align}
	\bmmp\paren{\bmr,t_0+\frac{\TP}{2}+\TE} =
		e^{\paren{\TE-\frac{\TP}{2}}\bmA\pr} \bmmp\paren{\bmr,t_0+\TP} 
		+ \paren{\eye{6} - e^{\paren{\TE-\frac{\TP}{2}}\bmA\pr}} \bmmzero\pr.
	\label{eq:mwf,dess-m-te-def}
\end{align}
To compute the magnetization
at the acquisition time 
before excitation,
we consider the free precession, relaxation, and exchange
that occurs between signal reception and excitation:
\begin{align}
	\bmmp\prtz =
		e^{\paren{\TE-\frac{\TP}{2}}\bmA\pr} \bmmp\paren{\bmr,t_0+\frac{\TP}{2}-\TE} 
		+ \paren{\eye{6} - e^{\paren{\TE-\frac{\TP}{2}}\bmA\pr}} \bmmzero\pr.
	\label{eq:mwf,dess-m-te-ref}
\end{align}
Inserting \eqref{eq:mwf,dess-m-t0}
into \eqref{eq:mwf,dess-m-te-ref}
and rearranging 
gives an expression 
for $\bmmp\paren{\bmr,t_0+\frac{\TP}{2}-\TE}$.

The received signal is approximately proportional
to the integrated transverse magnetization
arising from both compartments.
To derive expressions,
we take assumptions very similar 
to those used in Subsection~\ref{ss,mwf,model,spgr}
and append additional distributional assumptions
on the compartmental phase accrual functions 
$\phipf\pr$ and $\phips\pr$:
\begin{enumerate}
	\item We assume that the signal is localized
		to a scale over which there is within-voxel variation
		of compartmental off-resonance effects, 
		but minimal intra-voxel variation 
		of other space-varying parameters
		$\mzero, \ff, \fs, \tf{1}, \ts{1}, \tf{2}, \ts{2}, \rfs, \rsf$.
		This assumption effectively fixes all compartmental properties
		other than $\phipf$, $\phips$, $\ompf$, and $\omps$
		over the volume $\setV$ 
		of a sufficiently small voxel.
		\label{item:dess,int} 
		
	\item
		We assume 
		that the dephasing gradient imparts 
		a sufficiently large integral number $\cyc$ 
		of across-voxel phase cycles
		such that full-repetition compartmental phase accruals
		$\phipf$ and $\phips$ 
		are distributed essentially uniformly
		as $\dist{\phipf} \gets \unif{0,2\pi\cyc}$ 
		and $\dist{\phips} \gets \unif{0,2\pi\cyc}$, 
		where $\cyc \in \set{1,2,3,\dots}$.
		\label{item:dess,ph}
		
	\item 
		We again assume
		that compartmental off-resonance phase accrues linearly
		between each excitation 
		and its adjacent data acquisition periods,
		and that off-resonance frequencies
		are independently distributed 
		within a localized voxel 
		with marginal distributions
		$\dist{\ompf} := \operatorname{Cauchy}\paren{\ompfmed,\Rtpf}$
		and $\dist{\omps} := \operatorname{Cauchy}\paren{\ompsmed,\Rtps}$,
		where $\ompfmed,\ompsmed$ are median off-resonance frequencies
		and $\Rtpf,\Rtps$ are broadening bandwidths.
		\label{item:dess,freq}
		
	\item 
		We assume 
		for short echo times $\TE$
		that negligible exchange occurs
		between each excitation 
		and its adjacent data acquisition periods.
		This assumption facilitates expansion 
		of the matrix exponentials
		explicitly visible 
		in \eqref{eq:mwf,dess-m-te-def}-\eqref{eq:mwf,dess-m-te-ref}
		and separates off-resonance frequency broadening integrals
		by compartment.
		\label{item:dess,exchg0}
\end{enumerate}
With these assumptions,
the noiseless two-compartment steady-state DESS signal models
for a voxel centered
at position $\bmr$
and with volume defined
by $\setV\pr$ are (to within constants):
\begin{align}
	\dess\paren{\bmr,t_0+\frac{\TP}{2}+\TE} 
		&\propto \int_{\setV\paren{\bmr}}
			\brac{1, 1, i, i, 0, 0} \bmmp\paren{\bmr,t_0+\frac{\TP}{2}+\TE} \dercubed{\bmr}
			\label{eq:mwf,dess-def,int} \\
		&\approx \paren{\int_{\reals{2}} 
			\brac{1, 1, i, i, 0, 0} e^{\paren{\TE-\frac{\TP}{2}}\bmA\pr}
				\dist{\ompf}\paren{\ompf} \dist{\omps}\paren{\omps} 
				\der{\ompf} \der{\omps}}
				\nonumber \\
		&\times \int_{\reals{2}} \bmmp\paren{\bmr,t_0+\TP}
			\dist{\phipf} \paren{\phipf} \dist{\phips} \paren{\phips} 
			\der{\phips} \der{\phipf}
			\label{eq:mwf,dess-def,int-freq} \\
		&= 
			\begin{bmatrix}
				e^{-\paren{1/\tf{2}\pr+\Rtpf\pr+i\ompfmed\pr}\paren{\TE-\frac{\TP}{2}}} \\
				e^{-\paren{1/\ts{2}\pr+\Rtps\pr+i\ompsmed\pr}\paren{\TE-\frac{\TP}{2}}} \\
				ie^{-\paren{1/\tf{2}\pr+\Rtpf\pr+i\ompfmed\pr}\paren{\TE-\frac{\TP}{2}}} \\
				ie^{-\paren{1/\ts{2}\pr+\Rtps\pr+i\ompsmed\pr}\paren{\TE-\frac{\TP}{2}}} \\
				0 \\
				0
			\end{bmatrix}\tpose 
			\nonumber \\
		&\times \int_{\reals{2}} \bmmp\paren{\bmr,t_0+\TP}
			\dist{\phipf} \paren{\phipf} \dist{\phips} \paren{\phips} 
			\der{\phips} \der{\phipf};
			\label{eq:mwf,dess-def,int-exchg0} \\
	\dess\paren{\bmr,t_0+\frac{\TP}{2}-\TE}
		&\propto \int_{\setV\paren{\bmr}}
			\brac{1, 1, i, i, 0, 0} \bmmp\paren{\bmr,t_0+\frac{\TP}{2}-\TE} \dercubed{\bmr}
			\label{eq:mwf,dess-ref,int} \\
		&\approx \paren{\int_{\reals{2}} 
			\brac{1, 1, i, i, 0, 0} e^{-\paren{\TE-\frac{\TP}{2}}\bmA\pr}
				\dist{\ompf}\paren{\ompf} \dist{\omps}\paren{\omps} 
				\der{\ompf} \der{\omps}}
				\nonumber \\
		&\times \int_{\reals{2}} \bmmp\paren{\bmr,t_0}
			\dist{\phipf} \paren{\phipf} \dist{\phips} \paren{\phips} 
			\der{\phips} \der{\phipf}
			\label{eq:mwf,dess-ref,int-freq} \\
		&= 
			\begin{bmatrix}
				e^{+\paren{1/\tf{2}\pr-\Rtpf\pr+i\ompfmed\pr}\paren{\TE-\frac{\TP}{2}}} \\
				e^{+\paren{1/\ts{2}\pr-\Rtps\pr+i\ompsmed\pr}\paren{\TE-\frac{\TP}{2}}} \\
				ie^{+\paren{1/\tf{2}\pr-\Rtpf\pr+i\ompfmed\pr}\paren{\TE-\frac{\TP}{2}}} \\
				ie^{+\paren{1/\ts{2}\pr-\Rtps\pr+i\ompsmed\pr}\paren{\TE-\frac{\TP}{2}}} \\
				0 \\
				0
			\end{bmatrix}\tpose 
			\nonumber \\
		&\times \int_{\reals{2}} \bmmp\paren{\bmr,t_0}
			\dist{\phipf} \paren{\phipf} \dist{\phips} \paren{\phips} 
			\der{\phips} \der{\phipf}.
			\label{eq:mwf,dess-ref,int-exchg0}
\end{align}
Observe that
\eqref{eq:mwf,dess-def,int-freq} and \eqref{eq:mwf,dess-ref,int-freq}
use Assumption~\ref{item:dess,int}
to separate the phase and frequency broadening integrals
respectively defined 
in Assumptions~\ref{item:dess,ph} and \ref{item:dess,freq}.
Similar to SPGR calculations 
in Subsection~\ref{ss,mwf,model,spgr},
\eqref{eq:mwf,dess-def,int-exchg0} and \eqref{eq:mwf,dess-ref,int-exchg0}
use Assumption~\ref{item:dess,exchg0}
to evaluate the frequency broadening integral compartment-wise.
However,
the phase broadening integrals
in \eqref{eq:mwf,dess-def,int-exchg0} and \eqref{eq:mwf,dess-ref,int-exchg0}
do not separate.
Thus,
even with the simple first-order Magnus expansion
taken in \eqref{eq:mwf,epr-eq,t-dep},
we find in DESS 
that magnetization compartments mix 
due not only to exchange effects
but also to differences across compartments
in off-resonance phase accrual.

In the special case
where exchange is altogether neglected
(which is a stronger assumption
than Assumption~\ref{item:dess,exchg0},
especially for longer $\TR$),
the phase broadening integrals 
in \eqref{eq:mwf,dess-def,int-exchg0} and \eqref{eq:mwf,dess-ref,int-exchg0}
separate across voxels
and admit the closed-form expressions
\begin{align}
	\dess\paren{\bmr,t_0+\frac{\TP}{2}+\TE} &\propto
		+i\mzero\pr \tan{\frac{\flip\pr}{2}} 
		\label{eq:mwf,dess-def,mod-exchg0} \\
	&\times 
		\bigg(
			\paren{1-\frac{\etaf\paren{\bmr,\TR-\TP}}{\xif\paren{\bmr,\TR-\TP}}}
			e^{-\paren{1/\tf{2}\pr+\Rtpf\pr+i\ompfmed\pr}\paren{\TE-\frac{\TP}{2}}}
			\nonumber \\
	&+	
			\paren{1-\frac{\etas\paren{\bmr,\TR-\TP}}{\xis\paren{\bmr,\TR-\TP}}}
			e^{-\paren{1/\ts{2}\pr+\Rtps\pr+i\ompsmed\pr}\paren{\TE-\frac{\TP}{2}}}
		\bigg);
		\nonumber \\
	\dess\paren{\bmr,t_0+\frac{\TP}{2}-\TE} &\propto
		-i\mzero\pr \tan{\frac{\flip\pr}{2}}
		\label{eq:mwf,dess-ref,mod-exchg0} \\
	&\times
		\bigg(
			\paren{1-\etaf\paren{\bmr,\TR-\TP}}
			e^{+\paren{1/\tf{2}\pr-\Rtpf\pr+i\ompfmed\pr}\paren{\TE-\frac{\TP}{2}}}
			\nonumber \\
	&+
			\paren{1-\etas\paren{\bmr,\TR-\TP}}
			e^{+\paren{1/\ts{2}\pr-\Rtps\pr+i\ompsmed\pr}\paren{\TE-\frac{\TP}{2}}}
		\bigg),
		\nonumber
\end{align}
where $\etaf$, $\etas$, $\xif$, and $\xis$ 
are intermediate variables defined as
\begin{align}
	\etaf\prt &:=
		\sqrt{\frac{1-\paren{\expa{-t/\tf{2}}}^2}{1-\paren{\expa{-t/\tf{2}}/\xif\prt}^2}}
		\qquad
	\xif\prt &:=
		\frac{1-\expa{-t/\tf{1}}\cos{\flip\pr}}{\expa{-t/\tf{1}}-\cos{\flip\pr}};
		\nonumber \\
	\etas\prt &:=
		\sqrt{\frac{1-\paren{\expa{-t/\ts{2}}}^2}{1-\paren{\expa{-t/\tf{2}}/\xis\prt}^2}}
		\qquad
	\xis\prt &:=
		\frac{1-\expa{-t/\ts{1}}\cos{\flip\pr}}{\expa{-t/\ts{1}}-\cos{\flip\pr}}.
		\nonumber
\end{align}
Eqs.~\eqref{eq:mwf,dess-def,mod-exchg0}-\eqref{eq:mwf,dess-ref,mod-exchg0}
naturally extend single-compartment models
\eqref{eq:dess-def-model} and \eqref{eq:dess-ref-model},
and elucidate the intuitive result
that compartmental signals simply add
if exchange is neglected.

%%%%%%%%%%%%%%%%%%%%%%%%%%%%%%%%%%%%%%%%%%%%%%%%%%%
\section{A Fast Acquisition for Precise MWF Estimation}
\label{s,mwf,acq}
%%%%%%%%%%%%%%%%%%%%%%%%%%%%%%%%%%%%%%%%%%%%%%%%%%%

This section develops a new scan profile
consisting of fast steady-state pulse sequences
for precise MWF estimation.
Subsection~\ref{ss,mwf,acq,design} 
first motivates the need
for more scalable scan design
in MWF imaging 
and then adapts our method 
for QMRI scan design
(introduced for a simpler problem 
in Chapter~\ref{c,scn-dsgn}) appropriately.
Subsection~\ref{ss,mwf,acq,detail}
details how we use scalable scan design
(with the two-compartment signal models developed 
in Section~\ref{s,mwf,model})
to design a fast SPGR/DESS acquisition 
for precise MWF imaging.

%%%%%%%%%%%%%%%%%%%%%%%%%%%%%%%%%%%%%%%%%%%%%%%%%%%
\subsection{SPGR/DESS Acquisition Design}
\label{ss,mwf,acq,design}

Recall from Subsection~\ref{ss,scn-dsgn,crb,sig} 
that the inverse 
of the Fisher information 
$\Fisher{\bmx;\bmnu,\bmP} \in \complexes{L \times L}$ 
lower-bounds the covariance
of unbiased estimates 
of $L$ latent object parameters $\bmx \in \complexes{L}$,
given $K$ known object parameters $\bmnu \in \complexes{K}$ 
and $A$ tunable acquisition parameters 
for each of $D$ datasets 
$\bmP \in \reals{A \times D}$.
As before, we continue to focus 
on minimizing a weighted average 
of the latent parameter variances
and thus study the objective function
\begin{align}
	\costa{\bmx; \bmnu, \bmP} := 
		\trace{\bmW \bmF^{-1}\paren{\bmx;\bmnu,\bmP} \bmW\tpose},
	\label{eq:mwf,cost}
\end{align}
where $\bmW$ is a diagonal weighting matrix
and $\trace{\cdot}$ denotes the matrix trace operation.
For scan design,
we seek to minimize $\cost$
with respect to acquisition parameters $\bmP$.

In Subsection~\ref{ss,scn-dsgn,crb,minmax}, 
we addressed the dependance of $\cost$ 
on space-varying object parameters $\bmx$ and $\bmnu$
through a min-max optimization problem.
The associated ``worst-case'' design criterion 
requires only weak assumptions
on object parameter distributions
but is non-differentiable in $\bmP$.
For the relatively simple application described
in Section~\ref{s,scn-dsgn,opt},
the min-max criterion was studied 
through exhaustive search
and so non-differentiability did not matter.
However,
MWF imaging requires estimation
of several more latent parameters
and thus necessitates scan parameter selection
for a greater number of datasets
and thus over a larger search space.
Since exhaustive search 
over this higher-dimensional search space
is prohibitively expensive computationally,
we study here an alternate design criterion
that is differentiable in $\bmP$ 
and is thus amenable 
to gradient-based local optimization.
Specifically,
we seek an acquisition parameter $\bmP$ 
that minimizes the \emph{expected} weighted average
of latent parameter variances
over a search space $\setP$:
\begin{align}
	\est{\bmP} &\in 
		\set{\argmin{\bmP \in \setP} \expcost\paren{\bmP}}, \where
		\label{eq:mwf,P-hat} \\
	\expcost\paren{\bmP} &:= 
		\expect{\bmx,\bmnu}{\costa{\bmx; \bmnu, \bmP}}
		\label{eq:mwf,expcost}
\end{align}
and $\expect{\bmx,\bmnu}{\cdot}$ denotes joint expectation
with respect to prior joint distribution $\dist{\bmx,\bmnu}$ on $\bmx,\bmnu$.

Unlike min-max cost \eqref{eq:scn-dsgn,cost-tight},
expected cost \eqref{eq:mwf,expcost} is often differentiable in $\bmP$.
We next construct the gradient matrix 
$\grada{\bmP}{\expcost\paren{\bmP}} \in \reals{A \times D}$
and provide sufficient conditions
for when this gradient matrix exists.
Our simple strategy involves 
first constructing $\grada{\bmP}{\costa{\bmx; \bmnu, \bmP}}$
element-wise 
for fixed $\bmx,\bmnu$
and then relating
$\grada{\bmP}{\expcost\paren{\bmP}}$
to $\grada{\bmP}{\costa{\bmx; \bmnu, \bmP}}$.
Let $\dela{p_{a,d}}$ be the $(a,d)$th element
of matrix operator $\grada{\bmP}$.
By standard matrix derivative identities,
we have
\begin{align}
	\dela{p_{a,d}} \costa{\bmx; \bmnu, \bmP}
		&= 
		\dela{p_{a,d}} \trace{\bmW \bmF^{-1}\paren{\bmx;\bmnu,\bmP} \bmW\tpose}
		\nonumber \\
		&= -\trace{\bmW \bmF^{-1}\paren{\bmx;\bmnu,\bmP} 
		\dela{p_{a,d}}\paren{\Fisher{\bmx;\bmnu,\bmP}} 
		\bmF^{-1}\paren{\bmx;\bmnu,\bmP} \bmW\tpose}.
		\label{eq:mwf,cost-der}
\end{align}
For image data corrupted
by additive complex Gaussian noise 
with zero mean 
and covariance $\bmSig$,
the Fisher information
(repeated from \eqref{eq:scn-dsgn,fisher} for clarity)
is given by
\begin{align}
	\Fisher{\bmx; \bmnu, \bmP}
		&=
		\paren{\grada{\bmx} \bms\paren{\bmx; \bmnu, \bmP}}\ctpose
    \bmSig^{-1} \grada{\bmx} \bms\paren{\bmx; \bmnu, \bmP},
   \label{eq:mwf,fisher}
\end{align}
where $\bms := \brac{s_1,\dots,s_D}\tpose$ 
is the signal model.
Furthermore, if datasets are assumed independent
(as is typical),
$\bmSig$ takes diagonal structure
$\bmSig \gets \diag{\sigma_1^2,\dots,\sigma_D^2}$
and
\begin{align}
	\dela{p_{a,d}}\paren{\Fisher{\bmx;\bmnu,\bmP}} 
		&= 
		\dela{p_{a,d}} \sum_{d'=1}^D \frac{1}{\sigma_{d'}^2}
		\paren{\grada{\bmx}{s_{d'}\paren{\bmx; \bmnu, \bmp_{d'}}}}\ctpose
		\grada{\bmx}{s_{d'}\paren{\bmx; \bmnu, \bmp_{d'}}}
		\nonumber \\
		&= 
		\frac{1}{\sigma_d^2} \dela{p_{a,d}} 
		\paren{\paren{\grada{\bmx}{s_d\paren{\bmx; \bmnu, \bmp_d}}}\ctpose
		\grada{\bmx}{s_d\paren{\bmx; \bmnu, \bmp_d}}}.
		\label{eq:mwf,fisher-der}
\end{align}
Substituting \eqref{eq:mwf,fisher}-\eqref{eq:mwf,fisher-der}
into \eqref{eq:mwf,cost-der} 
gives expressions 
in terms of signal model derivatives
for each element 
of $\grada{\bmP}{\costa{\bmx; \bmnu, \bmP}}$.
These expressions are well-defined 
if $\bmF$ is invertible 
and if mixed partial derivative matrices
$\grada{\bmp_1}{\paren{\grada{\bmx}{s_1}}\tpose},
\dots,
\grada{\bmp_D}{\paren{\grada{\bmx}{s_D}}\tpose}
\in \complexes{L \times A}$
exist and are continuous in $\bmx,\bmP$. 
Further assuming 
that $\grada{\bmP}{\costa{\bmx; \bmnu, \bmP}}$ remains bounded 
for all $\bmx,\bmnu$, 
\begin{align}
	\grada{\bmP}{\expcost\paren{\bmx; \bmnu, \bmP}} 
		&= 
		\grada{\bmP}{\expect{\bmx,\bmnu}{\costa{\bmx; \bmnu, \bmP}}}
		\nonumber \\
		&= 
		\expect{\bmx,\bmnu}{\grada{\bmP}{\costa{\bmx; \bmnu, \bmP}}},
		\label{eq:mwf,expcost-Pgrad}
\end{align} 
which provides an expression
for the gradient of the expected cost,
as desired.

If \eqref{eq:mwf,expcost-Pgrad} exists,
\eqref{eq:mwf,P-hat} can be solved iteratively
for a convex search space $\setP$
via updates
\begin{align}
	\iter{\bmP}{i} \gets 
		\proja{\setP}{\iter{\bmP}{i-1}-\grada{\bmP}{\expcost\paren{\iter{\bmP}{i-1}}}},
		\label{eq:mwf,P-iter}
\end{align}
where $\proja{\setP}{\cdot}$ denotes projection onto $\setP$
and $i$ indexes iteration.
Since $\expcost$ is non-convex in $\bmP$ in general,
such iterations achieve
only locally optimal convergence
(further discussed in Subsection~\ref{ss,bkgrd,opt,loc})
and the locally convergent minimizer
depends on initialization.

%%%%%%%%%%%%%%%%%%%%%%%%%%%%%%%%%%%%%%%%%%%%%%%%%%%
\subsection{Scan Design Implementation Details}
\label{ss,mwf,acq,detail}

% why spgr/dess (?)
% neglect exchange, same off resonance, lump r2p/f into m0, etc
% optimize flips and trs
% search space (flip, tr constraints, time)
% prior distributions - estimated v measured
% greedy scan construction
% greedy scan design initialization
% noise - to have realistic coef of var

%%%%%%%%%%%%%%%%%%%%%%%%%%%%%%%%%%%%%%%%%%%%%%%%%%%
\section{Experimentation}
\label{s,mwf,exp}
%%%%%%%%%%%%%%%%%%%%%%%%%%%%%%%%%%%%%%%%%%%%%%%%%%%

% maybe repeat some details from ch 4
% krr details
% preliminary image

%%%%%%%%%%%%%%%%%%%%%%%%%%%%%%%%%%%%%%%%%%%%%%%%%%%
\section{Summary and Future Work}
\label{s,mwf,summ}
%%%%%%%%%%%%%%%%%%%%%%%%%%%%%%%%%%%%%%%%%%%%%%%%%%%

\todo{discuss negative $\ff$ range}
