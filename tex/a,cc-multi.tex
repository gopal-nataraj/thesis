% multiple-data complex coil combination

%%%%%%%%%%%%%%%%%%%%%%%%%%%%%%%%%%%%%%%%%%%%%%%%%%%
\section{Introduction}
\label{s,cc-multi,intro}
%%%%%%%%%%%%%%%%%%%%%%%%%%%%%%%%%%%%%%%%%%%%%%%%%%%

This appendix introduces
an unpublished algorithm
for simultaneously coil-combining 
a collection of MR coil image datasets
without prior knowledge 
of coil sensitivity maps.
The algorithm expects
that MR coil datasets arise
from a single acquisition
with little inter-scan motion
but allows for variable MR contrasts
across datasets.
Such multiple contrast, multiple receive coil acquisitions
arise naturally
in many QMRI applications,
including the ones studied in Chapters~\ref{c,scn-dsgn}-\ref{c,mwf}.
The algorithm extends 
a similar sensitivity-blind method
for coil-combining a single dataset
\cite{ying:07:jir}
to exploit coil sensitivity redundancy
across datasets
and thereby improve problem conditioning
over coil-combining multiple datasets separately.

%%%%%%%%%%%%%%%%%%%%%%%%%%%%%%%%%%%%%%%%%%%%%%%%%%%
\section{Problem Formulation}
\label{s,cc-multi,prob}
%%%%%%%%%%%%%%%%%%%%%%%%%%%%%%%%%%%%%%%%%%%%%%%%%%%

Suppose that we are presented image-domain coil data
from $C$ receive coils,
$D$ contrasts,
and $V$ voxel locations.
A simple model
for each single-coil, single-contrast image reads
\begin{align}
	\bmy_{c,d} = \diag{\bmsrx_{c}} \bmx_d + \bmeps_{c,d},
	\label{eq:cc-multi,model}
\end{align}
where $\bmy_{c,d} \in \complexes{V}$ 
denotes a noisy image dataset
at the $c$th coil and $d$th contrast;
$\bmsrx_{c} \in \complexes{V}$ 
denotes the unknown sensitivity 
of the $c$th receive coil;
$\bmx_d \in \complexes{V}$ 
denotes the unknown magnetization
of the $d$th contrast;
$\bmeps_{c,d} \in \cgauss{\zeros{V}}{\bmSig}$ 
is complex Gaussian noise;
$c \in \set{1,\dots,C}$;
and $d \in \set{1,\dots,D}$.
This model allows for spatial correlations
but for ease of exposition
ignores correlations across coils
and correlations across contrasts.

We seek to estimate $\bmx_1,\dots,\bmx_D$
from $\bmy_{1,1},\dots,\bmy_{C,D}$ 
without knowledge
of $\bmsrx_1,\dots,\bmsrx_C$.
We utilize a cost function
that not only penalizes data inconsistency 
but also encourages spatially smooth sensitivity profiles
through quadratic regularization:
\begin{align}
	\cost\paren{\bmsrx_1,\dots,\bmsrx_C,\bmx_1,\dots,\bmx_D} :=
		\sum_{c=1}^C \paren{%
			\sum_{d=1}^D \norm{\bmy_{c,d}-\diag{\bmsrx_c}\bmx_d}^2_{\bmSig^{-1}}
			+ \beta_c \norm{\bmD\bmsrx_c}^2_2,
		}%
	\label{eq:cc-multi,cost}
\end{align}
where $\bmD$ is a pure-real first-order finite differencing matrix 
and $\beta_c$ is a regularization parameter.
Observe that $\Psi$ is not coercive 
in $\bmx_1,\dots,\bmx_D$:
scaling each $\bmx_1,\dots,\bmx_D$ 
by $\alpha$
while scaling each $\bmsrx_1,\dots,\bmsrx_C$ 
by $\frac{1}{\alpha}$
only reduces $\Psi$
for arbitrarily large scale factor $\alpha$.
To resolve this scale ambiguity,
we seek to minimize \eqref{eq:cc-multi,cost}
under constraints: 
\begin{align}
	\paren{\bmsrxest_1,\dots,\bmsrxest_C,\est{\bmx}_1,\dots,\est{\bmx}_D} 
		&\in \set{%
			\argmin{%
				\substack{%
					\bmsrx_1,\dots,\bmsrx_C \in \complexes{V} \\
					\bmx_1,\dots,\bmx_D \in \complexes{V}
				}%
			}%
			\cost\paren{\bmsrx_1,\dots,\bmsrx_C,\bmx_1,\dots,\bmx_D}
		} 
		\nonumber
		\\
	\mathrm{such\,\,that} \qquad
	\ones{V}\tpose\bmx_d 
		&= \const{d} \qquad \forall d \in \set{1,\dots,D},
	\label{eq:cc-multi,prob}
\end{align}
where $\ones{V}$ is length-$V$ vector of ones
and $c_d \in \complex$ is a fixed constant.	

%%%%%%%%%%%%%%%%%%%%%%%%%%%%%%%%%%%%%%%%%%%%%%%%%%%
\section{Alternating Minimization Algorithm}
\label{s,cc-multi,alg}
%%%%%%%%%%%%%%%%%%%%%%%%%%%%%%%%%%%%%%%%%%%%%%%%%%%

We solve \eqref{eq:cc-multi,prob}
by alternating between
updating coil sensitivity variables 
while holding contrast variables fixed
and
updating contrast variables
while holding coil sensitivity variables fixed.
We next describe each of these updates in turn.

Because we assumed for simplicity
that noise is uncorrelated across receive coils,
each of the $C$ coil updates
can be updated in parallel.
We update the $c$th coil variable
by minimizing a quadratic cost function:
\begin{align}
	\iter{\bmsrx_c}{i+1} \gets
		\argmin{\bmsrx_c\in\complexes{V}}
		\sum_{d=1}^D \norm{\bmy_{c,d}-\diag{\iter{\bmx_d}{i}}\bmsrx_c}^2_{\bmSig^{-1}}
		+ \beta_c \norm{\bmD\bmsrx_c}^2_2.
	\label{eq:cc-multi,srx-prob}
\end{align}
where $\iter{\paren{\cdot}}{i}$ 
denotes the $i$th iterate.
For moderate $V$,
\eqref{eq:cc-multi,srx-prob} can be solved directly:
\begin{align}
	\iter{\bmsrx_c}{i+1} \gets \pinv{%
		\diag{\iter{\bmx_d}{i}}\ctpose \bmSig^{-1} \diag{\iter{\bmx_d}{i}}
		+ \beta_c \bmD\ctpose \bmD
	}%
	\diag{\iter{\bmx_d}{i}}\ctpose \bmSig^{-1} \bmy_{c,d}.
	\label{eq:cc-multi,srx-update}
\end{align}
For larger $V$ with non-diagonal $\bmSig$,
more sophisticated inner updates may be necessary.

Because we assumed 
that noise is also uncorrelated across contrasts,
each of the $D$ contrast variables
can be updated in parallel.
We update the $d$th contrast variable
at the $\paren{i+1}$th iteration
by solving a constrained problem:
\begin{align}
	\iter{\bmx}{i+1}_d &\gets
		\argmin{\bmx_d\in\complexes{V}}
		\sum_{c=1}^C \norm{\bmy_{c,d}-\diag{\iter{\bmsrx_c}{i}}\bmx_d}^2_{\bmSig^{-1}} \,\,\,
		\mathrm{such\,\,that} \,\,\,
		\ones{V}\tpose\bmx_d = \const{d}.
	\label{eq:cc-multi,x-prob}
\end{align}
We solve \eqref{eq:cc-multi,x-prob}
using the method of Lagrange multipliers.
The Lagrangian is
\begin{align}
	\Lambda\paren{\bmx_d,\lambda_d} :=
		\sum_{c=1}^C \norm{\bmy_{c,d}-\diag{\iter{\bmsrx_c}{i}}\bmx_d}^2_{\bmSig^{-1}}
		+ 4 \re{\bar{\lambda}_d \paren{\ones{V}\tpose\bmx_d-\const{d}}},
	\label{eq:cc-multi,x-lagrangian}
\end{align}
where $\lambda_d\in\complex$ is a Lagrange multiplier;
$\re{\cdot}$ extracts the real component of its argument;
and $\bar{\paren{\cdot}}$ denotes complex conjugate.
Setting $\grada{\bmx_d}{\Lambda\paren{\bmx_d,\lambda_d}} = \zeros{V}$
yields the update
\begin{align}
	\iter{\bmx}{i+1}_d \gets
		\pinv{%
			\sum_{c=1}^C \diag{\iter{\bmsrx_c}{i}}\ctpose 
				\bmSig^{-1} \diag{\iter{\bmsrx_c}{i}}
		}%
		\paren{%
			\sum_{c=1}^C \diag{\iter{\bmsrx_c}{i}}\ctpose \bmSig^{-1} \bmy_{c,d}
			- \lambda_d \ones{V}
		}%
	\label{eq:cc-multi,x-update}
\end{align}
where $\pinv{\cdot}$ denotes pseudoinverse
and $\lambda_d$ is at each iteration updated
such that the constraint remains satisfied: 
\begin{align}
	\iter{\lambda}{i+1}_d \gets \frac{%
		\ones{V}\tpose \pinv{%
			\sum_{c=1}^C \diag{\iter{\bmsrx_c}{i}}\ctpose \bmSig^{-1} \diag{\iter{\bmsrx_c}{i}}
		}%
		\paren{%
			\sum_{c=1}^C \diag{\iter{\bmsrx_c}{i}}\ctpose \bmSig^{-1} \bmy_{c,d}
		}%
		- \const{d}
	}{%
		\ones{V}\tpose \pinv{%
			\sum_{c=1}^C \diag{\iter{\bmsrx_c}{i}}\ctpose 
				\bmSig^{-1} \diag{\iter{\bmsrx_c}{i}}
		}%
		\ones{V}
	}.
	\label{eq:cc-multi,x-lam}
\end{align}
Observe that matrix divisions 
required to implement \eqref{eq:cc-multi,x-lam}
need not depend on contrast index $d$.
Thus $D$ contrast updates 
require $D+1$ matrix division subproblems per iteration,
of which $D$ divisions are amenable to parallelization.

Since \eqref{eq:cc-multi,prob} is non-convex,
solutions will depend on initialization.
We initialize each contrast separately
using the square root of the sum 
over that contrast's magnitude coil images squared,
a popular conventional estimator:
\begin{align}
	\iter{\bmx}{0}_d \gets
		\diag{\sqrt{\sum_{c=1}^C \abs{\bmy_{c,d}}^2}}
		e^{j \angle{\bmy_{1,d}}},
	\label{eq:cc-multi,x-init}
\end{align}
where $\sqrt{\cdot}$, $\abs{\cdot}$, and $\angle{\paren{\cdot}}$
respectively denote 
element-wise square root, absolute value, and phase angle operators.
Here, assigning phase to $\iter{\bmx_d}{0}$
using the phase of the first coil's data
causes the phase 
of coil sensitivity estimates
to be relative to the first coil's phase.
