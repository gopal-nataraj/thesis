% kernel ridge regression for parameter estimation

\section{Introduction}
\label{s,krr,intro}

In MRI \emph{parameter estimation},
one seeks to quantify biomarker ''maps''
(\ie, parameter images) 
from data.
Because MRI acquisitions 
are tunably sensitive
to many physical processes
(\eg, relaxation \cite{bloch:1946:ni-paper}, 
diffusion \cite{torrey:56:bew}, 
and chemical exchange \cite{mcconnell:58:rrb}),
MRI parameter estimation is important
in many QMRI applications
(\eg, relaxometry \cite{bloembergen:1948:rei}, 
diffusivity tensor imaging \cite{bihan:01:dti}, 
and multi-compartmental imaging \cite{mackay:94:ivv}). 
Motivated by such widespread applications,
this chapter describes a method
for fast MRI parameter estimation.

% signal models nonlinear
% so parameter estimation requires nonconvex optimization
% previous chapter described likelihood models
% several works [cite] have had success with this
% however these work for simple problems like single t1/t2 estimation
% for larger problems, undesirable or even intractable
Chapter~\ref{c,relax} applied
a common parameter estimation strategy
to QMRI
that involves minimization
of an objective function
related to the likelihood function
(and possibly regularization terms).
Because MR signal models are typically nonlinear functions
of latent object parameters,
such likelihood-based estimation
requires non-convex optimization in general.
To seek global optima,
several of our 
\cite{nataraj:14:rje,nataraj:14:mbe,nataraj::oms}
and others' 
\cite{staroswiecki:12:seo,ma:13:mrf,beneliezer:15:raa,zhao:16:mlr}
works approach estimation 
via exhaustive grid search,
which requires either storing
or computing on-the-fly a ``dictionary'' 
of scan profile signal vectors.
These works estimate a small (2-3)
number of latent parameters,
and so grid search is practical.
However, 
for even moderately sized problems,
the required number 
of dictionary elements
renders grid search undesirable or even intractable,
unless one assumes artificially restrictive latent parameter constraints.

% clear need for a method that scales well with # parameters?
% multi-compartment: 6-11
% diffusion: at least 7
% phase-based methods: flow, b1, b0, asl? 
There are numerous QMRI applications
that could benefit from an alternative MRI parameter estimation method
that scales well with the number of latent parameters.
For example,
vector (\eg, flow \cite{feinberg:85:mri})
and tensor 
(\eg, diffusivity \cite{bihan:01:dti} or conductivity \cite{tuch:01:ctm})
field mapping techniques
require estimation 
of at minimum 4 and 7 latent parameters per voxel,
respectively.
Phase-based longitudinal \cite{sekihara:85:nif} 
or transverse \cite{morrell:08:aps,sacolick:10:bmb} field mapping
could avoid noise-amplifying algebraic manipulations
on reconstructed image data
that are conventionally used
to reduce signal dependencies 
on nuisance latent parameters.
Compartmental fraction mapping \cite{mackay:94:ivv}
from steady-state pulse sequences
requires estimation of at least 7 \cite{deoni:08:gmt}
and as many as 10 \cite{deoni:13:oct}
latent parameters per voxel.
In these and other applications,
greater estimation accuracy
requires more complete signal models
that involve more latent parameters,
which only increases the need 
for scalable estimation methods.

% omit
% what has been done for larger problems?
% several works tried sim annealing, global opt, region contraction
% however, slow and have been shown to yield constraint-dependent estimates

% kernel methods
% fundamental challenge: nonlinear model
% kernel methods used to allow nonlinearity in classification, novelty detection, regression
% can relate estimation (where
The fundamental challenge 
of scalable MRI parameter estimation
stems from MR signal model nonlinearity:
standard linear estimators
would be scalable but inaccurate.
Opportunely,
so-called \emph{kernel functions} \cite{berg:84:hao}
are often well-suited 
to transform nonlinear problems
into linear (albeit higher-dimensional) problems
that can be solved efficiently. 
Associated \emph{kernel methods} \cite{scholkopf:01:agr}
are quite general
and yield remarkably simple nonlinear extensions
to otherwise linear methods
for various machine learning problems,
\eg classification,
anomaly detection,
dimensionality reduction,
and regression.

% idea here is to link parameter estimation 
% to a problem of regression
% ideas of machine learning can link estimation (seek parameter estimates from data assuming model) to regression (seek regression function relating 
This chapter introduces a fast, scalable method 
for nonlinear MRI parameter estimation
via kernel ridge regression (KRR).
We observe that 
for voxel-wise separable MRI parameter estimation problems,
one can rapidly simulate many instances
of latent parameter inputs and signal outputs
from the nonlinear signal model.
We take such input-output pairs
as simulated \emph{training points}
and propose to then \emph{learn}
(using an appropriate kernel function)
a non-iterative estimator
(\ie, a nonlinear regression function)
from the training points.
The proposed KRR-based estimator scales considerably better
with the number of estimated latent parameters
than previously-discussed likelihood-based estimators.

This chapter is organized as follows.
Section~\ref{s,krr,meth}
reviews the general signal model
for an MR scan profile,
constructs and solves 
an appropriate functional optimization problem,
and further reduces the proposed estimator's
computational requirements
through a kernel approximation.
Section~\ref{s,krr,exp}
applies KRR-based estimation
to quantify six parameters arising 
from models describing the steady-state 
magnetization dynamics
of two water compartments,
a challenging application
of interest in myelin water fraction imaging 
(discussed in Chapter~\ref{c,mwf}).
Section~\ref{s,krr,summ}
discusses possible extensions 
and provides concluding remarks.
