% steady-state rf pulse design

%%%%%%%%%%%%%%%%%%%%%%%%%%%%%%%%%%%%%%%%%%%%%%%%%%%
\section{Introduction}
\label{s,ss-rf,intro}
%%%%%%%%%%%%%%%%%%%%%%%%%%%%%%%%%%%%%%%%%%%%%%%%%%%

Modern radiofrequency (RF) pulse design methods 
often relate a desired magnetization pattern 
to the underlying RF pulse and excitation gradients. 
Conventional techniques assume
negligible residual transverse magnetization 
immediately prior to excitation 
to show that under the small-excitation approximation, 
this relation is linear with respect to the RF pulse \cite{pauly:89:aks}. 
For fixed gradients, 
this facilitates rapid pulse design algorithms 
using linear filter design principles \cite{shinnar:89:tuo, pauly:91:prf} 
or iterative algorithms \cite{yip:05:irp, setsompop:08:mls}. 
In practice, 
pulse designers can realize 
near-complete transverse relaxation 
prior to excitation with long repetition times, 
large unbalanced spoiler gradients, 
or tip-up pulses \cite{nielsen:13:stf}. 
However, all of these methods require additional scan time 
beyond excitation and data acquisition. 

This appendix investigates 
the small-excitation RF pulse design problem 
when the usual assumption of near-complete decay 
of residual transverse magnetization is not taken.
Omitting this assumption
reveals the influence of RF pulses
on the steady-state (SS) transverse magnetization
rather than on the conventionally modeled 
single-repetition transverse magnetization.
Because the former is more directly related
to the received signal,
such SS-informed RF pulse design
might allow for more accurate excitation patterns.

%%%%%%%%%%%%%%%%%%%%%%%%%%%%%%%%%%%%%%%%%%%%%%%%%%%
\section{Signal Model}
\label{s,ss-rf,model}
%%%%%%%%%%%%%%%%%%%%%%%%%%%%%%%%%%%%%%%%%%%%%%%%%%%

We begin with the Bloch equations
in a non-inertial reference frame 
rotating clockwise about the $z$-axis 
at the Larmor frequency:
\begin{align}
	\dela{t} \mxyp\prt &= i\gam\paren{\mzp\prt \bxyp\prt - \mxyp\prt \bzp\prt} -
		\frac{\mxyp\prt}{\Tt\pr} ;
		\label{eq:ss-rf,bloch-mxyp} \\
	\dela{t} \mzp\prt &= \gam\paren{\mxp\prt \byp\prt - \myp\prt \bxp\prt} - 
		\frac{\mzp\prt - \mzero\pr}{\To\pr}.
		\label{eq:ss-rf,bloch-mzp}
\end{align}
Here,
$\mxyp\prt \equiv \mxp\prt + i\myp\prt$ 
and 
$\mzp\prt$ 
are respectively the rotating-frame transverse and longitudinal magnetization
at position $\bmr \in \reals{3}$ and time $t\geq0$;
$\bxyp\prt \equiv \bxp\prt + i\byp\prt$
and 
$\bzp\prt$
are respectively the apparent transverse and apparent longitudinal magnetic field;
$\To\pr$ and $\Tt\pr$ 
are spin-lattice and spin-spin relaxation time constants;
$\mzero\pr$ is the equilibrium magnetization;
$\gam$ is the gyromagnetic ratio;
and $i:=\sqrt{-1}$.
These coupled differential equations
are challenging to solve outright.
To proceed, 
we take two simplifying assumptions:
\begin{itemize}
	\item{%
		We assume that RF pulses 
		do not strongly perturb the longitudinal magnetization
		from its initial state 
		at time $t \gets t_0$,
		\ie $\mzp\prt\bxyp\prt \approx \mzp\prtz\bxyp\prt$
		for all $t~\in~\brac{t_0,t_0+\TP}$
		where $\TP \geq 0$ denotes RF pulse duration.
		This small-excitation assumption
		differs from the conventional small-excitation assumption \cite{pauly:89:aks}
		(that instead approximates 
		$\mzp\prt \approx \mzero\pr$ 
		for all $t~\in~\brac{t_0,t_0+\TP}$)
		in that it allows 
		for memory of the longitudinal magnetization
		prior to excitation.
	}%
	\item{%
		We assume that the apparent transverse magnetic field 
		separates in position and time,
		\ie $\bxyp\prt \approx \stx\pr\bone\pt$
		where $\stx\pr \in \complex$ 
		is the RF coil spatial variation
		and $\bone\pt \equiv \bonex\pt + i\boney\pt$ 
		is the RF excitation envelope.
	}%
\end{itemize}
With these assumptions, 
the rotating-frame Bloch equations read
\begin{align}
	\dela{t}{\mxyp}\prt &= 
		i\gam \paren{\mzp\prtz\stx\pr\bone\pt - \mxyp\prt \bzp\prt} 
			- \frac{\mxyp\prt}{\Tt\pr}; 
		\label{eq:ss-rf,mxyp} \\
	\dela{t}{\mzp}\prt &= 
		\gam \paren{\re{\bar{\stx}\pr\mxyp\prt}\boney\pt-\im{\bar{\stx}\pr\mxyp\prt} \bonex\pt} 
			- \frac{\mzp\prt-\mzero\pr}{\To\pr}, 
		\label{eq:ss-rf,mzp}
\end{align} 
where $\re{\cdot}$ and $\im{\cdot}$ 
respectively extract real and imaginary components
and $\bar{\paren{\cdot}}$ denotes complex conjugation.
Expressing \eqref{eq:ss-rf,mxyp}-\eqref{eq:ss-rf,mzp}
in terms of $\mxytp\prt \equiv \mxtp\prt + i\mytp\prt := \bar{\stx}\pr\mxyp\prt$
allows further simplification:
\begin{align}
	\dela{t} \mxytp\prt &= 
		i\gam \paren{\mzp\prtz \abs{\stx\pr}^2 \bone\pt - \mxytp\prt \bzp\prt}
			- \frac{\mxytp\prt}{\Tt\pr} ; 
		\label{eq:ss-rf,mxytp} \\
	\dela{t}{m}'_{z}\prt &= 
	\gam \paren{\mxtp\prt \boney\pt - \mytp\prt \bonex\pt} 
			- \frac{\mzp\prt - \mzero\pr}{\To\pr}. 
		\label{eq:ss-rf,mztp}
\end{align}
The small-excitation assumption above 
decouples \eqref{eq:ss-rf,mxytp} 
from \eqref{eq:ss-rf,mztp}.
Thus we can solve directly
for $\mxytp\prt$
and then obtain an expression for $\mzp\prt$
via substitution into \eqref{eq:ss-rf,mztp}.
We solve \eqref{eq:ss-rf,mxytp}
using the method of integrating factors.
The solution for times $t \geq t_0$ is
\begin{align}
		\mxytp\prt &= 
			\mxytp\prtz e^{i \phi\prtzt{t}} e^{-(t-t_0)/\Tt\pr} 
			\nonumber 
			\\
		&+ i \gam \mzp\prtz \abs{\stx\pr}^2 \int_{t_0}^t b_{xy}'(t') 
				e^{i \phi(\bmr, t; t')} e^{-(t-t')/\Tt\pr} \der t',
		\label{eq:ss-rf,mxyp-sol}
\end{align}
where $\phi(\bmr, t; t') := -\gam \int_{t'}^t \bzp(\bmr, \tau) \der \tau$
denotes the phase accumulation 
from $t'$ to $t$ 
due to off-resonance effects.
Note that if one assumes negligible transverse magnetization
from prior excitations
(\ie, $\mxytp\prtz \approx 0$)
and complete longitudinal recovery
(\ie, $\mzp\prtz \approx \mzero\pr$),
one recovers the conventional linear relation
between magnetization and RF field
for small excitations \cite{pauly:89:aks}.
Without these assumptions,
the longitudinal magnetization 
must also be considered.
The solution to \eqref{eq:ss-rf,mztp}
expressed in terms of \eqref{eq:ss-rf,mxyp-sol}
for $t \geq t_0$ is
\begin{align}
	\mzp\prt &= 
		\mzp\prtz e^{-(t-t_0)/\To\pr} + \mzero\pr \paren{1 - e^{-(t-t_0)/\To\pr}}
		\nonumber \\
	&+ 
		\gam \int_{t_0}^t e^{-(t-t')/\To\pr} \paren{\tilde{m}_x'(\bmr, t') b_y'(t') - \tilde{m}_y'(\bmr, t') b_x'(t')} \der t'.
		\label{eq:ss-rf,mzp-sol}
\end{align}

We next impose a steady-state condition 
to solve for the magnetization
at an initial time $t_0$	
selected well into the steady-state.
After many periodic repetition cycles,
the magnetization following one full repetition will equilibrate
under certain mild assumptions \cite{scheffler:99:apd}
to the initial magnetization.
We specifically assume 
that $t_0$ marks the beginning
of a steady-state repetition interval of length $\TR\geq\TP$
during which RF excitation may be nonzero
for $t \in \brac{t_0,t_0+\TP}$ 
but $\bxyp\prt=0$ 
for free precession period $t \in \paren{t_0+\TP,t_0+\TR}$.
Under these assumptions,
the steady-state relations
\begin{align}
	\mxytp(\bmr, t_0+\TR) &= 
		\mxytp\prtz 
		\label{eq:ss-rf,mxyp-ss}
		\\
	\mzp(\bmr, t_0+\TR) &= 
		\mzp\prtz
		\label{eq:ss-rf,mzp-ss}
\end{align} 
provide an algebraic system of equations
for the initial magnetization components.
With some algebra
and after reversion from intermediate variable $\mxytp\prtz$
to apparent transverse magnetization $\mxyp\prtz$,
we find that steady-state initial magnetization is
\begin{align}
	\mxyp\prtz &= 
		\frac{%
			i \gam \mzero\pr \stx\pr (1-\EoR) \int_{t_0}^{t_0+\TP} \bone(t') 
				e^{i \phi(\bmr, t_0+\TR; t')} e^{-(t_0+\TR-t')/\Tt\pr} \der t'
		}{%
			\paren{1-\EtR e^{i \phi(\bmr, t_0 + \TR; t_0)}} \paren{(1-\EoR) + q\pr}
		};%
		\label{eq:ss-rf,mxyp-t0} 
		\\
	\mzp\prtz &= 
		\frac{%
			\mzero\pr \paren{1 - \EoR}
		}{%
			\paren{1-\EoR} + q\pr
		},% 
	\label{eq:ss-rf,mzp-t0}
\end{align}
where $\Eo\prt := e^{-t/\To\pr}$;
$\Et\prt := e^{-t/\Tt\pr}$; 
and 
\begin{alignat}{3}
	&q\pr&
		&= \gam^2 \abs{\stx\pr}^2 
			&&\int_{t_0}^{t_0+\TP} \int_{t_0}^{t'} 
				e^{-(\TR + t_0 - t')/\To\pr} e^{-(t'-\tau)/\Tt\pr} 
				\re{\bone(t') b'^*_{xy}(\tau) e^{-i \phi(\bmr, t'; \tau)}} \der \tau \der t' 
				\nonumber \\
	& &
		&= \frac{\gam^2 \abs{\stx\pr}^2 }{2} 
			&&\int_{t_0}^{t_0+\TP} \int_{t_0}^{t_0+\TP} \bone(t') 
				b'^*_{xy}(\tau) e^{-(\TR + t_0 - \max{(\tau, t'))}/\To\pr} 
				e^{-\abs{t'-\tau}/\Tt\pr} e^{-i \phi(\bmr, t'; \tau)} \der \tau \der t'.
				\label{eq:ss-rf,q}
\end{alignat}
It is intuitive 
to design RF excitations 
that achieve a desired magnetization pattern 
at echo time $\TE\geq\TP$ 
following the start of excitation. 
We obtain the magnetization
at echo time
by substituting \eqref{eq:ss-rf,mxyp-t0}-\eqref{eq:ss-rf,mzp-t0} 
into \eqref{eq:ss-rf,mxyp-sol} 
and evaluating at time $t \gets t_0+\TE$:
\begin{align}
	\mxyp(\bmr, t_0 + \TE) &= 
		\frac{%
			i \gam \mzero\pr \stx\pr (1-\EoR) 
			\int_{t_0}^{t_0+\TP} \bone(t') e^{i \phi(\bmr, t_0+\TE; t')} 
			e^{-(t_0+\TE-t')/\Tt\pr} \der t'
			}{%
				\paren{1-\EtR e^{i \phi(\bmr, t_0 + \TR; t_0)}} \paren{(1-\EoR) + q\pr}
			}.%
	\label{eq:ss-rf,mxyp-te}
\end{align}
Comparing with \eqref{eq:ss-rf,mxyp-t0}, 
we reassuringly recover the SS condition
as $\TE$ approaches $\TR$: 
\begin{align}
	\lim_{\TE \rightarrow \TR} \mxyp(\bmr, t_0 + \TE) 
		= \mxyp(\bmr, t_0 + \TR) 
		= \mxyp\prtz.
	\label{eq:ss-rf,mxyp-te,sanity}
\end{align}

The received signal is approximately proportional
to the integrated transverse magnetization 
over a voxel volume $\setV$. 
Further approximating the received signal
as directly proportional 
to the steady-state magnetization
(as is typical in RF pulse design)
requires both standard and nonstandard additional assumptions:
\begin{itemize}
	\item{%
		We first assume 
		that the signal is localized
		to a scale over which 
		there is minimal variation
		of $\mzero\pr$, $\To\pr$, and $\Tt\pr$,
		a common assumption
		in RF pulse design.
	}%
	\item{%
		We next assume 
		that $\phi(\bmr, t'; \tau)$ 
		is a slowly-varying function with $\bmr$
		over the voxel volume
		for all fixed time point pairs 
		$(\tau, t') \in \set{(\tau, t') | t_0 \le \tau \le t' \le t_0 + \TP}$. 
		In regions where through-voxel field-inhomogeneity gradients are reasonably small, 
		this can be accomplished by restricting excitation gradient amplitudes 
		from being too large 
		and by balancing the excitation gradients 
		to have zero net area.
	}%
	\item{%
		We last assume
		that $\phi(\bmr, t_0 + \TE; t_0 + \TP)$ 
		and $\phi(\bmr, t_0 + \TR; t_0)$ 
		also exhibit slow spatial variation. 
		If excitation gradients are balanced, 
		the former condition only further requires imaging gradients 
		to balance at the echo time
		as is typical. 
		However,
		the latter condition 
		requires \emph{all} gradients to be balanced, 
		and thus restricts the analysis hereafter 
		to unspoiled or weakly spoiled sequences.
	}%
\end{itemize}
With these additional assumptions,
the received steady-state signal
for a typically sized voxel
centered at position $\bmr$
is (to within constants):
\begin{align}
	s(\bmr,t_0+\TE) 
		&\propto 
			\int_{\setV\pr} \mxyp(\bmr, t_0+\TE) \dercubed \bmr'
			\label{eq:ss-rf,s}
			\\
		&\approx
			\abs{\setV\pr} \mxyp(\bmr, t_0+\TE),
			\label{eq:ss-rf,s-apprx}
\end{align}
where $\abs{\cdot}$ here denotes the volume of a set.
By including the contributions of previous (unspoiled) excitations, 
we have derived a \emph{nonlinear} dependence
of SS signal on small-excitation RF fields.
Exploiting this nonlinear dependence
will require specialized algorithms
to efficiently design SS-informed RF pulses 
and/or excitation gradients.

Conventional small-excitation RF pulse design
utilizes an assumption
that off-resonance phase
grows linearly in time
to develop convenient Fourier-type relations
between the excited magnetization
and the excitation $\bmk$-space trajectory.
Because the SS magnetization's dependence
on off-resonance phase is nonlinear,
a direct Fourier-type relation no longer applies.
However,
the integrals within 
SS magnetization \eqref{eq:ss-rf,mxyp-te}
may \emph{individually} be expressed 
as Fourier-type relations,
so the possibility remains
for signal model implementation
using multiple Fourier transforms.
With the affine off-resonance assumption
$\bzp\prt \approx \frac{\omp\pr}{\gam} + \bmr \cdot \bmg\pt$
the integral in the numerator 
of \eqref{eq:ss-rf,mxyp-te}
describes the conventional Fourier-type relation \cite{pauly:89:aks},
where $\omp\pr$ denotes off-resonance frequency
and $\bmg\pt$ denotes a linear excitation gradient trajectory.
We next focus on developing a Fourier-type relation
for the integral in the denominator 
of \eqref{eq:ss-rf,mxyp-te},
contained within $q\pr$.
Defining \emph{k}-space trajectory 
$\bmk\pt := -\frac{\gam}{2\pi} \int_{t_0}^t \bmg(t') \der t'$, 
phase accrual $\phi(\bmr, t'; \tau)$ 
can be approximated as 
\begin{align}
	\phi(\bmr, t'; \tau) &\approx 
		-\omp\pr \paren{t' - \tau} 
		+ 2\pi \bmr \cdot \paren{\bmk(t') - \bmk(\tau)}.
	\label{eq:ss-rf,phi-apprx}
\end{align}
Substituting \eqref{eq:ss-rf,phi-apprx} 
into \eqref{eq:ss-rf,q} and simplifying yields
\begin{align}
	q\pr &\approx 
		\frac{\gam^2 \abs{\stx\pr}^2 }{2} \int_{t_0}^{t_0+\TP} \int_{t_0}^{t_0+\TP} 
		\bone(t') b'^*_{xy}(\tau) \zeta(\bmr, t'; \tau) e^{-2\pi i \bmr \cdot
		\paren{\bmk(t') - \bmk(\tau)}} \der \tau \der t' 
		\label{eq:ss-rf,q-zeta} 
		\\
	&= 
		\frac{\gam^2 \abs{\stx\pr}^2 }{2} \int_{\real^3} \int_{\real^3} 
		p(\bmr, \mathbf{k_1}, \mathbf{k_2}) e^{2\pi i \bmr \cdot (\bmk_1 + \bmk_2)} 
		\der \bmk_1 \der \bmk_2,
		\label{eq:ss-rf,q-fourier}
\end{align}
where $\zeta(\bmr, t', \tau)$ abbreviates 
relaxation and constant off-resonance effects as
\begin{align}
	\zeta(\bmr, t'; \tau) := 
		e^{-(\TR + t_0 - \max{(\tau, t'))}/\To\pr} e^{-\abs{t'-\tau}/\Tt\pr} 
		e^{+i\omp\pr \paren{t'-\tau}}
	\label{eq:ss-rf,zeta}
\end{align}
and $p(\bmr, \bmk_1, \bmk_2)$ defines a path in a six-dimensional $k$-space:
\begin{align}
	p(\bmr, \bmk_1, \bmk_2) := 
		\int_{t_0}^{t_0+\TP} \int_{t_0}^{t_0+\TP} \bone(t') 
		b'^*_{xy}(\tau) \zeta(\bmr, t'; \tau) \delta_3(\bmk(t') + \bmk_1) 
		\delta_3(\bmk(\tau) - \bmk_2) \der \tau \der t',
	\label{eq:ss-rf,path}
\end{align}
where $\delta_n\paren{\cdot}$ denotes 
an $n$-dimensional Dirac delta function. 
Eq.~\eqref{eq:ss-rf,q-fourier} is not a true Fourier relation 
because $p(\bmr, \bmk_1, \bmk_2)$ 
depends on $\bmr$. 
In the special case where relaxation and spatial variation 
in $\omp\pr$ are neglected,
$\zeta(\bmr, t'; \tau)$ separates in $t',\tau$
and \eqref{eq:ss-rf,q-fourier} simplifies 
to a true Fourier transform 
for which fast methods (\eg, \cite{yip:05:irp}) 
to separately evaluate integrals 
over $\bmk_1,\bmk_2$ are available. 
In the general case, 
a similar Fourier relation could be constructed 
if $\zeta(\bmr, t'; \tau)$ were approximated 
with a ``rank''-$L$ tensor-product expansion 
of the form
\begin{align}
	\zeta(\bmr, t'; \tau) &\approx \sum_{l=1}^L u_l\pr v_l(t') \bar{v}_l(\tau), 
	\label{eq:ss-rf,zeta-apprx}
\end{align}
where $u_l\pr \in \real$ 
and $v_l\pt \in \complex$ 
are the $l$th spatial and temporal basis functions. 
Inserting \eqref{eq:ss-rf,zeta-apprx} 
into \eqref{eq:ss-rf,q-zeta} 
would then yield a true Fourier relation 
similar to the one in \cite{fessler:05:tbi}:
\begin{align}
	q\pr &\approx 
		\frac{\gam^2 \abs{\stx\pr}^2 }{2} \sum_{l=1}^L u_l\pr 
		\abs{ \int_{t_0}^{t_0+\TP} v_l(t') \bone(t') e^{-2\pi i \bmr \cdot \bmk(t')} 
		\der t'}^2 
		\nonumber \\
	&= 
		\frac{\gam^2 \abs{\stx\pr}^2 }{2} \sum_{l=1}^L u_l\pr 
		\abs{ \mathcal{F}^{-1}_{3,\bmk} \brac{\int_{t_0}^{t_0+\TP} v_l(t') 
		\bone(t') \delta_3(\bmk(t') + \bmk) \der t'} \paren{\bmr}}^2,
		\label{eq:ss-rf,q-apprx}
\end{align}
where $\mathcal{F}^{-1}_{3,\bmk}\brac{\cdot}\pr$ denotes
an $3$-dimensional inverse Fourier transform
with respect to $\bmk$
and whose output is a function of $\bmr$.
Unfortunately, 
\eqref{eq:ss-rf,zeta-apprx} is a \emph{partially-complex} tensor-product expansion, 
and would require a tailored decomposition algorithm. 
Alternately,
we could expand 
the pure-real relaxation and complex off-resonance terms 
of $\zeta(\bmr, t'; \tau)$ separately:
\begin{align}
	\zeta(\bmr, t'; \tau) &=: 
		\xi(\bmr, t'; \tau) e^{+i\omp\pr t'} e^{-i\omp\pr \tau} 
		\label{eq:ss-rf,xi} \\
	&\approx 
		\paren{\sum_{l=1}^L u_l\pr v_l(t') v_l(\tau)} 
		\paren{\sum_{k=1}^{K} u'_k\pr v'_k(t')} 
		\paren{\sum_{k'=1}^{K} \bar{u}'_{k'}\pr \bar{v}'_{k'}(\tau)},
		\label{eq:ss-rf,xi-apprx}
	\end{align}	
where now $u_l\pr, v_l\pt \in \real$ 
and $u'_k\pr, v'_k\pt \in \complex$. 
Inserting \eqref{eq:ss-rf,xi-apprx}
into \eqref{eq:ss-rf,q-zeta} would yield
	\begin{align}
		q\pr &\approx 
			\frac{\gam^2 \abs{\stx\pr}^2 }{2} 
			\sum_{l=1}^L u_l\pr \abs{\sum_{k=1}^K u'_k\pr 
			\int_{t_0}^{t_0+\TP} v_l(t') v'_k(t') \bone(t') 
			e^{-2\pi i \bmr \cdot \bmk(t')} \der t'}^2 
			\nonumber 
			\\
		&= 
			\frac{\gam^2 \abs{\stx\pr}^2 }{2} 
			\sum_{l=1}^L u_l\pr \abs{\sum_{k=1}^K u'_k\pr 
			\mathcal{F}^{-1}_{3,\bmk} \brac{\int_{t_0}^{t_0+\TP} v_l(t') v'_k(t') 
			\bone(t') \delta_3(\bmk(t') + \bmk) \der t'} \paren{\bmr}}^2. 
			\label{eq:ss-rf,q-apprx,alt}
	\end{align}
Though \eqref{eq:ss-rf,xi} simplifies the tensor product to work over $\real$ only, 
its tensor decomposition of $\xi(\bmr, t'; \tau)$
may still be challenging in practice.
For ease of exposition 
in this early work, 
we revert to partially-complex expansion \eqref{eq:ss-rf,zeta-apprx} 
and choose convenient (complex) temporal basis functions 
$v_l\pt$ for $l = 1, \dots, L$, 
as in prior works \cite{noll:91:ahc,sutton:03:fii}. 
We then propose to estimate pure-real spatial basis functions 
$u_l\pt$ for $l = 1, \dots, L$ 
via linear least-squares.

We next discretize SS signal model \eqref{eq:ss-rf,mxyp-te}
for computer-aided SS-informed RF pulse design.
Let $\bmb := \brac{\bone(\To), \dots, \bone(t_J)}\tpose \in \complexes{J}$
discretize the excitation waveform into $J$ timepoints
from $t\gets t_0$ 
to $t \gets t_0 + \TP$.
Let 
\begin{align}
	\mathbf{m} &:= 
		\brac{\frac{s(\bmr_1, t_0+\TE)}{\mzero(\bmr_1)\abs{\setV\paren{\bmr_1}}}, 
		\dots, 
		\frac{s(\bmr_N, t_0+\TE)}{\mzero(\bmr_N)\abs{\setV\paren{\bmr_1}}}}\tpose 
		\in \complexes{N}
	\label{eq:ss-rf,m}
\end{align}
discretize the (unitless) relative magnetization
into $N$ voxels, respectively. 
Let 
\begin{align}
	a_{nj} &:=
		\frac{i (\gam \stx(\bmr_n) \Delta t)  
		(1-\Eo(\bmr_n)) e^{i \phi(\bmr_n, t_0 + \TE; t_j)} e^{-(t_0+\TE-t_j)/\Tt(\bmr_n)}}
		{\paren{1 - \Et(\bmr_n) e^{i \phi(t_0 + \TR; t_0)}}}; 
		\label{eq:ss-rf,anj} 
		\\
	d_{n{n'}} &:=
		\delta[n - n'] \paren{1 - \Eo(\bmr_n)}; 
		\label{eq:ss-rf,dnn} \\
	s_{j{j'}n} &:=	
		\frac{\abs{\gam \stx(\bmr_n) \Delta t}^2}{2} 
		\zeta(\bmr_n, t_j, t_{j'}) e^{-2\pi i \bmr_n \cdot 
		(\bmk(t_{j'}) - \bmk(t_j))} 
		\nonumber 
		\\
	&\approx
		\frac{\abs{\gam \stx(\bmr_n) \Delta t}^2}{2} 
		\sum_{l=1}^L u_l(\bmr_n) v_l(t_{j'}) v_l^*(t_j) 
		e^{-2\pi i\bmr_n \cdot (\bmk(t_{j'}) - \bmk(t_j))},
		\label{eq:ss-rf,sjjn}
\end{align}
respectively denote scalar elements 
of the matrices 
$\bmA \in \complexes{N \times J}$;
$\bmD \in \reals{N \times N}$;
and tensor $\mathcal{S} \in \complexes{J \times J \times N}$.
Here $\Delta t := \TP/J$ denotes the RF pulse sampling interval
and $\delta\brac{\cdot}$ denotes the Kronecker delta function.
Then the noiseless discretized signal model reads
\begin{align}
	\mathbf{m} &= 
		\inv{\bmD + \diag{\brac{\bmb\ctpose\bmS_1\bmb,\dots,\bmb\ctpose\bmS_N\bmb}\tpose}}
		\bmA \bmb,
	\label{eq:ss-rf,discrete}
\end{align}
where $\bmS_n \in \complexes{J \times J}$
denotes the $n$th layer of $\mathcal{S}$
for each $n \in \set{1,\dots,N}$.
	
%%%%%%%%%%%%%%%%%%%%%%%%%%%%%%%%%%%%%%%%%%%%%%%%%%%
\section{Two Iterative Algorithms}
\label{s,ss-rf,alg}
%%%%%%%%%%%%%%%%%%%%%%%%%%%%%%%%%%%%%%%%%%%%%%%%%%%	

This section develops two iterative algorithms
for designing SS-informed RF pulses.
Subsection~\ref{ss,ss-rf,alg,pert}
employs a perturbative expansion approach
that takes inspiration 
from a similar algorithm 
for large-tip angle RF pulse design \cite{grissom:09:flt}.
Subsection~\ref{ss,ss-rf,alg,admm}
employs an optimization approach
that uses variable splitting 
and the alternating direction 
method of multipliers (ADMM) algorithm \cite{gabay:76:ada}.
Both subsections focus 
on designing the RF pulse only
and assume the excitation gradient trajectory is fixed. 

%%%%%%%%%%%%%%%%%%%%%%%%%%%%%%%%%%%%%%%%%%%%%%%%%%%	
\subsection{Perturbative Expansions}
\label{ss,ss-rf,alg,pert}

Given an RF pulse design 
at the $i$th iteration $\iter{\bmb}{i}$, 
one can predict the magnetization $\iter{\bmm}{i}$ 
via either \eqref{eq:ss-rf,discrete} or Bloch simulations. 
If the deviation $\Delta \iter{\bmm}{i+1} := \mathbf{m} - \iter{\bmm}{i}$ 
from the desired pattern is small, 
it should be possible 
to reduce the subsequent deviation 
with small perturbation $\Delta \iter{\bmb}{i+1}$ to $\iter{\bmb}{i}$. 
We can then iteratively improve the RF pulse 
by accumulating small perturbations, \ie  
$\iter{\bmb}{i+1} \gets \iter{\bmb}{i} + \Delta \iter{\bmb}{i+1} 
	\equiv \iter{\bmb}{0} + \sum_{\iota=1}^{i+1} \Delta \iter{\bmb}{\iota}$.

Here we apply a first-order perturbative expansion 
to approximate how the $(i+1)$th incremental excitation 
$\Delta \iter{\bmb}{i+1}$ 
relates to the $(i+1)$th pattern deviation 
$\Delta \iter{\bmm}{i+1}$: 
	\begin{align}
		\mathbf{A} \paren{\iter{\bmb}{i} + \Delta \iter{\bmb}{i+1}} &= 
			\paren{\bmD + \diagn{\paren{\iter{\bmb}{i} 
			+ \Delta \iter{\bmb}{i+1}}\ctpose \bmS_n \paren{\iter{\bmb}{i} 
			+ \Delta \iter{\bmb}{i+1}}}}
			\paren{\iter{\bmm}{i} + \Delta \iter{\bmm}{i+1}} 
			\label{eq:ss-rf,pert-ex} 
			\\
		&\approx 
			\paren{\bmD + \diagn{(\iter{\bmb}{i})\ctpose \bmS_n \iter{\bmb}{i}}}
			\paren{\iter{\bmm}{i} + \Delta \iter{\bmm}{i+1}} 
			\nonumber 
			\\
		&+ 
			2 \diagn{\re{(\iter{\bmb}{i})\ctpose 
			\bmS_n \Delta \iter{\bmb}{i+1}}} \iter{\bmm}{i}, 
			\label{eq:ss-rf,pert-apprx}
	\end{align}
where $\diagn{\bmb\ctpose\bmS_n\bmb'} :=
	\diag{\brac{\bmb\ctpose\bmS_1\bmb',\dots,\bmb\ctpose\bmS_n\bmb'}\tpose}$
and the approximation drops all higher-order terms.
Canceling out zeroth-order terms 
\begin{align}
	\bmA \iter{\bmb}{i} \approx
		\paren{\bmD + \diagn{(\iter{\bmb}{i})\ctpose\bmS_n\iter{\bmb}{i}}} \iter{\bmm}{i}
\end{align}
that either are exactly equal 
(if $\iter{\bmm}{i}$ is computed via \eqref{eq:ss-rf,discrete})
or nearly equal
(if $\iter{\bmm}{i}$ is computed via Bloch simulations)
and rearranging yields the update
\begin{align}
	\re{\Delta\iter{\bmb}{i+1}} &\gets
		\pinv{\bmA-2\diag{\iter{\bmm}{i}}\re{\iter{\bmG}{i}}}
		\paren{\bmD + \diagn{(\iter{\bmb}{i})\ctpose\bmS_n\iter{\bmb}{i}}}
		\Delta \iter{\bmm}{i+1};
		\nonumber
		\\
	\im{\Delta\iter{\bmb}{i+1}} &\gets
		\pinv{i\bmA+2\diag{\iter{\bmm}{i}}\im{\iter{\bmG}{i}}}
		\paren{\bmD + \diagn{(\iter{\bmb}{i})\ctpose\bmS_n\iter{\bmb}{i}}}
		\Delta \iter{\bmm}{i+1},
	\label{eq:ss-rf,pert-update}
\end{align}
where $\iter{\bmG}{i} := 
	\brac{\bmS_1\iter{\bmb}{i},\dots,\bmS_n\iter{\bmb}{i}}\ctpose 
	\in \complexes{N \times J}$.
Each perturbative correction update
requires solving two size-$J$ linear least-squares problems.

%%%%%%%%%%%%%%%%%%%%%%%%%%%%%%%%%%%%%%%%%%%%%%%%%%%	
\subsection{Variable Splitting and ADMM}
\label{ss,ss-rf,alg,admm}
	
Alternately,
we can approach SS-informed RF pulse design
by solving a suitable optimization problem.
We would like to solve
the non-convex optimization problem
\begin{align}
	\est{\bmb} &\in \set{%
		\argmin{\bmb \in \complexes{J}}
		\norm{\inv{\bmD + \diagn{\bmb\ctpose\bmS_n\bmb}}\bmA\bmb-\bmm}^2_{\bmW}
		+ \beta\norm{\bmC\bmb}^2_2
	}%
	\label{eq:ss-rf,prob-orig}
\end{align}
where $\bmW \in \reals{N \times N}$ denotes a spatial weighting matrix;
$\bmC$ is a pure-real first-order finite differencing matrix;
and $\beta \in \real$ is a regularization parameter.
Since solving \eqref{eq:ss-rf,prob-orig} is challenging,
we instead study a related constrained problem:
\begin{alignat}{3}
	&\paren{\est{\bmb},\est{\bmz}_1,\est{\bmz}_2} 
		&&\in
			\set{\argmin{\bmb,\bmz_1,\bmz_2\in\complexes{J}} \costa{\bmb,\bmz_1,\bmz_2}}
		&&\text{subject to}
		\label{eq:ss-rf,prob} \\
	&\bmz_1
		&&= \bmf(\bmb,\bmz_2)+\bmd
		&&\text{and}
		\nonumber \\
	&\bmz_2
		&&= \bmb,
		&&\text{where}
		\nonumber \\
	&\costa{\bmb,\bmz_1,\bmz_2}
		&&:= \norm{\bmA\bmb-\diag{\bmm}\bmz_1}^2_{\bmW}+\beta\norm{\bmC\bmb}^2_2;
		\label{eq:ss-rf,cost}
\end{alignat}
$\bmf(\bmb,\bmb') := \brac{\bmb\ctpose\bmS_1\bmb',\dots,\bmb\ctpose\bmS_N\bmb'}$
evaluates $\bmf : \complexes{J}\times\complexes{J}\mapsto\complexes{N}$
for arbitrary $\bmb,\bmb'$;
and $\bmd := \bmD \ones{N} \in \reals{N}$.
We solve \eqref{eq:ss-rf,prob} via ADMM \cite{gabay:76:ada}.
The augmented Lagrangian is
\begin{alignat}{2}
	&\Lambda(\bmb,\bmz_1,\bmz_2,\bmnu_1,\bmnu_2) := \Psi(\bmb, \bmz_1, \bmz_2) 
		&&+ \re{\bmnu_1\ctpose \paren{\bmf(\bmb,\bmz_2) + \mathbf{d}-\bmz_1}} 
			+ \frac{\rho_1}{2} \norm{\bmf(\bmb,\bmz_2) + \mathbf{d} - \bmz_1}^2_2
		\nonumber 
		\\
	& 
		&&+ \re{\bmnu_2\ctpose \paren{\bmb - \bmz_2}} 
			+ \frac{\rho_2}{2} \norm{\bmb - \bmz_2}^2_2,
		\label{eq:ss-rf,lagrangian-nu}
\end{alignat}
where $\bmnu_1\in\complexes{N}$ and $\bmnu_2\in\complexes{J}$ 
are dual variables
and $\rho_1,\rho_2>0$ are constraint penalty parameters.
Rescaling the dual variables 
as $\bmu_1 := \frac{\bmnu_1}{\rho_1}$ 
and $\bmu_2 := \frac{\bmnu_2}{\rho_2}$, 
completing the square,
and dropping constants reveals 
an alternate but equivalent form 
of the augmented Lagrangian 
that leads to simpler variable updates:
\begin{alignat}{2}
	&\Lambda'(\bmb, \bmz_1,\bmz_2,\bmu_1,\bmu_2) := \Psi(\bmb,\bmz_1,\bmz_2) 
		&&+ \frac{\rho_1}{2}\norm{\bmf(\bmb, \bmz_2) + \mathbf{d}-\bmz_1+\bmu_1}^2_2 			
		- \frac{\rho_1}{2} \norm{\bmu_1}^2_2 
		\nonumber 
		\\
	& 
		&&+ \frac{\rho_2}{2} \norm{\bmb - \bmz_2 + \bmu_2}^2_2 
		- \frac{\rho_2}{2} \norm{\bmu_2}^2_2. \label{eq:ss-rf,lagrangian-u}
\end{alignat}
ADMM cycles through updating the primal variables $\bmb,\bmz_1,\bmz_2$
followed by gradient ascent on the scaled dual variables $\bmu_1,\bmu_2$, 
holding other variables fixed from previous iterations. 
Though $\rho_1$ and $\rho_2$ influence (local) convergence rates, 
they do not affect the solution. 
We next describe each of these updates.
\begin{enumerate}
	\item{%
		The $\bmb$ update involves a size-$J$ linear least-squares subproblem:
		\begin{alignat}{2}
			\iter{\bmb}{i+1} 
				&\gets \argmin{\bmb\in\complexes{J}}
					\Lambda'(\bmb,\iter{\bmz}{i}_1,\iter{\bmz}{i}_2,
						\iter{\bmu}{i}_1,\iter{\bmu}{i}_2)
						\nonumber \\
				&= \argmin{\bmb\in\complexes{J}}
					\costa{\bmb,\iter{\bmz}{i}_1,\iter{\bmz}{i}_2}
					&&+ \frac{\rho_1}{2}\norm{%
							\bmf(\bmb,\iter{\bmz}{i}_2)-\paren{\iter{\bmz}{i}_1
							-\bmd-\iter{\bmu}{i}_1}^2_2
						}%
						\nonumber \\
				&
					&&+ \frac{\rho_2}{2}\norm{\bmb-
						\paren{\iter{\bmz}{i}_2-\iter{\bmu}{i}_2}}^2_2.
						\label{eq:ss-rf,b-update}
		\end{alignat}
	}%
	\item{%
		The $\bmz_1$ update involves a trivial linear least-squares subproblem:
		\begin{align}
			\iter{\bmz}{i+1}_1 
				&\gets \argmin{\bmz_1\in\complexes{N}}
					\Lambda'(\iter{\bmb}{i+1},\bmz_1,\iter{\bmz}{i}_2,
						\iter{\bmu}{i}_1,\iter{\bmu}{i}_2)
						\nonumber \\
				&= \argmin{\bmz_1\in\complexes{N}}
					\norm{\diag{\bmm}\bmz_1-\bmA\iter{\bmb}{i+1}}^2_\bmW
					+ \frac{\rho_1}{2}\norm{%
						\bmz_1-\paren{\bmf(\iter{\bmb}{i+1},\iter{\bmz}{i}_2)
							+ \bmd + \iter{\bmu}{i}_1}
						}^2_2.
						\label{eq:ss-rf,z1-update}
		\end{align}
	}%
	\item{%
		The $\bmz_2$ update involves another size-$J$ linear least-squares subproblem:
		\begin{align}
			\iter{\bmz}{i+1}_2
				&\gets \argmin{\bmz_2\in\complexes{J}}
					\Lambda'(\iter{\bmb}{i+1},\iter{\bmz}{i+1}_1,\bmz_2,
						\iter{\bmu}{i}_1,\iter{\bmu}{i}_2)
						\nonumber \\
				&= \argmin{\bmz_2\in\complexes{J}}
					\frac{\rho_1}{2}\norm{%
						\bmf(\iter{\bmb}{i+1},\bmz_2)
						-\paren{\iter{\bmz}{i+1}_1-\bmd-\iter{\bmu}{i}_1}
					}^2_2
					+\frac{\rho_2}{2}\norm{\bmz_2-\paren{\iter{\bmb}{i+1}+\iter{\bmu}{i}_2}}^2_2.
					\label{eq:ss-rf,z2-update}
		\end{align}	
	}%
	\item{%
		The scaled dual variables $\bmu_1,\bmu_2$ 
		are updated via one gradient ascent iteration:
		\begin{align}
			\iter{\bmu}{i+1}_1 
				&\gets \iter{\bmu}{i}_1 + \rho_1 
					\paren{\bmf(\iter{\bmb}{i+1},\iter{\bmz}{i+1}_2)+\bmd-\iter{\bmz}{i+1}_1};
					\label{eq:ss-rf,u1-update}
					\\
			\iter{\bmu}{i+1}_2 
				&\gets \iter{\bmu}{i}_2 + \rho_2
					\paren{\iter{\bmb}{i+1}-\iter{\bmz}{i+1}_2}.
					\label{eq:ss-rf,u2-update}
		\end{align}
	}%
\end{enumerate}
If a saddle point local to the initialization exists,
ADMM will converge to that saddle point
and the solution will correspond
to a local solution 
to constrained problem \eqref{eq:ss-rf,prob}.
