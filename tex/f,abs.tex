% abstract
\setlength{\parindent}{0ex}
Quantitative magnetic resonance imaging (QMRI)
produces images of \edit{potential} MR biomarkers: 
measurable tissue properties
related to physiological processes
that characterize the onset and progression
of specific disorders.
Though QMRI has potential 
to be more \edit{diagnostic} 
than conventional qualitative MRI,
QMRI poses challenges 
beyond those of conventional MRI
that limit its feasibility 
for routine clinical use.
This thesis first seeks to address 
two of these challenges.
It then applies these solutions
to develop a new method
for myelin water imaging,
a challenging application 
that may be specifically indicative
of certain white matter disorders. 

\setlength{\parindent}{4ex}
One challenge 
that presently precludes widespread clinical adoption of QMRI
involves relatively long scan durations:
to disentangle potential biomarkers
from numerous nuisance MR contrast mechanisms,
QMRI typically requires more data than conventional MRI
and thus longer scans. 
Even allowing for long scans,
it has previously been unclear 
how to systematically tune the ``knobs'' 
of highly flexible MR acquisitions
so as to reliably enable precise biomarker estimation.
Chapter~\ref{c,scn-dsgn} formalizes these challenges
as a min-max optimal acquisition design problem
\edit{%
	that seeks scan parameter combinations
	that robustly enable precise object parameter estimation.
	It applies this technique
	to optimize combinations
	of spoiled gradient-recalled echo (SPGR) 
	and dual-echo steady-state (DESS) scans
	for $\To,\Tt$ relaxometry
	in white matter (WM) and grey matter (GM) regions
	of the human brain at 3T field strength.
	Phantom accuracy experiments showed
	that SPGR/DESS scan combinations
	are in excellent agreement 
	with reference measurements.
	Phantom precision experiments showed
	that trends in $\To,\Tt$ pooled sample standard deviations
	reflect theoretical predictions.
	\Invivo experiments showed
	that in WM and GM,
	$\To,\Tt$ estimates
	from a pair of optimized DESS scans
	exhibit precision comparable 
	to that of optimized combinations 
	of SPGR and DESS scans.
	To our knowledge,
	$\To$ maps from DESS acquisitions
	are new.
	This example application illustrates
	that acquisition design can enable 
	new biomarker estimation techniques
	from established MR pulse sequences,
	a fact that subsequent chapters exploit.
}%

Another QMRI challenge
involves the typically nonlinear dependence 
of MR signal models
on the underlying potential biomarkers of interest:
these nonlinearities 
cause conventional likelihood-based estimators
to either scale very poorly 
with the number of unknowns
or risk producing suboptimal estimates
\edit{%
	due to spurious local minima.
	Chapter~\ref{c,perk} instead introduces
	a fast, general method 
	for dictionary-free QMRI parameter estimation
	via regression with kernels (PERK).
	PERK first uses prior distributions 
	and the nonlinear MR signal model 
	to simulate many parameter-measurement pairs.
	Inspired by machine learning,
	PERK then takes these parameter-measurement pairs
	as labeled training points
	and learns from them a nonlinear regression function
	using kernel functions and convex optimization.
	PERK admits a simple implementation
	as per-voxel nonlinear lifting
	of MRI measurements 
	followed by linear minimum mean-squared error regression.
	Chapter~\ref{c,perk} demonstrates PERK
	for $\To,\Tt$ estimation
	using one of the SPGR/DESS acquisitions
	optimized in Chapter~\ref{c,scn-dsgn}.
	Numerical simulations 
	as well as single-slice phantom and \invivo experiments
	demonstrated that PERK 
	and two well-suited maximum-likelihood (ML) estimators
	produce comparable $\To,\Tt$ estimates in WM and GM,
	but PERK is consistently at least $140\times$ faster.
	Similar comparisons to an ML estimator  
	in a more challenging QMRI estimation problem
	(described in Chapter~\ref{c,mwf})
	suggest that this $140\times$ acceleration factor
	will increase by several orders of magnitude
	for full-volume QMRI estimation problems
	involving more latent parameters per voxel.
}%

Chapter~\ref{c,mwf} applies ideas 
developed in previous chapters
to design a new fast method
for imaging myelin water content,
a \edit{potential} biomarker for healthy myelin.
Since myelin degeneration characterizes
certain white matter disorders
(\eg, multiple sclerosis),
myelin water quantification
could improve MRI specificity
for monitoring the onset and progression
of such demyelinating conditions.
\edit{%
	Specifically,
	Chapter~\ref{c,mwf} first develops 
	a two-compartment DESS signal model
	and then uses a Bayesian variation 
	of acquisition design (Chapter~\ref{c,scn-dsgn})
	to optimize a new DESS acquisition
	for precise myelin water imaging.
	The precision-optimized acquisition
	is as fast as conventional SS myelin water imaging acquisitions,
	but enables $\sim$2-3$\times$ better
	expected coefficients of variation
	in fast-relaxing fraction $\ff$ estimates.
	Simulations without model mismatch demonstrate
	that PERK (Chapter~\ref{c,perk}) and ML $\ff$ estimates
	from the proposed DESS acquisition
	exhibit comparable root mean-squared errors,
	but PERK is more than $500\times$ faster.
	Simulations with modest levels of model mismatch suggest
	that \invivo differences between DESS 
	and conventional multi-echo spin-echo (MESE) 
	myelin water images 
	may be attributable 
	to either unaccounted bulk-$\To$ modeling errors
	and accounted flip angle variation 
	in MESE estimates 
	and/or 
	the two-compartment assumption 
	in DESS estimates.
	Nevertheless,
	\invivo experiments are to our knowledge the first
	to demonstrate lateral WM myelin water content estimates
	from a fast (3m15s) SS acquisition
	that are similar 
	to conventional estimates 
	from a slower (32m4s) MESE acquisition.
}%
