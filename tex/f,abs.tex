% abstract
\setlength{\parindent}{0ex}
Quantitative magnetic resonance imaging (QMRI)
produces images of MR biomarkers: 
measurable tissue properties
related to physiological processes
that characterize the onset and progression
of specific disorders.
Though QMRI has potential 
to be more informative 
than conventional qualitative MRI,
QMRI poses challenges 
beyond those of conventional MRI
that currently limit its feasibility 
for routine clinical use.
This thesis first seeks to address 
two of these challenges.
It then applies these solutions
to develop a new method
for myelin water imaging,
a challenging application 
that may be specifically indicative
of certain white matter disorders. 

\setlength{\parindent}{4ex}
One challenge 
that presently precludes widespread clinical adoption of QMRI
involves relatively long scan durations:
to disentangle biomarkers
from numerous nuisance MR contrast mechanisms,
QMRI typically requires more data than conventional MRI
and thus longer scans. 
Even allowing for long scans,
it has previously been unclear 
how to systematically tune the ``knobs'' 
of highly flexible MR acquisitions
so as to reliably enable precise biomarker estimation.
Chapter~\ref{c,scn-dsgn} formalizes these challenges
as an optimal acquisition design problem
and solves this problem for MR relaxometry,
a simple and popular application.
The resulting optimized acquisition designs
enable MR relaxometry
in much less time 
than conventional relaxometry acquisitions, 
but perhaps more importantly illustrate
that acquisition design can enable 
new biomarker estimation techniques
from established MR pulse sequences,
a fact that subsequent chapters exploit.

Another QMRI challenge
involves the typically nonlinear dependence 
of MR signal models
on the underlying biomarkers of interest:
these nonlinearities 
cause conventional biomarker estimators
to either scale very poorly 
with the number of unknowns
or risk producing suboptimal estimates.
Chapter~\ref{c,perk} instead formulates
this challenging parameter estimation problem
as supervised learning problem
that admits a very efficient solution.
With proper training,
simulations and experiments 
demonstrate orders-of-magnitude acceleration
over conventional estimators,
with comparable accuracy and precision.

Chapter~\ref{c,mwf} applies ideas 
developed in previous chapters
to design a new fast method
for imaging myelin water content,
a biomarker for healthy myelin.
Since myelin degeneration characterizes
certain white matter disorders
(\ie, multiple sclerosis),
myelin water quantification
could improve MRI specificity
for monitoring the onset and progression
of such demyelinating conditions.
The tools developed in this thesis enable 
whole-brain high-resolution myelin water imaging 
in about 10 minutes of scan time,
whereas state-of-the-art methods
require 30 minutes or more.
