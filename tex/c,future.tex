% concluding remarks

This chapter suggests avenues
for further QMRI research,
focusing on relatively broad future directions
that this thesis has not explored.
Discussion sections
within main body chapters
offer more focused ideas 
for topic-specific extensions,
whereas the appendices organize 
partially investigated but still immature topics
that are less related to QMRI.

%%%%%%%%%%%%%%%%%%%%%%%%%%%%%%%%%%%%%%%%%%%%%%%%%%%
\section{%
	\edit{%
		Combining PERK with Image Reconstruction%
	}%
}%
\label{s,future,recon}
%%%%%%%%%%%%%%%%%%%%%%%%%%%%%%%%%%%%%%%%%%%%%%%%%%%

This thesis has considered parameter estimation
to be separate from image reconstruction.
This separation affords fast data processing
but may leave room for improved estimation performance,
especially when raw data is undersampled.
One could instead seek 
to estimate parameters 
directly from raw data.
One approach 
to combining image reconstruction
with PERK-based estimation
might seek to solve
the joint optimization problem
\begin{align}
	\paren{\est{\bmX},\est{\bmY}} &\in \set{%
		\argmin{%
			\substack{%
				\bmX \in \complexes{L \times V} \\
				\bmY \in \complexes{D \times V} \\
			}%
		}%
		\frob{\bmD-\bmY\bmA}^2 
			+ \beta \sum_{v=1}^V 
				\norm{\bmh\paren{\bmy_v,\bmnu_v}+\bmb-\bmx_v}^2_2
	},
	\label{eq:future,recon}
\end{align}
where 
$\bmX := \brac{\bmx_1,\dots,\bmx_V} \in \complexes{L \times V}$
and
$\bmY := \brac{\bmy_1,\dots,\bmy_V} \in \complexes{D \times V}$
respectively collect $L$ latent parameters 
and $D$ image datasets 
at $V$ voxels;
$\bmh : \complexes{D+N} \mapsto \complexes{L}$ 
and $\bmb \in \complexes{L}$ 
together denote a pre-trained PERK regression function with offset;
$\bmD \in \complexes{D \times K}$
collects $D$ raw $\bmk$-space datasets 
each acquired with $K$ samples;
$\bmA \in \complexes{V \times K}$ 
denotes the MRI system matrix
(that models receive coil sensitivity spatial variation 
and $\bmk$-space sampling);
$\bmnu_v$ denotes a known parameter at the $v$th voxel;
and $\beta$ is a free parameter
that balances cost function terms.
Here, 
the first term enforces 
image fidelity to raw data
and the second term entangles 
image reconstruction and parameter estimation.

For continuously differentiable kernels,
\eqref{eq:future,recon} is amenable 
to iterative local optimization 
via alternating minimization.
One simple algorithm iterates
the following updates:
\begin{align}
	\iter{\bmX}{i+1} &\gets
		\brac{%
			\bmh\paren{\iter{\bmy}{i}_1,\bmnu_1}+\bmb,
			\dots,
			\bmh\paren{\iter{\bmy}{i}_V,\bmnu_1}+\bmb
		}%
		\label{eq:future,X-update}
		\\
	\iter{\bmY}{i+1} &\gets
		\argmin{%
			\bmY \in \complexes{D \times V}
		}%
		\frob{\bmD-\bmY\bmA}^2 + 
			\beta \sum_{v=1}^V \norm{%
				\bmh\paren{\bmy_v,\bmnu_v}+\bmb-\iter{\bmx}{i+1}_v
			}^2_2%
		\label{eq:future,Y-update}		
\end{align}
where $\iter{\paren{\cdot}}{i}$ denotes the $i$th iterate.
Latent parameter update \eqref{eq:future,X-update}
applies PERK voxel-wise
at the current image iterate.
Image update \eqref{eq:future,Y-update}
enforces consistency 
not only with data
but also with latent parameter iterates.
Locally solving the inner problem 
within \eqref{eq:future,Y-update}
via gradient-based optimization
requires (often simple) kernel gradients
but does not require signal model gradients.
Thus, 
\eqref{eq:future,recon} or similar variations
may be useful even when analytical signal models
are cumbersome or altogether unavailable.
 
%%%%%%%%%%%%%%%%%%%%%%%%%%%%%%%%%%%%%%%%%%%%%%%%%%%
\section{Exploiting Off-Resonance for Myelin Water Imaging}
\label{s,future,off-res}
%%%%%%%%%%%%%%%%%%%%%%%%%%%%%%%%%%%%%%%%%%%%%%%%%%%

Though early sections in Chapter~\ref{c,mwf} 
provided a simple model  
of off-resonance distribution variation 
across intravoxel compartments,
experiments therein 
ultimately used simpler magnitude signal models
that neglected off-resonance effects.
However,
off-resonance distributions
do often differ significantly 
across compartments
in cerebral tissue \cite{miller:10:aot-1,miller:10:aot-2},
so accounting for compartment-specific off-resonance effects
could aid in better distinguishing compartments
and could thereby enable further-improved myelin water imaging.
For designing off-resonance-informed myelin water imaging acquisitions,
it is reasonable
to consider pulse sequences
whose acquisition parameters
can strongly influence signal sensitivity 
to off-resonance effects.
In this respect,
the small-tip fast recovery (STFR) sequence \cite{nielsen:13:stf}
may be well-suited for off-resonance-informed myelin water imaging
because its tip-up pulse magnitude and phase
provide additional degrees of freedom
by which to sensitize acquisitions
to compartmental off-resonance effects.
Off-resonance-informed myelin water imaging
using (spoiled) STFR sequences
is an active area of research 
in our group.

%%%%%%%%%%%%%%%%%%%%%%%%%%%%%%%%%%%%%%%%%%%%%%%%%%%
\section{Correlating with Other Myelin Biomarkers}
\label{s,future,myelin}
%%%%%%%%%%%%%%%%%%%%%%%%%%%%%%%%%%%%%%%%%%%%%%%%%%%

As discussed in Section~\ref{s,mwf,disc},
fundamental differences 
between DESS $\ff$ and MESE MWF imaging
may limit their quantitative comparability.
To build evidence 
that DESS $\ff$ imaging
is nevertheless a specific biomarker
for intact myelin content,
we are also interested 
in how $\ff$ correlates 
with other myelin biomarkers.
One contending noninvasive marker arises
from MR pulse sequences sensitized
to the inhomogeneous magnetization transfer (ihMT) effect
\cite{varma:15:mtf},
which has recently been shown
to be specific
to the large membrane lipids
that comprise much of myelin
\cite{varma:15:iom, swanson:17:mda}.
Multi-compartmental and ihHT MRI markers 
could be successively compared 
through \exvivo, healthy volunteer, and patient studies.
Outside MRI,
invasive measurements
from histology
have been used to study myelin 
(as in \eg, \cite{gareau:00:mta, webb:03:imt})
and could serve as a gold-standard \insitu marker.
MRI and histological markers
could be compared 
by correlating respective \exvivo and \insitu studies.
