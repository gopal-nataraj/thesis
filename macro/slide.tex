% colors
\definecolor{mich-blue}{HTML}{00274C}
\definecolor{mich-maize}{HTML}{FFCB05}
\definecolor{law-stone}{HTML}{655A52}
\definecolor{burton-beige}{HTML}{9B9A9D}
\definecolor{arch-ivy}{HTML}{7E732F}

% use colors
\setbeamercolor{title}{
  fg=mich-blue!90!white
}
\setbeamercolor{frametitle}{
  bg=mich-blue,
  fg=mich-maize
}
\setbeamercolor{normal text}{
  bg=white
}

% colored citations
\newcommand{\citec}[1]{%
  \textcolor{arch-ivy}{\cite{#1}}%
}

% colored highlighting
\newcommand{\hlo}[1]{%
  \textcolor{darkorange}{#1}%
}
\newcommand{\hlb}[1]{%
  \textcolor{blue}{#1}%
}
\newcommand{\hlg}[1]{%
  \textcolor{forestgreen}{#1}%
}
\newcommand{\hlv}[1]{%
  \textcolor{violet}{#1}%
}
\newcommand{\hlm}[1]{%
  \textcolor{magenta}{#1}%
}

% invisible citations
\newcommand{\citei}[1]{
  \textcolor{white}{\cite{#1}}
}
	
% boxed items
\newcommand*{\boxedcolor}{green}
\makeatletter
\newcommand{\cbox}[1]{\textcolor{\boxedcolor}{%
  \fbox{\normalcolor\m@th$\displaystyle#1$}}}
\makeatother

% pos/neg bullets
\newcommand{\gch}{\textcolor{forestgreen}{\Checkmark}}
\newcommand{\rx}{\textcolor{red}{\XSolidBrush}}

% translucent text
\newcommand{\tluc}[2][35]{\color{fg!#1}#2}

% covered content invisible before revealed
% covered content translucent after revealed
\setbeamercovered{
  still covered={\opaqueness<1->{0}},
  again covered={\opaqueness<1->{30}}
}

% tiny references
\setbeamerfont{bibliography item}{size=\tiny}
\setbeamerfont{bibliography entry author}{size=\tiny}
\setbeamerfont{bibliography entry title}{size=\tiny}
\setbeamerfont{bibliography entry location}{size=\tiny}
\setbeamerfont{bibliography entry note}{size=\tiny}

